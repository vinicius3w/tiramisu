%% Garcia's latex template for Activities Drescriptive Memorial Report
%% Version 0.2
%% (c) 2015 Vinicius Cardoso Garcia (vcg@cin.ufpe.br)
%%
%% This document is based on latex template from Prof. Daniel Cunha.
%%
%% Reference commands. Use the following commands to make references in your
%% text:
%%          \figref  -- for Figure reference
%%          \tabref  -- for Table reference
%%          \eqnref  -- for equation reference
%%          \chapref -- for chapter reference
%%          \secref  -- for section reference
%%          \appref  -- for appendix reference
%%          \axiref  -- for axiom reference
%%          \conjref -- for conjecture reference
%%          \defref  -- for definition reference
%%          \lemref  -- for lemma reference
%%          \theoref -- for theorem reference
%%          \corref  -- for corollary reference
%%          \propref -- for proprosition reference
%%          \pgref   -- for page reference
%%
%%          Example: See \chapref{chap:introduction}
%%%%%%%%%%%%%%%%%%%%%%%%%%%%%%%%%%%%%%%%%%%%%%%%%%%%%%%%%%%%%%%%%%%%%%%%%%%%%%%

\documentclass[a4paper,oneside,10pt]{article}

\usepackage{graphicx}
\usepackage{amsmath,amstext,amssymb,amsfonts}
\usepackage[none]{hyphenat}
\usepackage{fancyhdr}
\usepackage{cite}
\usepackage{indentfirst}
%\usepackage{path}
\usepackage[usenames, dvipsnames]{color}

\usepackage{pdfsync}
\usepackage{url}
\usepackage{setspace}
\usepackage[latin1,utf8]{inputenc}
\usepackage[T1]{fontenc}
\usepackage{colortbl}
\newcommand{\SetRowColor}[1]{\noalign{\gdef\RowColorName{#1}}\rowcolor{\RowColorName}}

\definecolor{MyRed}{rgb}{1,0.2,0.1}
\definecolor{light-gray}{gray}{0.95}
\definecolor{gray}{gray}{0.6}

\usepackage{booktabs}
\usepackage{ctable}
\usepackage{setspace}
\usepackage[multidot]{grffile}
\usepackage[final]{pdfpages}

\usepackage{titlesec}

\usepackage[brazil]{babel}

\usepackage[toc,page]{appendix}
\newcommand{\appref}[1]{\@appendixname~\ref{#1}\xspace}
%\usepackage{xifthen}

\usepackage[bookmarks,colorlinks,pdfpagelabels,
pdftitle={Memorial Descritivo de Atividades}, pdfauthor={Vinicius Cardoso Garcia},
pdfsubject={Solicita\c{c}\~{a}o de progress\~{a}o funcional docente de Adjunto N\'{\i}vel 3 para Adjunto N\'{\i}vel 4 apresentada \`{a} Comiss\~{a}o de Avalia\c{c}\~{a}o de Progress\~{a}o Horizontal do Centro de Inform\'{a}tica da Universidade Federal de Pernambuco.},
pdfcreator={Vinicius Cardoso Garcia}, pdfkeywords={Progressão, Adjunto, UFPE, CIn, Docente, Horizontal}]{hyperref}

\newcommand{\otoprule}{\midrule[\heavyrulewidth]}

% Definicao de margens
\setlength{\textwidth}{16cm} %
\setlength{\textheight}{23cm} %
\setlength{\oddsidemargin}{0cm} %
\setlength{\evensidemargin}{0cm} %
\setlength{\topmargin}{0cm}

\renewcommand{\abstractname}{Resumo}
\renewcommand{\contentsname}{\'{I}ndice Anal\'{\i}tico}
\renewcommand{\refname}{Refer\^{e}ncias}
\renewcommand{\appendixname}{Ap\^{e}ndice}
\renewcommand{\tablename}{Tabela}

%% Formatacao do cabecalho e rodape
\lhead{\footnotesize Memorial Descritivo de Atividades} %
\rhead{\footnotesize \emph{Vinicius Cardoso Garcia}} %
\chead{} %
\cfoot{} %
\lfoot{\footnotesize\nouppercase\leftmark} %
\rfoot{\bfseries\thepage}
\renewcommand{\footrulewidth}{0.1pt}

% Comando para inserir numero de documento
\newcounter{document}%[section]
\setcounter{document}{0}
\renewcommand\theenumi{\arabic{section}.\arabic{enumi}}
\newcommand\Doc{{\addtocounter{document}{1}\mbox{\sffamily\bfseries [Doc. \arabic{document}]}}}

% Comando para repetir um numero de documento ja citado
% \mbox{\sffamily{\bfseries{[Doc. XX]}}}

%% Alternativa na edicao dos comandos.
%% Comando para inserir numero de documento
% \newcommand\thedocument{%
%    \ifthenelse{\arabic{subsection}=0}
%      {\thesection.\arabic{document}}
%      {\thesubsection.\arabic{document}}}
% \newcounter{document}[section]
% \setcounter{document}{0}
% \renewcommand\theenumi{\arabic{section}.\arabic{enumi}}
% \ifthenelse{\arabic{subsection} = 0}{\newcommand\Doc{{\stepcounter{document}\bfseries [Doc. \arabic{section}.\arabic{document}]}}}{\newcommand\Doc{{\stepcounter{document}\bfseries [Doc. \arabic{section}.\arabic{subsection}.\arabic{document}]}}}
% %\newcommand\Doc{{\addtocounter{document}{1}\mbox{\bfseries [Doc. \arabic{document}]}}}


% Ambiente para centralizar vertical
\newenvironment{vcenterpage}
     {\newpage\vspace*{\fill}}
     {\vspace*{\fill}\par\pagebreak}

\sloppy

\pagestyle{fancy}

\setcounter{secnumdepth}{4}

\begin{document}
\pagenumbering{Alph}
\begin{titlepage}

\vspace{-5.0cm}

\begin{figure}[!htb]
 \centering{\includegraphics[scale=4.0]{logo2.eps}}
 \label{fig:UFPE_logo}
\end{figure}

\begin{center}
\vspace{1cm}
%{\huge \textsf{Solicita\c{c}\~{a}o de Progress\~{a}o Funcional Docente}} \\[1cm]
\rule{1.0\textwidth}{1pt} \\ [0.5cm]
{\Huge \textbf{\textsf{Memorial Descritivo de Atividades}}} \\
\rule{1.0\textwidth}{1pt} \\
\vspace{2cm}

\doublespacing
{\Large \textsf{Solicita\c{c}\~{a}o de progresss\~{a}o funcional docente de \textbf{Adjunto N\'{\i}vel 3 para Adjunto N\'{\i}vel 4} apresentada \`{a} Comiss\~{a}o de Avalia\c{c}\~{a}o de Progress\~{a}o Horizontal do Centro de Inform\'{a}tica da Universidade Federal de Pernambuco}}\\
\vspace{1.5cm}
{\LARGE \textsf{Solicitante: \textbf{Vinicius Cardoso Garcia}}}\\
\vspace{0.5cm}
{\Large \textsf{SIAPE: \textbf{1807586}}} \\
\vspace{0.5cm}
{\Large \textsf{Per\'{\i}odo: \textbf{20/08/2014 - 19/08/2016}}} \\

\vspace{2.0cm}

\normalsize \textsf{\today}

\end{center}
\thispagestyle{empty}
\end{titlepage}
\pagenumbering{arabic}

\tableofcontents
%\include{Lista_Anexos} \cleartooddpage[\thispagestyle{empty}]

%%%%%%%%%%%%%%%%%%%%%%%%%%%%%%%%%%%%%%%%%%%%%%%%%%%%%%%%%%%%%%%%%%%%%%%%%%%%%%%
% APRESENTACAO
%%%%%%%%%%%%%%%%%%%%%%%%%%%%%%%%%%%%%%%%%%%%%%%%%%%%%%%%%%%%%%%%%%%%%%%%%%%%%%%

\newpage
\section*{Apresenta\c{c}\~{a}o}
\vspace{0.3cm}

\begin{onehalfspace}

Memorial apresentado por \textbf{Vinicius Cardoso Garcia}, Professor Adjunto 3, lotado no Departamento de Informa\c{c}\~{a}o e Sistemas do Centro de Inform\'{a}tica (CIn) da Universidade Federal de Pernambuco (UFPE), para avalia\c{c}\~{a}o de desempenho acad\^{e}mico, para fins de acesso por Progress\~{a}o Horizontal \`{a} classe de Professor Adjunto N\'{\i}vel 4.

O presente memorial relata as atividades desempenhadas no per\'{\i}odo de \textbf{20 de Agosto de 2014 a 19 de Agosto de 2016} e foi elaborado com base nas diretrizes estabelecidas nas Resolu\c{c}\~{o}es 04/2008 e 02/2011 do Conselho Universit\'{a}rio (CONSUN) da UFPE e na Portaria 554, de 20/06/2013, do Minist\'{e}rio da Educa\c{c}\~{a}o (MEC). Os documentos comprobat\'{o}rios referenciados neste memorial est\~{a}o organizados em volumes anexos devidamente numerados. Por fim, os Anexos que se referem a artigos publicados em confer\^{e}ncias e peri\'{o}dicos cont\'{e}m apenas a primeira p\'{a}gina do trabalho e o email informando a aceita\c{c}\~{a}o do trabalhao para publica\c{c}\~{a}o (quando pertinente).

\end{onehalfspace}

\subsection*{Esclarecimentos}

A seguir, s\~{a}o listadas algumas observa\c{c}\~{o}es acerca da documenta\c{c}\~{a}o comprobat\'{o}ria para dirimir poss\'{\i}veis d\'{u}vidas da Comiss\~{a}o Avaliadora. A numera\c{c}\~{a}o indicada corresponde \`{a} numera\c{c}\~{a}o do Anexo.

\begin{itemize}

\item [18.] Artigo dispon\'{\i}vel em \verb"dx.doi.org/10.14209/sbrt.2013.182".
\item [19.] Lista de artigos aceitos dispon\'{\i}vel em \verb"http://www.sbrt.org.br/sbrt2012/artigos.html". O artigo em quest\~{a}o possui identifica\c{c}\~{a}o 98879.
\item [20.] Artigo dispon\'{\i}vel em \verb"http://cba2012.dee.ufcg.edu.br/anais".
\item [21.] Artigo dispon\'{\i}vel em \verb"dx.doi.org/10.14209/sbrt.2013.179".
\item [22.] Artigo dispon\'{\i}vel em \verb"dx.doi.org/10.14209/sbrt.2013.181".
\item [23.] Artigo dispon\'{\i}vel em \verb"dx.doi.org/10.14209/sbrt.2013.247".

\item [32.] Consta o e-mail \verb"dccunha@upe.poli.br" no cabe\c{c}alho do documento, pois os dados do avaliador ainda n\~{a}o haviam sido atualizados no sistema AgilFAP.

\item [35.] Artigo dispon\'{\i}vel em \verb"http://www.scielo.br/pdf/tema/v13n2/a09v13n2.pdf".
\item [36.] Artigo dispon\'{\i}vel em \verb"http://asp.eurasipjournals.com/content/pdf/1687-6180-2013-28.pdf".

\end{itemize}


%%%%%%%%%%%%%%%%%%%%%%%%%%%%%%%%%%%%%%%%%%%%%%%%%%%%%%%%%%%%%%%%%%%%%%%%%%%%%%%
% Grupo 1 - Atividades de Ensino
%%%%%%%%%%%%%%%%%%%%%%%%%%%%%%%%%%%%%%%%%%%%%%%%%%%%%%%%%%%%%%%%%%%%%%%%%%%%%%%
\newpage
\section{Atividades de Ensino}

%A seguir, listo as atividades de ensino que realizei no per\'{\i}odo, separadas por subgrupo, conforme rege o documento tal.

%%%%%%%%%%%%%%%%%%%%%%%%%%%%%%%%%%%%%%%%%%%%%%%%%%%%%%%%%%%%%%%%%%%%%%%%%%%%%%%
% Subgrupo 1.1 - Orienta\c{c}\~{o}es e Co-Orienta\c{c}\~{o}es
%%%%%%%%%%%%%%%%%%%%%%%%%%%%%%%%%%%%%%%%%%%%%%%%%%%%%%%%%%%%%%%%%%%%%%%%%%%%%%%
\subsection{Supervis\~{a}o de est\'{a}gios curriculares e extracurriculares}
\vspace{0.3cm}

\begin{enumerate}
\renewcommand{\labelenumi}{{\large\bfseries\arabic{enumi}.}}

\item       \textbf{Aluno:}  \Doc \\
            \textbf{Curso:} \\
            \textbf{Empresa:}  \\
            \textbf{Período:} 

\end{enumerate}

%------------------------------------------------------------------------------

\subsection{Orienta\c{c}\~{o}es e Co-Orienta\c{c}\~{o}es}
\vspace{0.3cm}

\subsubsection{Orienta\c{c}\~{a}o de Teses de Doutorado Conclu\'{i}das}
\vspace{0.3cm}

\begin{enumerate}
\renewcommand{\labelenumi}{{\large\bfseries\arabic{enumi}.}}

\item       \textbf{Aluno:}  \Doc \\
            \textbf{T\'{\i}tulo da Tese:} 
            \textbf{Data da Defesa:}  \\
            \textbf{Institui\c{c}\~{a}o:} 

\end{enumerate}

%------------------------------------------------------------------------------

\subsubsection{Co-Orienta\c{c}\~{a}o de Teses de Doutorado Conclu\'{i}das}
\vspace{0.3cm}

\begin{enumerate}
\renewcommand{\labelenumi}{{\large\bfseries\arabic{enumi}.}}

\item       \textbf{Aluno:}  \Doc \\
            \textbf{T\'{\i}tulo da Tese:} 
            \textbf{Data da Defesa:}  \\
            \textbf{Institui\c{c}\~{a}o:} 

\end{enumerate}

%------------------------------------------------------------------------------

\subsubsection{Orienta\c{c}\~{a}o de Teses de Doutorado em Andamento}
\vspace{0.3cm}

\begin{enumerate}
\renewcommand{\labelenumi}{{\large\bfseries\arabic{enumi}.}}

\item       \textbf{Aluno:} Clovis Holanda do Nascimento \mbox{\sffamily{\bfseries{[Doc. \ref{app:orientacoes-andamento}]}}} \\
            \textbf{T\'{\i}tulo da Tese:}  \\
            \textbf{Data de In\'{\i}cio:} Mar\c{c}o de 2015 \\
            \textbf{Institui\c{c}\~{a}o:} CIn/UFPE

\item       \textbf{Aluno:} Felipe Oliveira Gutierrez \mbox{\sffamily{\bfseries{[Doc. \ref{app:orientacoes-andamento}]}}}\\
            \textbf{T\'{\i}tulo da Tese:}  \\
            \textbf{Data de In\'{\i}cio:} Mar\c{c}o de 2015 \\
            \textbf{Institui\c{c}\~{a}o:} CIn/UFPE

\item       \textbf{Aluno:} Carlo Marcelo Revoredo da Silva \mbox{\sffamily{\bfseries{[Doc. \ref{app:orientacoes-andamento}]}}}\\
            \textbf{T\'{\i}tulo da Tese:}  A Reference Architecture for Cloud Storage Services\\
            \textbf{Data de In\'{\i}cio:} Mar\c{c}o de 2014 \\
            \textbf{Institui\c{c}\~{a}o:} CIn/UFPE

\item       \textbf{Aluno:} Josino Rodrigues Neto \mbox{\sffamily{\bfseries{[Doc. \ref{app:orientacoes-andamento}]}}} \\
            \textbf{T\'{\i}tulo da Tese:} UstoreDB: Towards An Approach for Data Storage in a Private Cloud \\
            \textbf{Data de In\'{\i}cio:} Mar\c{c}o de 2014 \\
            \textbf{Institui\c{c}\~{a}o:} CIn/UFPE

\item       \textbf{Aluno:} Anderson Fonseca e Silva \mbox{\sffamily{\bfseries{[Doc. \ref{app:orientacoes-andamento}]}}} \\
            \textbf{T\'{\i}tulo da Tese:}  Uma Arquitetura de Refer\^{e}ncia para Solu\c{c}\~{o}es de Storage de Dados em Nuvem\\
            \textbf{Data de In\'{\i}cio:} Mar\c{c}o de 2013 \\
            \textbf{Institui\c{c}\~{a}o:} CIn/UFPE

\item       \textbf{Aluno:} Andreza Leite de Alencar \mbox{\sffamily{\bfseries{[Doc. \ref{app:orientacoes-andamento}]}}} \\
            \textbf{T\'{\i}tulo da Tese:} Um Framework para Integração de Grandes Volumes de Dados no Ambiente de Computação em Nuvem \\
            \textbf{Data de In\'{\i}cio:} Mar\c{c}o de 2013 \\
            \textbf{Institui\c{c}\~{a}o:} CIn/UFPE

\item       \textbf{Aluno:} Paulo Anselmo da Mota Silveira Neto \mbox{\sffamily{\bfseries{[Doc. \ref{app:orientacoes-andamento}]}}} \\
            \textbf{T\'{\i}tulo da Tese:} Assessing Non-Functional Security Features on Software Product Line Architectures: A Decision Model \\
            \textbf{Data de In\'{\i}cio:} Mar\c{c}o de 2012 \\
            \textbf{Institui\c{c}\~{a}o:} CIn/UFPE

\item       \textbf{Aluno:} Ricardo Alexandre Afonso \mbox{\sffamily{\bfseries{[Doc. \ref{app:orientacoes-andamento}]}}} \\
            \textbf{T\'{\i}tulo da Tese:} Smart City Maturity Model \\
            \textbf{Data de In\'{\i}cio:} Mar\c{c}o de 2012 \\
            \textbf{Institui\c{c}\~{a}o:} CIn/UFPE

\end{enumerate}

%------------------------------------------------------------------------------

\subsubsection{Co-Orienta\c{c}\~{a}o de Teses de Doutorado em Andamento}
\vspace{0.3cm}

\begin{enumerate}
\renewcommand{\labelenumi}{{\large\bfseries\arabic{enumi}.}}

\item       \textbf{Aluno:}  \Doc \\
            \textbf{T\'{\i}tulo da Tese:} \\
            \textbf{Data de In\'{\i}cio:}  \\
            \textbf{Institui\c{c}\~{a}o:} 

\item       \textbf{Aluno:} José Fernando Santos de Carvalho \mbox{\sffamily{\bfseries{[Doc. \ref{app:co-orientacoes-andamento}]}}} \\
            \textbf{T\'{\i}tulo da Tese:} Uma plataforma de computação em nuvem privada para apoio a e-Science \\
            \textbf{Data de In\'{\i}cio:} Março de 2014 \\
            \textbf{Institui\c{c}\~{a}o:} CIn/UFPE

\item       \textbf{Aluno:} Jackson Raniel Florencio da Silva \mbox{\sffamily{\bfseries{[Doc. \ref{app:co-orientacoes-andamento}]}}} \\
            \textbf{T\'{\i}tulo da Tese:} Impacto financeiro da variabilidade em tempo de execução em sistemas orientados a serviço\\
            \textbf{Data de In\'{\i}cio:} Março de 2014 \\
            \textbf{Institui\c{c}\~{a}o:} CIn/UFPE

\item       \textbf{Aluno:} Ricardo Batista Rodrigues \mbox{\sffamily{\bfseries{[Doc. \ref{app:co-orientacoes-andamento}]}}} \\
            \textbf{T\'{\i}tulo da Tese:} Um Modelo de Recomendação Híbrido Para Sistemas de Armazenamento em Nuvem\\
            \textbf{Data de In\'{\i}cio:} Março de 2014\\
            \textbf{Institui\c{c}\~{a}o:} CIn/UFPE 
            

\end{enumerate}

%------------------------------------------------------------------------------

\subsubsection{Orienta\c{c}\~{a}o de Disserta\c{c}\~{o}es de Mestrado Conclu\'{i}das}
\vspace{0.3cm}

\begin{enumerate}
\renewcommand{\labelenumi}{{\large\bfseries\arabic{enumi}.}}

\item       \textbf{Aluno:} \Doc \\
            \textbf{T\'{\i}tulo da Disserta\c{c}\~{a}o:} \\
            \textbf{Tipo:} Acadêmico | Profissional\\
            \textbf{Data da Defesa:} \\
            \textbf{Institui\c{c}\~{a}o:} CIn/UFPE

% TODO
\item       \textbf{Aluno:} Alfredo Pupak Pereira Virote \mbox{\sffamily{\bfseries{[Doc. \ref{app:orientacoes-mprof-concluidas}]}}} \\
            \textbf{T\'{\i}tulo da Disserta\c{c}\~{a}o:} Aproveitamento racional e otimizado dos recursos computacionais do IF Goiano: contextualização do processo baseado na computação em nuvem\\
            \textbf{Tipo:} Profissional\\
            \textbf{Data da Defesa:} \colorbox{yellow}{\textbf{A DEFINIR}}\\
            \textbf{Institui\c{c}\~{a}o:} CIn/UFPE

% TODO
\item       \textbf{Aluno:} Fernando Estrela Vaz \mbox{\sffamily{\bfseries{[Doc. \ref{app:orientacoes-mprof-concluidas}]}}} \\
            \textbf{T\'{\i}tulo da Disserta\c{c}\~{a}o:} Título a ser definido\\
            \textbf{Tipo:} Profissional\\
            \textbf{Data da Defesa:} \colorbox{yellow}{\textbf{A DEFINIR}}\\
            \textbf{Institui\c{c}\~{a}o:} CIn/UFPE

% TODO
\item       \textbf{Aluno:} Giovani Felipe Jahn \mbox{\sffamily{\bfseries{[Doc. \ref{app:orientacoes-mprof-concluidas}]}}} \\
            \textbf{T\'{\i}tulo da Disserta\c{c}\~{a}o:} Como o uso de lagoa de dados beneficiaria a evolução das cidades inteligentes\\
            \textbf{Tipo:} Profissional\\
            \textbf{Data da Defesa:} \colorbox{yellow}{\textbf{A DEFINIR}}\\
            \textbf{Institui\c{c}\~{a}o:} CIn/UFPE

% TODO
\item       \textbf{Aluno:} Jadson Fabio Santos Junior \mbox{\sffamily{\bfseries{[Doc. \ref{app:orientacoes-mprof-concluidas}]}}} \\
            \textbf{T\'{\i}tulo da Disserta\c{c}\~{a}o:} Proteção de dados na Nuvem Comunitária. Estudo no Instituto Federal de Sergipe e Universidade Federal de Sergipe\\
            \textbf{Tipo:} Profissional\\
            \textbf{Data da Defesa:} \colorbox{yellow}{\textbf{A DEFINIR}}\\
            \textbf{Institui\c{c}\~{a}o:} CIn/UFPE

% TODO
\item       \textbf{Aluno:} Willamys Gomes Fonseca Araujo \mbox{\sffamily{\bfseries{[Doc. \ref{app:orientacoes-mprof-concluidas}]}}} \\
            \textbf{T\'{\i}tulo da Disserta\c{c}\~{a}o:} Uma Proposta de Implantação de uma Plataforma como Serviço (PaaS) para Auxiliar o Ensino de Programação no IFCE Campus Tianguá\\
            \textbf{Tipo:} Profissional\\
            \textbf{Data da Defesa:} \colorbox{yellow}{\textbf{A DEFINIR}}\\
            \textbf{Institui\c{c}\~{a}o:} CIn/UFPE

% TODO
\item       \textbf{Aluno:} Charles Everton Oliveira Gomes \mbox{\sffamily{\bfseries{[Doc. \ref{app:orientacoes-concluidas}]}}} \\
            \textbf{T\'{\i}tulo da Disserta\c{c}\~{a}o:} Uma Arquitetura para Rastreabilidade Baseado em Ferramentas Open Source\\
            \textbf{Tipo:} Acadêmico\\
            \textbf{Data da Defesa:} \colorbox{yellow}{\textbf{A DEFINIR}}\\
            \textbf{Institui\c{c}\~{a}o:} CIn/UFPE

\item       \textbf{Aluno:} Ricardo Marinho de Melo \mbox{\sffamily{\bfseries{[Doc. \ref{app:orientacoes-concluidas}]}}} \\
            \textbf{T\'{\i}tulo da Disserta\c{c}\~{a}o:} Uma Arquitetura de Referência Guiada à Conformidades de Segurança para Web Services Baseados em Nuvem Privada\\
            \textbf{Tipo:} Acadêmico\\
            \textbf{Data da Defesa:} 03 de Março de 2016\\
            \textbf{Institui\c{c}\~{a}o:} CIn/UFPE

\item       \textbf{Aluno:} Hilson Barbosa da Silva \mbox{\sffamily{\bfseries{[Doc. \ref{app:orientacoes-mprof-concluidas}]}}} \\
            \textbf{T\'{\i}tulo da Disserta\c{c}\~{a}o:} Computação em Nuvem: Uma Abordagem Comparativa do Investimento do Processo de Migração entre os Ambientes de Nuvem (Software como Serviço, SaaS) e o Modelo Tradicional de Software\\
            \textbf{Tipo:} Profissional\\
            \textbf{Data da Defesa:} 22 de Janeiro de 2016\\
            \textbf{Institui\c{c}\~{a}o:} CIn/UFPE

\item       \textbf{Aluno:} José Rafael Moraes Garcia da Rocha \mbox{\sffamily{\bfseries{[Doc. \ref{app:orientacoes-mprof-concluidas}]}}} \\
            \textbf{T\'{\i}tulo da Disserta\c{c}\~{a}o:} Developing programming skills on digital native children through the interaction with smart devices\\
            \textbf{Tipo:} Profissional\\
            \textbf{Data da Defesa:} 15 de Janeiro de 2016\\
            \textbf{Institui\c{c}\~{a}o:} CIn/UFPE

\item       \textbf{Aluno:} Helaine Solange Lins Barreiros \mbox{\sffamily{\bfseries{[Doc. \ref{app:orientacoes-concluidas}]}}} \\
            \textbf{T\'{\i}tulo da Disserta\c{c}\~{a}o:} Análise da Completude dos Relatos de Experimentos em Elasticidade na Computação em Nuvem: Um Mapeamento Sistemático\\
            \textbf{Tipo:} Acadêmico\\
            \textbf{Data da Defesa:} 31 de Agosto de 2015\\
            \textbf{Institui\c{c}\~{a}o:} CIn/UFPE

\item       \textbf{Aluno:} Álvaro Magnum Barbosa Neto \mbox{\sffamily{\bfseries{[Doc. \ref{app:orientacoes-mprof-concluidas}]}}} \\
            \textbf{T\'{\i}tulo da Disserta\c{c}\~{a}o:} Em Direção a um Ambiente de Desenvolvimento de Software Orientado por Comportamento\\
            \textbf{Tipo:} Profissional\\
            \textbf{Data da Defesa:} 22 de Maio de 2015\\
            \textbf{Institui\c{c}\~{a}o:} CIn/UFPE

\item       \textbf{Aluno:} Jean Louis Brasil Fernandes Costa \mbox{\sffamily{\bfseries{[Doc. \ref{app:orientacoes-concluidas}]}}} \\
            \textbf{T\'{\i}tulo da Disserta\c{c}\~{a}o:} Uma Proposta de Solução para Distribuição de Conteúdos Educacionais\\
            \textbf{Tipo:} Acadêmico\\
            \textbf{Data da Defesa:} 2 de Setembro de 2014\\
            \textbf{Institui\c{c}\~{a}o:} CIn/UFPE

\item       \textbf{Aluno:} Welington Manoel da Silva \mbox{\sffamily{\bfseries{[Doc. \ref{app:orientacoes-concluidas}]}}} \\
            \textbf{T\'{\i}tulo da Disserta\c{c}\~{a}o:} Go!SIP: Um Framework de Privacidade para Cidades Inteligentes Baseado em Pessoas Como Sensores\\
            \textbf{Tipo:} Acadêmico\\
            \textbf{Data da Defesa:} 25 de Agosto de 2014\\
            \textbf{Institui\c{c}\~{a}o:} CIn/UFPE
\end{enumerate}

%------------------------------------------------------------------------------

\subsubsection{Co-Orienta\c{c}\~{a}o de Disserta\c{c}\~{o}es de Mestrado Conclu\'{i}das}
\vspace{0.3cm}

\begin{enumerate}
\renewcommand{\labelenumi}{{\large\bfseries\arabic{enumi}.}}

\item       \textbf{Aluno:}  \Doc \\
            \textbf{T\'{\i}tulo da Disserta\c{c}\~{a}o:} \\
            \textbf{Tipo:} Acadêmico | Profissional\\
            \textbf{Data da Defesa:}  \\
            \textbf{Institui\c{c}\~{a}o:} CIn/UFPE

\end{enumerate}

%------------------------------------------------------------------------------

\subsubsection{Orienta\c{c}\~{a}o de Disserta\c{c}\~{o}es de Mestrado em Andamento}
\vspace{0.3cm}

\begin{enumerate}
\renewcommand{\labelenumi}{{\large\bfseries\arabic{enumi}.}}

\item       \textbf{Aluno:}  \Doc \\
            \textbf{T\'{\i}tulo da Disserta\c{c}\~{a}o:} \\
            \textbf{Tipo:} Acadêmico | Profissional\\
            \textbf{Data de In\'{\i}cio:} \\
            \textbf{Institui\c{c}\~{a}o:} CIn/UFPE

\item       \textbf{Aluno:}  \Doc \\
            \textbf{T\'{\i}tulo da Disserta\c{c}\~{a}o:} \\
            \textbf{Tipo:} Acadêmico | Profissional\\
            \textbf{Data de In\'{\i}cio:} \\
            \textbf{Institui\c{c}\~{a}o:} CIn/UFPE

\item       \textbf{Aluno:} Leonardo da Silva Leandro \mbox{\sffamily{\bfseries{[Doc. \ref{app:orientacoes-andamento}]}}} \\
            \textbf{T\'{\i}tulo da Disserta\c{c}\~{a}o:} \\
            \textbf{Tipo:} Acadêmico\\
            \textbf{Data de In\'{\i}cio:} \\
            \textbf{Institui\c{c}\~{a}o:} CIn/UFPE

\item       \textbf{Aluno:} Raphael Mandarino Junior \mbox{\sffamily{\bfseries{[Doc. \ref{app:orientacoes-andamento}]}}} \\
            \textbf{T\'{\i}tulo da Disserta\c{c}\~{a}o:} \\
            \textbf{Tipo:} Acadêmico\\
            \textbf{Data de In\'{\i}cio:} Março de 2015\\
            \textbf{Institui\c{c}\~{a}o:} CIn/UFPE

\item       \textbf{Aluno:} Wilson Alves da Silva \mbox{\sffamily{\bfseries{[Doc. \ref{app:orientacoes-andamento}]}}} \\
            \textbf{T\'{\i}tulo da Disserta\c{c}\~{a}o:} \\
            \textbf{Tipo:} Acadêmico\\
            \textbf{Data de In\'{\i}cio:} Março de 2015\\
            \textbf{Institui\c{c}\~{a}o:} CIn/UFPE

\item       \textbf{Aluno:} Plínio Antunes Garcia \mbox{\sffamily{\bfseries{[Doc. \ref{app:orientacoes-mprof-andamento}]}}} \\
            \textbf{T\'{\i}tulo da Disserta\c{c}\~{a}o:} Aplicação de Grounded Theory na elicitação de requisitos de um sistema de apoio à elaboração de Plano Diretor de TI\\
            \textbf{Tipo:} Profissional\\
            \textbf{Data de In\'{\i}cio:} Janeiro de 2015\\
            \textbf{Institui\c{c}\~{a}o:} CIn/UFPE

\item       \textbf{Aluno:} Charles Everton Oliveira Gomes \mbox{\sffamily{\bfseries{[Doc. \ref{app:orientacoes-andamento}]}}} \\
            \textbf{T\'{\i}tulo da Disserta\c{c}\~{a}o:} \\
            \textbf{Tipo:} Acadêmico\\
            \textbf{Data de In\'{\i}cio:} Março de 2014\\
            \textbf{Institui\c{c}\~{a}o:} CIn/UFPE

\item       \textbf{Aluno:} Ricardo Marinho de Melo \mbox{\sffamily{\bfseries{[Doc. \ref{app:orientacoes-andamento}]}}} \\
            \textbf{T\'{\i}tulo da Disserta\c{c}\~{a}o:} \\
            \textbf{Tipo:} Acadêmico\\
            \textbf{Data de In\'{\i}cio:} Março de 2013\\
            \textbf{Institui\c{c}\~{a}o:} CIn/UFPE

\end{enumerate}

%------------------------------------------------------------------------------
\subsubsection{Co-Orienta\c{c}\~{a}o de Disserta\c{c}\~{o}es de Mestrado em Andamento}
\vspace{0.3cm}

\begin{enumerate}
\renewcommand{\labelenumi}{{\large\bfseries\arabic{enumi}.}}

\item       \textbf{Aluno:}  \Doc \\
            \textbf{T\'{\i}tulo da Disserta\c{c}\~{a}o:} \\
            \textbf{Tipo:} Acadêmico | Profissional\\
            \textbf{Data de In\'{\i}cio:} \\
            \textbf{Institui\c{c}\~{a}o:} CIn/UFPE

\end{enumerate}

%------------------------------------------------------------------------------

\subsubsection{Orienta\c{c}\~{a}o de Trabalhos de Conclus\~{a}o de Curso}
\vspace{0.3cm}

\begin{enumerate}
\renewcommand{\labelenumi}{{\large\bfseries\arabic{enumi}.}}

\item       \textbf{Aluno:}  \Doc \\
            \textbf{Curso:} \\
            \textbf{T\'{\i}tulo da Monografia:} \\
            \textbf{Data da Defesa:} \\
            \textbf{Institui\c{c}\~{a}o:} CIn/UFPE

\item       \textbf{Aluno:}  Ronald Dener Bezerra Pessa \mbox{\sffamily{\bfseries{[Doc. \ref{app:orientacoes-tcc-concluidas}]}}}\\
            \textbf{Curso:} Graduação em Engenharia da Computação \\
            \textbf{T\'{\i}tulo da Monografia:} Internet das Coisas na Indústria Civil Brasileira\\
            \textbf{Data da Defesa:} 02 de Fevereiro de 2016\\
            \textbf{Institui\c{c}\~{a}o:} CIn/UFPE

\item       \textbf{Aluno:} José Welington de Almeida Filho \mbox{\sffamily{\bfseries{[Doc. \ref{app:orientacoes-tcc-concluidas}]}}}\\
            \textbf{Curso:} Graduação em Ciência da Computação\\
            \textbf{T\'{\i}tulo da Monografia:} Uma proposta de uso de Gamification para o ensino de software.\\
            \textbf{Data da Defesa:} 28 de Julho de 2015\\
            \textbf{Institui\c{c}\~{a}o:} CIn/UFPE

\item       \textbf{Aluno:} Isabel Oliveira Jordão do Amaral \mbox{\sffamily{\bfseries{[Doc. \ref{app:orientacoes-tcc-concluidas}]}}}\\
            \textbf{Curso:} Graduação em Sistemas de Informação\\
            \textbf{T\'{\i}tulo da Monografia:} Uma investigação do cuidado dos usuários na disponibilização de informações pessoais em Mídias Sociais.\\
            \textbf{Data da Defesa:} 13 de Julho de 2015\\
            \textbf{Institui\c{c}\~{a}o:} CIn/UFPE

\item       \textbf{Aluno:} Victor Chaves Casé \mbox{\sffamily{\bfseries{[Doc. \ref{app:orientacoes-tcc-concluidas}]}}}\\
            \textbf{Curso:} Graduação em Engenharia da Computação\\
            \textbf{T\'{\i}tulo da Monografia:} Em Direção a uma Abordagem para o Desenvolvimento de Aplicativos Móveis Multiplataforma.\\
            \textbf{Data da Defesa:} 27 de Fevereiro de 2015\\
            \textbf{Institui\c{c}\~{a}o:} CIn/UFPE


\end{enumerate}

%------------------------------------------------------------------------------

\subsubsection{Orienta\c{c}\~{a}o de Monitoria}
\vspace{0.3cm}

\begin{enumerate}
\renewcommand{\labelenumi}{{\large\bfseries\arabic{enumi}.}}

\item   \textbf{Aluno:}  \mbox{\sffamily{\bfseries{[Doc. \ref{app:orientacoes-monitoria}]}}} \\
        \textbf{Disciplina:}  Engenharia Software e Sistemas (turma SI)\\
        \textbf{Curso:} Sistemas de Informação\\
        \textbf{Semestre:} 2016.1

\item   \textbf{Aluno:} Sandrine Ventura Martins \mbox{\sffamily{\bfseries{[Doc. \ref{app:orientacoes-monitoria}]}}} \\
        \textbf{Disciplina:}  Engenharia Software e Sistemas (turma SI)\\
        \textbf{Curso:} Sistemas de Informação\\
        \textbf{Semestre:} 2015.2

\item   \textbf{Aluno:} Mariana Ferreira de Melo \mbox{\sffamily{\bfseries{[Doc. \ref{app:orientacoes-monitoria}]}}} \\
        \textbf{Disciplina:}  Engenharia Software e Sistemas (turma SI)\\
        \textbf{Curso:} Sistemas de Informação\\
        \textbf{Semestre:} 2015.2

\item   \textbf{Aluno:} Marcela Pereira de Oliveira \mbox{\sffamily{\bfseries{[Doc. \ref{app:orientacoes-monitoria}]}}} \\
        \textbf{Disciplina:}  Engenharia Software e Sistemas (turma SI)\\
        \textbf{Curso:} Sistemas de Informação\\
        \textbf{Semestre:} 2015.2

\item   \textbf{Aluno:} José Durval Carneiro Campello Neto \mbox{\sffamily{\bfseries{[Doc. \ref{app:orientacoes-monitoria}]}}} \\
        \textbf{Disciplina:}  Engenharia Software e Sistemas (turma SI)\\
        \textbf{Curso:} Sistemas de Informação\\
        \textbf{Semestre:} 2015.2

\item   \textbf{Aluno:} Jorge Henrique Cordeiro Linhares \mbox{\sffamily{\bfseries{[Doc. \ref{app:orientacoes-monitoria}]}}} \\
        \textbf{Disciplina:}  Engenharia Software e Sistemas (turma SI)\\
        \textbf{Curso:} Sistemas de Informação\\
        \textbf{Semestre:} 2015.2

\item   \textbf{Aluno:} Dênio Batista Brasileiro Bezerra \mbox{\sffamily{\bfseries{[Doc. \ref{app:orientacoes-monitoria}]}}} \\
        \textbf{Disciplina:}  Engenharia Software e Sistemas (turma SI)\\
        \textbf{Curso:} Sistemas de Informação\\
        \textbf{Semestre:} 2015.2

\item   \textbf{Aluno:} Avyner Henrique B. da Fonseca Lucena \mbox{\sffamily{\bfseries{[Doc. \ref{app:orientacoes-monitoria}]}}} \\
        \textbf{Disciplina:}  Engenharia Software e Sistemas (turma SI)\\
        \textbf{Curso:} Sistemas de Informação\\
        \textbf{Semestre:} 2015.2

\item   \textbf{Aluno:} Sandrine Ventura Martins \mbox{\sffamily{\bfseries{[Doc. \ref{app:orientacoes-monitoria}]}}} \\
        \textbf{Disciplina:}  Engenharia Software e Sistemas (turma SI)\\
        \textbf{Curso:} Sistemas de Informação\\
        \textbf{Semestre:} 2015.1

\item   \textbf{Aluno:} Pedro Paulo Sousa Neto \mbox{\sffamily{\bfseries{[Doc. \ref{app:orientacoes-monitoria}]}}} \\
        \textbf{Disciplina:}  Engenharia Software e Sistemas (turma SI)\\
        \textbf{Curso:} Sistemas de Informação\\
        \textbf{Semestre:} 2015.1

\item   \textbf{Aluno:} José Durval Carneiro Campello Neto \mbox{\sffamily{\bfseries{[Doc. \ref{app:orientacoes-monitoria}]}}} \\
        \textbf{Disciplina:}  Engenharia Software e Sistemas (turma SI)\\
        \textbf{Curso:} Sistemas de Informação\\
        \textbf{Semestre:} 2015.1

\item   \textbf{Aluno:} Jorge Henrique Cordeiro Linhares \mbox{\sffamily{\bfseries{[Doc. \ref{app:orientacoes-monitoria}]}}} \\
        \textbf{Disciplina:}  Engenharia Software e Sistemas (turma SI)\\
        \textbf{Curso:} Sistemas de Informação\\
        \textbf{Semestre:} 2015.1

\item   \textbf{Aluno:} Alysson dos Santos Lima \mbox{\sffamily{\bfseries{[Doc. \ref{app:orientacoes-monitoria}]}}} \\
        \textbf{Disciplina:}  Engenharia Software e Sistemas (turma SI)\\
        \textbf{Curso:} Sistemas de Informação\\
        \textbf{Semestre:} 2015.1

\item   \textbf{Aluno:} Sandrine Ventura Martins \mbox{\sffamily{\bfseries{[Doc. \ref{app:orientacoes-monitoria}]}}} \\
        \textbf{Disciplina:}  Engenharia Software e Sistemas (turma SI)\\
        \textbf{Curso:} Sistemas de Informação\\
        \textbf{Semestre:} 2014.2

\item   \textbf{Aluno:} Pedro Paulo Sousa Neto \mbox{\sffamily{\bfseries{[Doc. \ref{app:orientacoes-monitoria}]}}} \\
        \textbf{Disciplina:}  Engenharia Software e Sistemas (turma SI)\\
        \textbf{Curso:} Sistemas de Informação\\
        \textbf{Semestre:} 2014.2

\item   \textbf{Aluno:} Jorge Henrique Cordeiro Linhares \mbox{\sffamily{\bfseries{[Doc. \ref{app:orientacoes-monitoria}]}}} \\
        \textbf{Disciplina:}  Engenharia Software e Sistemas (turma SI)\\
        \textbf{Curso:} Sistemas de Informação\\
        \textbf{Semestre:} 2014.2

\item   \textbf{Aluno:} João Vitor Bizerra de Araujo \mbox{\sffamily{\bfseries{[Doc. \ref{app:orientacoes-monitoria}]}}} \\
        \textbf{Disciplina:}  Engenharia Software e Sistemas (turma SI)\\
        \textbf{Curso:} Sistemas de Informação\\
        \textbf{Semestre:} 2014.2

\item   \textbf{Aluno:} Alysson dos Santos Lima \mbox{\sffamily{\bfseries{[Doc. \ref{app:orientacoes-monitoria}]}}} \\
        \textbf{Disciplina:}  Engenharia Software e Sistemas (turma SI)\\
        \textbf{Curso:} Sistemas de Informação\\
        \textbf{Semestre:} 2014.2

\end{enumerate}

%------------------------------------------------------------------------------

\subsubsection{Orienta\c{c}\~{a}o de Trabalhos de Inicia\c{c}\~{a}o Cient\'{\i}fica}
\vspace{0.3cm}

\begin{enumerate}
\renewcommand{\labelenumi}{{\large\bfseries\arabic{enumi}.}}

\item   \textbf{Aluno:}  \Doc \\
        \textbf{Projeto:}  \\
        \textbf{Tema:}  \\
        \textbf{Per\'{\i}odo:}  \\
        \textbf{Financiamento:} 

\end{enumerate}

%------------------------------------------------------------------------------

\subsubsection{Orienta\c{c}\~{a}o de Trabalhos de Apoio Acad\^{e}mico}
\vspace{0.3cm}

\begin{enumerate}
\renewcommand{\labelenumi}{{\large\bfseries\arabic{enumi}.}}

\item   \textbf{Aluno:} Victor Barros de Miranda Neves \mbox{\sffamily{\bfseries{[Doc. \ref{app:orientacoes-bia}]}}}\\
        \textbf{Projeto:} Uso da Internet das Coisas no acesso multimodal a informações em um Smart Campus.\\
        \textbf{Tema:} Comunica\c{c}\~{o}es sem Fio \\
        \textbf{Categoria:} Iniciação acadêmica (Graduando em Ciência da Computação) – Universidade Federal de Pernambuco.\\
        \textbf{Per\'{\i}odo:} 01 de Outubro de 2014 a 30 de Setembro de 2015 \\
        \textbf{Financiamento:} Programa Institucional de Bolsa de Incentivo Acadêmico – BIA, Fundação de Amparo à Ciência e Tecnologia do Estado de Pernambuco (FACEPE), processo número: BIA-0155-1.03/14.

\end{enumerate}

%%%%%%%%%%%%%%%%%%%%%%%%%%%%%%%%%%%%%%%%%%%%%%%%%%%%%%%%%%%%%%%%%%%%%%%%%%%%%%%
% Subgrupo 1.2 - Participa\c{c}\~{a}o em Comiss\~{o}es Examinadoras
%%%%%%%%%%%%%%%%%%%%%%%%%%%%%%%%%%%%%%%%%%%%%%%%%%%%%%%%%%%%%%%%%%%%%%%%%%%%%%%
\subsection{Participa\c{c}\~{a}o em Comiss\~{o}es Examinadoras}
\vspace{0.3cm}

\subsubsection{Bancas Examinadoras de Concurso}
\vspace{0.3cm}

\begin{enumerate}
\renewcommand{\labelenumi}{{\large\bfseries\arabic{enumi}.}}
\vspace{0.3cm}

\item       \textbf{Descrição:} \Doc \\
            \textbf{Instituição:} \\
            \textbf{Data/Período:} 

\item       \textbf{Descrição:} Parecerista do(s) curso(s) de Sistemas de Informação da Avaliação de Cursos Superiores do Guia do Estudante (GE) 2015 \mbox{\sffamily{\bfseries{[Doc. \ref{app:2015-guia-estudante}]}}}\\
            \textbf{Instituição:} Guia do Estudante / Editora Abril \\
            \textbf{Data/Período:} 2015 %16 de Agosto de 2015

\item       \textbf{Descrição:} Membro Titular Externo da Comissão Examinadora do Concurso Público de Provas e Títulos para Professor da Carreira de Magistério Superior, da Classe A, da área ENSISO (Engenharia e Sistemas de Software), Matéria: Interface Homem-Computador, Tecnologia Educacional, em regime de Dedicação Exclusiva \mbox{\sffamily{\bfseries{[Doc. \ref{app:2014-ufrpe-edital-082014}]}}} \\
            \textbf{Instituição:} Universidade Federal Rural de Pernambuco (UFRPE)\\
            \textbf{Data/Período:} 09 a 12 de Março de 2015

\item       \textbf{Descrição:} Parecerista do(s) curso(s) de Sistemas de Informação da Avaliação de Cursos Superiores do Guia do Estudante (GE) 2014 \mbox{\sffamily{\bfseries{[Doc. \ref{app:2014-guia-estudante}]}}}\\
            \textbf{Instituição:} Guia do Estudante / Editora Abril \\
            \textbf{Data/Período:} 2014 %22 de Abril de 2014
\end{enumerate}

%------------------------------------------------------------------------------

\subsubsection{Bancas Congressos de Inicia\c{c}\~{a}o Cient\'{i}fica ou de Extens\~{a}o}
\vspace{0.3cm}

\begin{enumerate}
\renewcommand{\labelenumi}{{\large\bfseries\arabic{enumi}.}}
\vspace{0.3cm}

\item       \textbf{Descrição:} \Doc \\
            \textbf{Instituição:} \\
            \textbf{Data/Período:} 

\item       \textbf{Descrição:} Participação na Comissão Avaliadora dos trabalhos na área de Ciências Exatas apresentados na 19ª Jornada de Iniciação Científica da FACEPE \mbox{\sffamily{\bfseries{[Doc. \ref{app:2015-jic-facepe}]}}} \\
            \textbf{Instituição:} Fundação de Amparo a Ciência e Tecnologia do Estado de Pernambuco (FACEPE) \\
            \textbf{Data/Período:} 09 a 12 de Junho de 2015

\item       \textbf{Descrição:} Participação no Processo de avaliação do Relatório Final do Programa Institucional de Bolsas de Iniciação Científica – PIBIC/CNPq/UFPE (2013/2014), bem como do Resumo do XXII Congresso de Iniciação Científica da UFPE (2014) do(s) seguinte(s) projeto(s): 14014723PO \mbox{\sffamily{\bfseries{[Doc. \ref{app:2014-pibic-ufpe}]}}} \\
            \textbf{Instituição:} PIBIC/CNPq/UFPE\\
            \textbf{Data/Período:} 29 de Setembro de 2014

\end{enumerate}

%------------------------------------------------------------------------------

\subsubsection{Bancas de Trabalho de Conclus\~{a}o de Curso}
\vspace{0.3cm}

\begin{enumerate}
\renewcommand{\labelenumi}{{\large\bfseries\arabic{enumi}.}}
\vspace{0.3cm}

\item       \textbf{Aluno:} \Doc \\
            \textbf{T\'{\i}tulo da Monografia:} \\
            \textbf{Curso:} Curso de Gradua\c{c}\~{a}o em Engenharia da Computa\c{c}\~{a}o\\
            \textbf{Universidade:} Universidade Federal de Pernambuco (UFPE)\\
            \textbf{Data:} 

\item       \textbf{Aluno:} Ronald Dener Bezerra Pessa \mbox{\sffamily{\bfseries{[Doc. \ref{app:bancas-tcc}]}}}\\
            \textbf{T\'{\i}tulo da Monografia:} Internet das Coisas na Indústria Civil Brasileira\\
            \textbf{Curso:} Curso de Gradua\c{c}\~{a}o em Engenharia da Computa\c{c}\~{a}o\\
            \textbf{Universidade:} Universidade Federal de Pernambuco (UFPE)\\
            \textbf{Data:} 02 de Fevereiro de 2016

\item       \textbf{Aluno:} Laio Fonseca Marinho Alcântara \mbox{\sffamily{\bfseries{[Doc. \ref{app:bancas-tcc}]}}} \\
            \textbf{T\'{\i}tulo da Monografia:} Revisão Sistemática Sobre a Adoção de Software-Defined Networks (SDN)\\
            \textbf{Curso:} Curso de Gradua\c{c}\~{a}o em Sistemas de Informa\c{c}\~{a}o\\
            \textbf{Universidade:} Universidade Federal de Pernambuco (UFPE)\\
            \textbf{Data:} 22 de Janeiro de 2016

\item       \textbf{Aluno:} Carlos Eduardo da Costa Silva \mbox{\sffamily{\bfseries{[Doc. \ref{app:bancas-tcc}]}}} \\
            \textbf{T\'{\i}tulo da Monografia:} Revisão Exploratória de Plataformas de Backend as a Service\\
            \textbf{Curso:} Curso de Gradua\c{c}\~{a}o em Sistemas de Informa\c{c}\~{a}o\\
            \textbf{Universidade:} Universidade Federal de Pernambuco (UFPE)\\
            \textbf{Data:} 21 de Janeiro de 2016

\item       \textbf{Aluno:} Natalia Paola de Vasconcelos Cometti \mbox{\sffamily{\bfseries{[Doc. \ref{app:bancas-tcc}]}}} \\
            \textbf{T\'{\i}tulo da Monografia:} Um Estudo sobre a Tecnologia Blockchain e sua Aplicação em Sistemas de Votação\\
            \textbf{Curso:} Curso de Gradua\c{c}\~{a}o em Ciência da Computa\c{c}\~{a}o\\
            \textbf{Universidade:} Universidade Federal de Pernambuco (UFPE)\\
            \textbf{Data:} 20 de Janeiro de 2016

\item       \textbf{Aluno:} José Welington de Almeida Filho \mbox{\sffamily{\bfseries{[Doc. \ref{app:bancas-tcc}]}}}\\
            \textbf{T\'{\i}tulo da Monografia:} Uma proposta de uso de Gamification para o ensino de software.\\
            \textbf{Curso:} Graduação em Ciência da Computação\\
            \textbf{Universidade:} Universidade Federal de Pernambuco (UFPE)\\
            \textbf{Data:} 28 de Julho de 2015\\

\item       \textbf{Aluna:} Isabel Oliveira Jordão do Amaral \mbox{\sffamily{\bfseries{[Doc. \ref{app:bancas-tcc}]}}}\\
            \textbf{T\'{\i}tulo da Monografia:} Uma investigação do cuidado dos usuários na disponibilização de informações pessoais em Mídias Sociais.\\
            \textbf{Curso:} Graduação em Sistemas de Informação\\
            \textbf{Universidade:} Universidade Federal de Pernambuco (UFPE)\\
            \textbf{Data:} 13 de Julho de 2015\\

\item       \textbf{Aluno:} Victor Chaves Casé \mbox{\sffamily{\bfseries{[Doc. \ref{app:bancas-tcc}]}}}\\
            \textbf{T\'{\i}tulo da Monografia:} Em Direção a uma Abordagem para o Desenvolvimento de Aplicativos Móveis Multiplataforma.\\
            \textbf{Curso:} Curso de Gradua\c{c}\~{a}o em Engenharia da Computa\c{c}\~{a}o\\
            \textbf{Universidade:} Universidade Federal de Pernambuco (UFPE)\\
            \textbf{Data:} 27 de Fevereiro de 2015

\item       \textbf{Aluno:} Breno Rafael da Silva Passos \mbox{\sffamily{\bfseries{[Doc. \ref{app:bancas-tcc}]}}}\\
            \textbf{T\'{\i}tulo da Monografia:} Entendendo Abordagens de Gamefication Aplicadas à Educação: Uma Revisão Sistemática da Literatura.\\
            \textbf{Curso:} Curso de Gradua\c{c}\~{a}o em Sistemas de Informa\c{c}\~{a}o\\
            \textbf{Universidade:} Universidade Federal de Pernambuco (UFPE)\\
            \textbf{Data:} 02 de Março de 2015

\end{enumerate}

%------------------------------------------------------------------------------

\subsubsection{Bancas de Disserta\c{c}\~{a}o de Mestrado}
\vspace{0.3cm}

\begin{enumerate}
\renewcommand{\labelenumi}{{\large\bfseries\arabic{enumi}.}}
\vspace{0.3cm}

% TODO
\item       \textbf{Candidato:} Vinicius Nordi Esperança \colorbox{yellow}{\mbox{\sffamily{\bfseries{[Doc. \ref{app:2016-msc-vne}]}}}}\\
            \textbf{T\'{\i}tulo da Disserta\c{c}\~{a}o:} Uma abordagem ágil dirigida por modelos para distribuição tardia de aplicações\\
            \textbf{Tipo:} Acadêmico\\
            \textbf{Programa:} Programa de Pós-graduação em Ciência da Computação (PPG-CC)\\
            \textbf{Universidade:} Universidade Federal de São Carlos (UFSCar)\\
            \textbf{Data:} 07 de Março de 2016

\item       \textbf{Candidato:} Herbertt Barros Mangueira Diniz \mbox{\sffamily{\bfseries{[Doc. \ref{app:2016-msc-hbmd}]}}}\\
            \textbf{T\'{\i}tulo da Disserta\c{c}\~{a}o:} Linguagem Específica de Domínio para Abstração de Solução de Processamento de Eventos Complexos\\
            \textbf{Tipo:} Acadêmico\\
            \textbf{Programa:} Programa de Pós-Graduação em Ciência da Computação\\
            \textbf{Universidade:} Universidade Federal de Pernambuco (UFPE)\\
            \textbf{Data:} 04 de Março de 2016

\item       \textbf{Candidato:} Ricardo Marinho de Melo \mbox{\sffamily{\bfseries{[Doc. \ref{app:2016-msc-rmm}]}}}\\
            \textbf{T\'{\i}tulo da Disserta\c{c}\~{a}o:} Uma Arquitetura de Referência Guiada à Conformidades de Segurança para Web Services Baseados em Nuvem Privada\\
            \textbf{Tipo:} Acadêmico\\
            \textbf{Programa:} Programa de Pós-Graduação em Ciência da Computação\\
            \textbf{Universidade:} Universidade Federal de Pernambuco (UFPE)\\
            \textbf{Data:} 03 de Março de 2016

\item       \textbf{Candidato:} Hilson Barbosa da Silva \mbox{\sffamily{\bfseries{[Doc. \ref{app:2016-mprof-hbs}]}}}\\
            \textbf{T\'{\i}tulo da Disserta\c{c}\~{a}o:} Uma Investigação sobre o Processo Migratório para a Plataforma de Computação em Nuvem no Brasil\\
            \textbf{Tipo:} Profissional\\
            \textbf{Programa:} Programa de Pós-Graduação em Ciência da Computação\\
            \textbf{Universidade:} Universidade Federal de Pernambuco (UFPE)\\
            \textbf{Data:} 22 de Janeiro de 2016

\item       \textbf{Candidato:} José Rafael Moraes Garcia da Rocha \mbox{\sffamily{\bfseries{[Doc. \ref{app:2016-mprof-jrmgr}]}}}\\
            \textbf{T\'{\i}tulo da Disserta\c{c}\~{a}o:} Developing programming skills on digital native children through the interaction with smart devices\\
            \textbf{Tipo:} Profissional\\
            \textbf{Programa:} Programa de Pós-Graduação em Ciência da Computação\\
            \textbf{Universidade:} Universidade Federal de Pernambuco (UFPE)\\
            \textbf{Data:} 15 de Janeiro de 2016

\item       \textbf{Candidato:} Helaine Solange Lins Barreiros \mbox{\sffamily{\bfseries{[Doc. \ref{app:2015-msc-hslb}]}}} \\
            \textbf{T\'{\i}tulo da Disserta\c{c}\~{a}o:} Análise da Completude dos Relatos de Experimentos em Elasticidade na Computação em Nuvem: Um Mapeamento Sistemático\\
            \textbf{Tipo:} Acadêmico\\
            \textbf{Programa:} Programa de Pós-Graduação em Ciência da Computação\\
            \textbf{Universidade:} Universidade Federal de Pernambuco (UFPE)\\
            \textbf{Data:} 31 de Agosto de 2015

\item       \textbf{Candidato:} José Paulo da Silva Lima \mbox{\sffamily{\bfseries{[Doc. \ref{app:2015-msc-jpsl}]}}} \\
            \textbf{T\'{\i}tulo da Disserta\c{c}\~{a}o:} Validação de Dados Através de Hashes Criptográficos: Uma Avaliação na Perícia Forense Computacional Brasileira\\
            \textbf{Tipo:} Acadêmico\\
            \textbf{Programa:} Programa de Pós-Graduação em Ciência da Computação\\
            \textbf{Universidade:} Universidade Federal de Pernambuco (UFPE)\\
            \textbf{Data:} 31 de Agosto de 2015

\item       \textbf{Candidato:} Giselle Cirstina Moreno Tavares \mbox{\sffamily{\bfseries{[Doc. \ref{app:2015-mprof-gcmt}]}}} \\
            \textbf{T\'{\i}tulo da Disserta\c{c}\~{a}o:} Métricas em Projetos BPM no Contexto de Tecnologia da Informação: Um Estudo de Mapeamento Sistemático\\
            \textbf{Tipo:} Profissional\\
            \textbf{Programa:} Programa de Pós-Graduação em Ciência da Computação\\
            \textbf{Universidade:} Universidade Federal de Pernambuco (UFPE)\\
            \textbf{Data:} 18 de Junho de 2015

\item       \textbf{Candidato:} Álvaro Magnum Barbosa Neto \mbox{\sffamily{\bfseries{[Doc. \ref{app:2015-mprof-ambn}]}}} \\
            \textbf{T\'{\i}tulo da Disserta\c{c}\~{a}o:} Em Direção a um Ambiente de Desenvolvimento de Software Orientado por Comportamento\\
            \textbf{Tipo:} Profissional\\
            \textbf{Programa:} Programa de Pós-Graduação em Ciência da Computação\\
            \textbf{Universidade:} Universidade Federal de Pernambuco (UFPE)\\
            \textbf{Data:} 22 de Maio de 2015

\item       \textbf{Candidato:} Paulo Artur de Sousa Duarte \mbox{\sffamily{\bfseries{[Doc. \ref{app:2014-msc-ufc-pasd}]}}} \\
            \textbf{T\'{\i}tulo da Disserta\c{c}\~{a}o:} Uma Abordagem Dirigida por Modelos para Configuração de Aquisição de Contexto Intermediada por Middleware\\
            \textbf{Tipo:} Acadêmico\\
            \textbf{Programa:} Programa de Pós-Graduação em Ciências da Computação\\
            \textbf{Universidade:} Universidade Federal do Ceará (UFC)\\
            \textbf{Data:} 14 de Novembro de 2014

\item       \textbf{Candidato:} João Carlos Sedraz Silva \mbox{\sffamily{\bfseries{[Doc. \ref{app:2014-msc-jcss}]}}} \\
            \textbf{T\'{\i}tulo da Disserta\c{c}\~{a}o:} Análise de efetividade de componente curricular para motivar colaboradores em projetos de Software Público Brasileiro\\
            \textbf{Tipo:} Acadêmico\\
            \textbf{Programa:} Programa de Pós-Graduação em Ciência da Computação\\
            \textbf{Universidade:} Universidade Federal de Pernambuco (UFPE)\\
            \textbf{Data:} 26 de Setembro de 2014

\item       \textbf{Candidato:} Thiago Jamir e Silva \mbox{\sffamily{\bfseries{[Doc. \ref{app:2014-msc-tjs}]}}} \\
            \textbf{T\'{\i}tulo da Disserta\c{c}\~{a}o:} Uma Arquitetura de Cloud Storage para Backup de Arquivos\\
            \textbf{Tipo:} Acadêmico\\
            \textbf{Programa:} Programa de Pós-Graduação em Ciência da Computação\\
            \textbf{Universidade:} Universidade Federal de Pernambuco (UFPE)\\
            \textbf{Data:} 05 de Setembro de 2014

\item       \textbf{Candidato:} Jean Louis Brasil Fernandes Costa \mbox{\sffamily{\bfseries{[Doc. \ref{app:2014-msc-jlbfc}]}}} \\
            \textbf{T\'{\i}tulo da Disserta\c{c}\~{a}o:} Uma Proposta de Solução para Distribuição de Conteúdos Educacionais\\
            \textbf{Tipo:} Acadêmico\\
            \textbf{Programa:} Programa de Pós-Graduação em Ciência da Computação\\
            \textbf{Universidade:} Universidade Federal de Pernambuco (UFPE)\\
            \textbf{Data:} 02 de Setembro de 2014

\item       \textbf{Candidato:} Héldon José Oliveira Albuquerque \mbox{\sffamily{\bfseries{[Doc. \ref{app:2014-msc-hjoa}]}}} \\
            \textbf{T\'{\i}tulo da Disserta\c{c}\~{a}o:} Solução para Interoperabilidade de Protocolos em Smarthomes\\
            \textbf{Tipo:} Acadêmico\\
            \textbf{Programa:} Programa de Pós-Graduação em Sistemas e Computação\\
            \textbf{Universidade:} Universidade Federal do Rio Grande do Norte (UFRN)\\
            \textbf{Data:} 29 de Agosto de 2014

\item       \textbf{Candidato:} Welington Manoel da Silva \mbox{\sffamily{\bfseries{[Doc. \ref{app:2014-msc-wms}]}}} \\
            \textbf{T\'{\i}tulo da Disserta\c{c}\~{a}o:} Go!SIP: Um Framework de Privacidade para Cidades Inteligentes Baseado em Pessoas Como Sensores\\
            \textbf{Tipo:} Acadêmico \\
            \textbf{Programa:} Programa de Pós-Graduação em Ciência da Computação\\
            \textbf{Universidade:} Universidade Federal de Pernambuco (UFPE)\\
            \textbf{Data:} 25 de Agosto de 2014
\end{enumerate}

%------------------------------------------------------------------------------

\subsubsection{Bancas de Qualifica\c{c}\~ao / Proposta de Tese de Doutorado}
\vspace{0.3cm}

\begin{enumerate}
\renewcommand{\labelenumi}{{\large\bfseries\arabic{enumi}.}}
\vspace{0.3cm}

\item       \textbf{Candidato:} Paola Rodrigues Godoy Accioly \mbox{\sffamily{\bfseries{[Doc. \ref{app:2016-quali-phd-prga}]}}} \\
            \textbf{T\'{\i}tulo da Qualifica\c{c}\~{a}o:} Understanding Collaboration Conflicts Characteristics\\
            \textbf{Programa:} Programa de Pós-Graduação em Ciência da Computação\\
            \textbf{Universidade:} Universidade Federal de Pernambuco (UFPE)\\
            \textbf{Data:} 01 de Março de 2016

\item       \textbf{Candidato:} Ioram Schechtman Sette \mbox{\sffamily{\bfseries{[Doc. \ref{app:2015-quali-phd-iss}]}}} \\
            \textbf{T\'{\i}tulo da Qualifica\c{c}\~{a}o:} Authorisation Policy Federations for Heterogeneous IaaS Clouds\\
            \textbf{Programa:} Programa de Pós-Graduação em Ciência da Computação\\
            \textbf{Universidade:} Universidade Federal de Pernambuco (UFPE)\\
            \textbf{Data:} 18 de Dezembro de 2015

\item       \textbf{Candidato:} Thiago Monteiro Prota \mbox{\sffamily{\bfseries{[Doc. \ref{app:2015-quali-phd-tmp}]}}} \\
            \textbf{T\'{\i}tulo da Qualifica\c{c}\~{a}o:} Análises Estrutural e Comportamental Orientadas a Conformidade para o Desenvolvimento de Aplicações Multimídia\\
            \textbf{Programa:} Programa de Pós-Graduação em Ciência da Computação\\
            \textbf{Universidade:} Universidade Federal de Pernambuco (UFPE)\\
            \textbf{Data:} 18 de Setembro de 2015

\item       \textbf{Candidato:} Bruno Neiva Moreno \mbox{\sffamily{\bfseries{[Doc. \ref{app:2015-quali-phd-bnm}]}}} \\
            \textbf{T\'{\i}tulo da Qualifica\c{c}\~{a}o:} Análise de Encontro Espaço-Temporais de Usuários de Redes Sociais Baseadas em Localização\\
            \textbf{Programa:} Programa de Pós-Graduação em Ciência da Computação\\
            \textbf{Universidade:} Universidade Federal de Pernambuco (UFPE)\\
            \textbf{Data:} 06 de Abril de 2015

\item       \textbf{Candidato:} Julio Cesar Damasceno \mbox{\sffamily{\bfseries{[Doc. \ref{app:2015-quali-phd-jcd}]}}} \\
            \textbf{T\'{\i}tulo da Qualifica\c{c}\~{a}o:} UCloud: Uma Proposta para Datacenter Como Serviço para Ambientes de Computação em Nuvem\\
            \textbf{Programa:} Programa de Pós-Graduação em Ciência da Computação\\
            \textbf{Universidade:} Universidade Federal de Pernambuco (UFPE)\\
            \textbf{Data:} 20 de Março de 2015

\item       \textbf{Candidato:} Liliane Sheyla da Silva Fonseca \mbox{\sffamily{\bfseries{[Doc. \ref{app:2015-quali-phd-lssf}]}}} \\
            \textbf{T\'{\i}tulo da Qualifica\c{c}\~{a}o:} Towards a supportive approach to conduct software engineering experiments\\
            \textbf{Programa:} Programa de Pós-Graduação em Ciência da Computação\\
            \textbf{Universidade:} Universidade Federal de Pernambuco (UFPE)\\
            \textbf{Data:} 11 de Fevereiro de 2015

\item       \textbf{Candidato:} Rodrigo da Cruz Fujioka \mbox{\sffamily{\bfseries{[Doc. \ref{app:2015-quali-phd-rcf}]}}} \\
            \textbf{T\'{\i}tulo da Qualifica\c{c}\~{a}o:} Uma Linha de Produto de Software (LPS) Utilizando uma Arquitetura SOA para Geração de Ambientes Virtuais Tridimensionais\\
            \textbf{Programa:} Programa de Pós-Graduação em Ciência da Computação\\
            \textbf{Universidade:} Universidade Federal de Pernambuco (UFPE)\\
            \textbf{Data:} 23 de Janeiro de 2015

\item       \textbf{Candidato:} Elias Adriano Nogueira da Silva \mbox{\sffamily{\bfseries{[Doc. \ref{app:2014-quali-phd-eans}]}}} \\
            \textbf{T\'{\i}tulo da Qualifica\c{c}\~{a}o:} \\
            \textbf{Programa:} Programa de Pós-Graduação em Ciências de Computação e Matemática Computacional\\
            \textbf{Universidade:} Instituto de Ciências Matemáticas e de Computação (ICMC), Universidade de São Paulo (USP)\\
            \textbf{Data:} 21 de Agosto de 2014
\end{enumerate}

%------------------------------------------------------------------------------

\subsubsection{Bancas de Tese de Doutorado}
%\vspace{0.3cm}

\begin{enumerate}
\renewcommand{\labelenumi}{{\large\bfseries\arabic{enumi}.}}
\vspace{0.3cm}

\item       \textbf{Candidato:} Rodrigo Alves Costa \mbox{\sffamily{\bfseries{[Doc. \ref{app:2016-phd-rac2}]}}} \\
            \textbf{T\'{\i}tulo da Tese:} Uma Abordagem para Combate da Fraude de Clique Baseada em CAPTCHAs Clicáveis\\
            \textbf{Programa:} Programa de Pós-Graduação em Ciência da Computação\\
            \textbf{Universidade:} Universidade Federal de Pernambuco (UFPE)\\
            \textbf{Data:} 04 de Março de 2016

\item       \textbf{Candidato:} Julio Cesar Damasceno \mbox{\sffamily{\bfseries{[Doc. \ref{app:2015-phd-jcd}]}}} \\
            \textbf{T\'{\i}tulo da Tese:} UCloud: Uma Abordagem para Implantação de Nuvem Privada para a Administração Pública Federal\\
            \textbf{Programa:} Programa de Pós-Graduação em Ciência da Computação\\
            \textbf{Universidade:} Universidade Federal de Pernambuco (UFPE)\\
            \textbf{Data:} 03 de Setembro de 2015

\item       \textbf{Candidato:} Rodrigo Gusmão de Carvalho Rocha \mbox{\sffamily{\bfseries{[Doc. \ref{app:2015-phd-rgcr}]}}} \\
            \textbf{T\'{\i}tulo da Tese:} Uma Abordagem Baseada em Ontologias e Raciocínio Baseado em Casos para Apoiar o Desenvolvimento Distribuído de Software\\
            \textbf{Programa:} Programa de Pós-Graduação em Ciência da Computação\\
            \textbf{Universidade:} Universidade Federal de Pernambuco (UFPE)\\
            \textbf{Data:} 20 de Fevereiro de 2015

\item       \textbf{Candidato:} Cleviton Vinicius Fonseca Monteiro \mbox{\sffamily{\bfseries{[Doc. \ref{app:2015-phd-cvfm}]}}} \\
            \textbf{T\'{\i}tulo da Tese:} Innovative Behavior Model of Software Professionals\\
            \textbf{Programa:} Programa de Pós-Graduação em Ciência da Computação\\
            \textbf{Universidade:} Universidade Federal de Pernambuco (UFPE)\\
            \textbf{Data:} 25 de Novembro de 2014
\end{enumerate}

%------------------------------------------------------------------------------

\subsubsection{Participa\c{c}\~{a}o em Bancas de Sele\c{c}\~{a}o para ingresso e exames de Qualifica\c{c}\~ao de Programa de P\'{o}-Gradua\c{c}\~{a}o \textit{Stricto Sensu}}

\begin{enumerate}
\renewcommand{\labelenumi}{{\large\bfseries\arabic{enumi}.}}
\vspace{0.3cm}

\item       \textbf{Descrição:} Membro Externo da Comissão de Seleção para o Doutorado Sanduíche - PDSE da CAPES \mbox{\sffamily{\bfseries{[Doc. \ref{app:2015-pgcomp-pdse}]}}} \\
            \textbf{Programa:} Programa de Pós-Graduação em Ciência da Computação (PGCOMP) do aluno Alcemir Rodrigues Santos\\
            \textbf{Instituição:} Universidade Federal da Bahia (UFBA) \\
            \textbf{Per\'{\i}odo de Realiza\c{c}\~{a}o:} 15 de Junho de 2015

\end{enumerate}

%%%%%%%%%%%%%%%%%%%%%%%%%%%%%%%%%%%%%%%%%%%%%%%%%%%%%%%%%%%%%%%%%%%%%%%%%%%%%%%
% Subgrupo 1.3 - Atividades de Ensino na Graduação e na Pós-Graduação
%%%%%%%%%%%%%%%%%%%%%%%%%%%%%%%%%%%%%%%%%%%%%%%%%%%%%%%%%%%%%%%%%%%%%%%%%%%%%%%

\subsection{Atividades de Ensino na Graduação e na Pós-Graduação}
\vspace{0.3cm}

\subsubsection{Atividades de Ensino: Disciplinas na Gradua\c{c}\~{a}o}
\vspace{0.3cm}

Na Tabela \ref{Tab:Disc_Grad}, est\~{a}o listadas as disciplinas ministradas nos cursos de gradua\c{c}\~{a}o de Sistemas de Informa\c{c}\~{a}o no per\'{\i}odo de 20/08/2014 a 19/08/2016. As disciplinas est\~{a}o organizadas por semestre, com a identifica\c{c}\~{a}o do curso, da turma e da disciplina, esta \'{u}ltima com sua respectiva carga hor\'{a}ria. Adicionalmente, consta na referida Tabela a carga hor\'{a}ria m\'{e}dia semestral ao longo do per\'{\i}odo de avalia\c{c}\~{a}o.

\begin{table}[!htpb]
\centering \small
\caption{\texttt{Disciplinas de gradua\c{c}\~{a}o lecionadas no per\'{\i}odo considerado} }
\begin{tabular}{ccccc}
\toprule
\textbf{Semestre} & \textbf{Curso} & \textbf{Turma} & \textbf{Disciplina} & \textbf{C.H.} \\
\otoprule
  & Ci\^{e}ncia da Computa\c{c}\~{a}o & $I6$ & IF687 - Introdu\c{c}\~{a}o \`{a} Multim\'{\i}dia & 02\\
$2012.1$ & Ci\^{e}ncia da Computa\c{c}\~{a}o & $I6$ & IF690 - Hist\'{o}ria e Futuro da Computa\c{c}\~{a}o & 02\\
  & Engenharia da Computa\c{c}\~{a}o & $I6$ & IF690 - Hist\'{o}ria e Futuro da Computa\c{c}\~{a}o & 02\\
\cmidrule{1-5}
  & Ci\^{e}ncia da Computa\c{c}\~{a}o & $I6$ & IF687 - Introdu\c{c}\~{a}o \`{a} Multim\'{\i}dia & 02\\
$2012.2$ & Ci\^{e}ncia da Computa\c{c}\~{a}o & $I6$ & IF690 - Hist\'{o}ria e Futuro da Computa\c{c}\~{a}o & 02\\
  & Engenharia da Computa\c{c}\~{a}o & $E6$ & IF690 - Hist\'{o}ria e Futuro da Computa\c{c}\~{a}o & 02\\
  & Engenharia da Computa\c{c}\~{a}o & $E0$ & ES348 - Teoria da Informa\c{c}\~{a}o & 04\\
\cmidrule{1-5}
\multicolumn{4}{r}{\textbf{Carga Hor\'{a}ria M\'{e}dia Semestral} (2012)} & \textbf{08} \\
\otoprule
& Ci\^{e}ncia da Computa\c{c}\~{a}o & $I6$ & IF687 - Introdu\c{c}\~{a}o \`{a} Multim\'{\i}dia & 02\\
$2013.1$ & Ci\^{e}ncia da Computa\c{c}\~{a}o & $I6$ & IF690 - Hist\'{o}ria e Futuro da Computa\c{c}\~{a}o & 02\\
  & Engenharia da Computa\c{c}\~{a}o & $E6$ & IF690 - Hist\'{o}ria e Futuro da Computa\c{c}\~{a}o & 02\\
  & Engenharia da Computa\c{c}\~{a}o & $E0$ & IF856 - T\'{o}picos Avan\c{c}ados em Proc. Sinais & 04\\
\cmidrule{1-5}
& Ci\^{e}ncia da Computa\c{c}\~{a}o & $I6$ & IF687 - Introdu\c{c}\~{a}o \`{a} Multim\'{\i}dia & 02\\
$2013.2$ & Ci\^{e}ncia da Computa\c{c}\~{a}o & $I6$ & IF690 - Hist\'{o}ria e Futuro da Computa\c{c}\~{a}o & 02\\
  & Engenharia da Computa\c{c}\~{a}o & $I6$ & IF690 - Hist\'{o}ria e Futuro da Computa\c{c}\~{a}o & 02\\
\cmidrule{1-5}
\multicolumn{4}{r}{\textbf{Carga Hor\'{a}ria M\'{e}dia Semestral} (2013)} & \textbf{08} \\
\bottomrule
\end{tabular}
\label{Tab:Disc_Grad}
\end{table}

%%%%%%%%%%%%%%%%%%%%%%%%%%%%%%%%%%%%%%%%%%%%%%%%%%%%%%%%%%%%%%%%%%%%%%%%%%%%%%%
% Subgrupo 1.4 - Avaliação Didática Docente pelo Discente
%%%%%%%%%%%%%%%%%%%%%%%%%%%%%%%%%%%%%%%%%%%%%%%%%%%%%%%%%%%%%%%%%%%%%%%%%%%%%%%
\vspace{5.0cm}
\subsection{Avaliação Didática Docente pelo Discente}
\vspace{0.3cm}

%Nada a declarar neste subgrupo.

\begin{enumerate}
\renewcommand{\labelenumi}{{\large\bfseries\arabic{enumi}.}}

% TODO
\item   \textbf{Período:} 2016.1 \colorbox{yellow}{\textbf{[Doc. \ref{evaluation:2016-1}]}}\\
        \textbf{Disciplina:} IF1006 - TÓPICOS AVANÇADOS EM SI 3, Turma: S0\\
        \textbf{Disciplina:} IF977 - ENGENHARIA DE SOFTWARE, Turma: S4\\
        \textbf{Nota:} .

\item   \textbf{Período:} 2015.2 \textbf{[Doc. \ref{evaluation:2015-2}]}\\
        \textbf{Disciplina:} IF987 - AUDITORIA E SEGURANÇA EM SI, Turma: S60\\
        \textbf{Disciplina:} IF977 - ENGENHARIA DE SOFTWARE, Turma: S4\\
        \textbf{Disciplina:} IF987 - AUDITORIA E SEGURANÇA EM SI, Turma: S7\\
        \textbf{Nota:} 10.

\item   \textbf{Período:} 2015.1 \textbf{[Doc. \ref{evaluation:2015-1}]}\\
        \textbf{Disciplina:} IF1006 - TÓPICOS AVANÇADOS EM SI 3, Turma: S0\\
        \textbf{Disciplina:} IF977 - ENGENHARIA DE SOFTWARE, Turma: S4\\
        \textbf{Nota:} 10.

\item   \textbf{Período:} 2014.2 \\
        \textbf{Disciplina:} IF1006 - TÓPICOS AVANÇADOS EM SI 3, Turma: S0\\
        \textbf{Disciplina:} IF977 - ENGENHARIA DE SOFTWARE, Turma: S4\\
        \textbf{Nota:} \textbf{Não houve avaliação didática docente pelo discente}.

\end{enumerate}

%%%%%%%%%%%%%%%%%%%%%%%%%%%%%%%%%%%%%%%%%%%%%%%%%%%%%%%%%%%%%%%%%%%%%%%%%%%%%%%
% Grupo 2: Atividades de Produ\c{c}\~{a}o Cient\'{\i}fica e T\'{e}cnica, Art\'{i}stica e Cultural
%%%%%%%%%%%%%%%%%%%%%%%%%%%%%%%%%%%%%%%%%%%%%%%%%%%%%%%%%%%%%%%%%%%%%%%%%%%%%%%
\newpage
\section{Atividades de Produ\c{c}\~{a}o Cient\'{\i}fica e T\'{e}cnica, Art\'{i}stica e Cultural}

%%%%%%%%%%%%%%%%%%%%%%%%%%%%%%%%%%%%%%%%%%%%%%%%%%%%%%%%%%%%%%%%%%%%%%%%%%%%%%%
% Subgrupo 2.1 - Produtividade de Pesquisa
%%%%%%%%%%%%%%%%%%%%%%%%%%%%%%%%%%%%%%%%%%%%%%%%%%%%%%%%%%%%%%%%%%%%%%%%%%%%%%%
\subsection{Produtividade de Pesquisa}
\vspace{0.3cm}

%------------------------------------------------------------------------------

\subsubsection{Bolsista de produtividade em pesquisa e em inova\c{c}\~{a}o tecnol\'{o}gica}
\vspace{0.3cm}

Nada a declarar.

%------------------------------------------------------------------------------

\subsubsection{Participa\c{c}\~{a}o em Eventos Cient\'{\i}ficos (com apresenta\c{c}\~{a}o de trabalho ou oferecimento de cursos, palestras ou debates)}
\vspace{0.3cm}

\begin{enumerate}
\renewcommand{\labelenumi}{{\large\bfseries\arabic{enumi}.}}

\item   \textbf{Evento:} \Doc \\
        \textbf{Propósito:} \\
        \textbf{Período:} \\
        \textbf{Local:} , Brasil.

\item   \textbf{Evento:} Congresso Brasileiro de Software: Teoria e Prática \mbox{\sffamily{\bfseries{[Doc. \ref{app:2015-cbsoft}]}}} \\
        \textbf{Propósito:} Participante\\
        \textbf{Período:} 21 a 25 de Setembro de 2015\\
        \textbf{Local:} Belo Horizonte-MG, Brasil.

\item   \textbf{Evento:} XII Encontro Anual de Computação (EnAComp) \mbox{\sffamily{\bfseries{[Doc. \ref{app:2015-enacomp}]}}} \\
        \textbf{Propósito (i):} Ouvinte\\
        \textbf{Propósito (ii):} Apresentação da palestra ``Empreendedorismo e criação de negócios intensivos em software''\\
        \textbf{Propósito (ii):} Apresentação do minicurso ``Desenvolvimento de SaaS / Engenharia de Sofwtarte como Serviço'', com duração de 12 horas\\
        \textbf{Período:} 16 a 18 de Setembro de 2015\\
        \textbf{Local:} , Brasil.

\item   \textbf{Evento:} 19ª Jornada de Iniciação Científica da FACEPE \mbox{\sffamily{\bfseries{[Doc. \ref{app:2015-19jic-facepe}]}}} \\
        \textbf{Propósito:} Apresentação da palestra ``Cidades inteligentes: a contribuição de Alan Turing''\\
        \textbf{Período:} 09 a 12 de Junho de 2015\\
        \textbf{Local:} Recife-PE, Brasil.

\item   \textbf{Evento:} XI Simpósio Brasileiro de Sistemas de Informação (SBSI) \mbox{\sffamily{\bfseries{[Doc. \ref{app:2015-xi-sbsi}]}}} \\
        \textbf{Propósito (i):} Apresentação do minicurso ``Criando Startups: Métodos, Processos, Técnicas e Ferramentas'', com duração de 4 horas\\
        \textbf{Propósito (ii):} Apresentação do trabalho ``SmartCluster: Utilizando dados Públicos para Agrupar Cidades Inteligentes por Domínios''\\
        \textbf{Propósito (iii):} Apresentação do trabalho ``Um Modelo de Recomendação de Arquivos para Sistemas de Armazenamento em Nuvem''\\
        \textbf{Período:} 26 a 29 de Maio de 2015\\
        \textbf{Local:} Goiânia-GO, Brasil.

\item   \textbf{Evento:} Encontro Pedagógico 2015, promovido pelo Serviço Nacional de Aprendizagem Comercial (SENAC) \mbox{\sffamily{\bfseries{[Doc. \ref{app:2015-ep-senac}]}}} \\
        \textbf{Propósito:} Ministrante na Oficina ``Novas Tendências em Tecnologia da Informação''\\
        \textbf{Período:} 04 a 06 de Fevereiro de 2015\\
        \textbf{Local:} Recife-PE, Brasil.

\item   \textbf{Evento:} V Congresso Brasileiro de Software: Teoria e Prática \mbox{\sffamily{\bfseries{[Doc. \ref{app:2014-cbsoft}]}}} \\
        \textbf{Propósito:} Participante\\
        \textbf{Período:} 28 de Setembro a 03 de Outubro de 2014\\
        \textbf{Local:} Macei\'{o}-AL, Brasil.

\end{enumerate}

%------------------------------------------------------------------------------
% TODO
\subsubsection{Autoria de artigos completos publicados em anais de congresso, em jornais e revistas de circulação nacional e internacional na sua área}
\vspace{0.3cm}

\begin{enumerate}
\renewcommand{\labelenumi}{{\large\bfseries\arabic{enumi}.}}

% TODO
\item SILVA, Carlo Marcelo Revoredo da; SILVA, Jose Lutiano Costa da; RODRIGUES, Ricardo Batista; GATTI, Daniel Couto; MIRANDA, Pericles Cunha Barbosa de; BRITO, Kellyton dos Santos; NASCIMENTO, Leandro Marques do; ASSAD, Rodrigo Elia; QUEIROZ, Ruy José Guerra Barreto de; GARCIA, Vinicius Cardoso. A Meta-Analysis for Security Threats over the Web Ecosystem. In: IEEE/IFIP Network Operations and Management Symposium (NOMS 2016), Istanbul - Turkey, 25-29 APRIL 2016. \colorbox{yellow}{\textbf{[Doc. \ref{conf:2016-noms-b}]}}

% TODO
\item SILVA, Carlo Marcelo Revoredo da; SILVA, Jose Lutiano Costa da; RODRIGUES, Ricardo Batista; GATTI, Daniel Couto; MIRANDA, Pericles Cunha Barbosa de; BRITO, Kellyton dos Santos; NASCIMENTO, Leandro Marques do; ASSAD, Rodrigo Elia; QUEIROZ, Ruy José Guerra Barreto de; GARCIA, Vinicius Cardoso. Towards a Taxonomy for Security Threats on the Web Ecosystem. In: IEEE/IFIP Network Operations and Management Symposium (NOMS 2016), Istanbul - Turkey, 25-29 APRIL 2016. \colorbox{yellow}{\textbf{[Doc. \ref{conf:2016-noms-a}]}}

\item RODRIGUES, Ricardo Batista; GARCIA, Vinicius Cardoso; and MEIRA, Silvio Romero de Lemos. \textbf{Uma Metodologia de Recomendação Híbrida Para Sistemas de Armazenamento em Nuvem}. In: XII Workshop de Teses e Dissertações do 21th Brazilian Symposium on Multimedia and The Web (WTD - WebMedia'2015), 27 - 30 de Outubro - Manaus, AM. {\textbf{[Doc. \ref{conf:2015-wtd-webmedia}]}}

% TODO
\item AFONSO, Ricardo Alexandre; ALVARO, Alexandre; GARCIA, Vinicius Cardoso. \textbf{SCiAl: Usando Dados Públicos para Agrupar Cidades Inteligentes Alagoanas}. In: IV Simpósio Brasileiro de Tecnologia da Informação (SBTI), 14 a 16 de Outubro, Aracaju - SE. \colorbox{yellow}{\textbf{[Doc. \ref{conf:2015-sbti}]}}

\item SILVA, Carlo M. R.; LUTIANO, José; RODRIGUES, Ricardo Batista; BRITO, Kellyton Santos; DO NASCIMENTO, Leandro Marques; GATTI, Daniel; ASSAD, Rodrigo Elia; DE QUEIROZ, Ruy; GARCIA, Vinicius Cardoso. \textbf{Uma Avaliação da Proteção de Dados Sensíveis através do Navegador Web}. In: XV Simpósio Brasileiro em Segurança da Informação e de Sistemas Computacionais (SBSeg'15) - Trilha de Trabalhos Completos, 09 - 12 de Novembro - Florianópolis, SC, 2015, pp. 86-99. \textbf{[Doc. \ref{conf:2015-sbseg}]}

%\item SILVA, Carlo M. R.; LUTIANO, José; RODRIGUES, Ricardo Batista; BRITO, Kellyton Santos; DO NASCIMENTO, Leandro Marques; MIRANDA, Pericles; GATTI, Daniel; ASSAD, Rodrigo Elia; DE QUEIROZ, Ruy; GARCIA, Vinicius Cardoso. \textbf{An Evaluation for Sensitive Data Protection from Client-Side over the Web}. In: The 15th IEEE International Conference on Computer and Information Technology (CIT2015), 26-28 October 2015 - Liverpool, UK.  \textbf{[Doc. \ref{conf:2015-cti}]}

\item GAMA, Kiev Santos; WANDERLEY, Rafael Roballo; MARANHÃO, Daniel Barbosa; GARCIA, Vinicius Cardoso. \textbf{TagHunt: Uma plataforma Combinando a Internet das Coisas com Scavenger Hunt Games}. In: 7º Simpósio Brasileiro de Computação Ubíqua e Pervasiva (SBCUP), Evento do XXXV Congresso da Sociedade Brasileira de Computação, Recife, 20 a 23 de Julho de 2015. \textbf{[Doc. \ref{conf:2015-sbcup}]}

\item AFONSO, Ricardo Alexandre; BRITO, Kellyton dos Santos; DO NASCIMENTO, Clóvis Holanda; GARCIA, Vinicius Cardoso; and ÁLVARO, Alexandre. 2015. \textbf{Brazilian smart cities: using a maturity model to measure and compare inequality in cities}. In Proceedings of the 16th Annual International Conference on Digital Government Research (dg.o'15). ACM, New York, NY, USA, 230-238. doi: 10.1145/2757401.2757426 \textbf{[Doc. \ref{conf:2015-dgo-41}]}

\item BRITO, Kellyton dos Santos; COSTA, Marcos Antônio da Silva; GARCIA, Vinicius Cardoso; and MEIRA, Silvio Romero de Lemos. 2015. \textbf{Assessing the benefits of open government data: the case of Meu Congresso Nacional in Brazilian elections 2014}. In Proceedings of the 16th Annual International Conference on Digital Government Research (dg.o'15). ACM, New York, NY, USA, 89-96. doi: 10.1145/2757401.2757422 \textbf{[Doc. \ref{conf:2015-dgo-37}]}

\item RODRIGUES, Ricardo BATISTA; SILVA, Carlo Marcelo Revoredo; GARCIA, Vinicius Cardoso; MEIRA, Silvio Romero de Lemos. \textbf{Criando Startups: Métodos, Processos, Técnicas e Ferramentas}. In: Anais do XI Simpósio Brasileiro de Sistemas de Informação, Sistemas de Informação: A Visão Sociotécnica da Computação. Goiânia, GO, Brasil: Instituto de Informática, Universidade Federal de Goiás, 2015. p.699 - 702. \textbf{[Doc. \ref{conf:2015-sbsi-minicurso}]}

\item AFONSO, Ricardo Alexandre; DO NASCIMENTO, Clóvis Holanda; GARCIA, Vinicius Cardoso; ALVARO, Alexandre. \textbf{SmartCluster: Utilizando Dados Públicos para Agrupar Cidades Inteligentes por Domínios}. In: Anais do XI Simpósio Brasileiro de Sistemas de Informação, Government Information Systems (SIG) Special Track. Goiânia, GO, Brasil: Instituto de Informática, Universidade Federal de Goiás, 2015. p.699 - 702. \textbf{[Doc. \ref{conf:2015-sbsi-special-track}]}

\item RODRIGUES, Ricardo Batista; SILVA, Carlo Marcelo Revoredo; DURÃO, Frederico Araújo; ASSAD, Rodrigo Elia; GARCIA, Vinicius Cardoso; MEIRA, Silvio Romero de Lemos. \textbf{Um Modelo de Recomendação de Arquivos para Sistemas de Armazenamento em Nuvem}. In: Anais do XI Simpósio Brasileiro de Sistemas de Informação, Prediction Methods and Recommender Systems Track. Goiânia, GO, Brasil: Instituto de Informática, Universidade Federal de Goiás, 2015. p.111 - 118. \textbf{[Doc. \ref{conf:2015-sbsi-main-track}]}

\item DO NASCIMENTO, Clóvis Holanda; AFONSO, Ricardo Alexandre; GARCIA, Vinicius Cardoso; DA SILVA, Carlo Marcelo Revoredo. \textbf{Convertview:A Tool for Conversion and Visualization of Open Heterogenic Governmental Data to the Rdf Standard}. In: The 11th International Conference on Web Information Systems and Technologies (WEBIST), Lisbon - Portugal, 20 - 22 May, 2015. doi: 10.5220/0005456902530258 \textbf{[Doc. \ref{conf:2015-webist}]}

\item BURÉGIO, Vanilson André de Arruda; BRITO, Kellyton dos Santos;  ROSA, Nelson; DOS SANTOS NETO, Misael Wanderley; GARCIA, Vinicius Cardoso and MEIRA, Silvio Romero de Lemos. \textbf{Towards Government as a Social Machine}. In Proceedings of the 24th International Conference on World Wide Web Companion (WWW'15 Companion). International World Wide Web Conferences Steering Committee, Republic and Canton of Geneva, Switzerland, 1131-1136. doi: 10.1145/2740908.2743976 \textbf{[Doc. \ref{conf:2015-socm-www}]}

\item GAMA, Kiev; ROBALLO, Rafael; MARANHÃO, Daniel; GARCIA, Vinicius Cardoso. \textbf{A Web-based platform for Scavenger Hunt Games using the Internet of Things}. In: IEEE Consumer Communications \& Networking Conference (CCNC) Held in conjunction with the International Consumer Electronics Show, Las Vegas, January 09-12, 2015. doi: 10.1109/CCNC.2015.7158040 \textbf{[Doc. \ref{conf:2015-ccnc}]}

\item AFONSO, R. A. ; LIMA, L. C. ; COSTA, L. C. ; ALVARO, Alexandre ; GARCIA, V. C. . \textbf{Mapeamento Sistemático de Informática Médica para Cidades Inteligentes}. In: CBIS (Congresso Brasileiro de Informática em Saúde), 2014, Santos / SP. CBIS 2014 - XIV Congresso Brasileiro de Informática em Saúde. Santos / SP: SBIS (Sociedade Brasileira de Informática em Saúde), 07 a 10 de Dezembro, 2014. \textbf{[Doc. \ref{conf:2014-cbis}]}

\item MACHADO, Marco André Santos; DURÃO, Frederico Araújo; ASSAD, Rodrigo Elia; GARCIA, Vinicius Cardoso. \textbf{Uma Abordagem para Indexação e Buscas Full-Text Baseadas em Conteúdo para Sistemas de Armazenamento em Nuvem}. In: Trilha da Indústria do Congresso Brasileiro de Software: Teoria e prática (CBSoft), September 28 to October 03, 2014, Maceió - AL, Brazil. \textbf{[Doc. \ref{conf:2014-cbsoft-trilha-marco}]}

\item CARVALHO, Jose Fernando S.; GARCIA, Vinicius Cardoso; ASSAD, Rodrigo Elia; REBELO, Vinod. \textbf{USTORE: Uma plataforma de computação em nuvem privada para apoio a e-Science}. In: Trilha da Indústria do Congresso Brasileiro de Software: Teoria e prática (CBSoft), September 28 to October 03, 2014, Maceió - AL, Brazil. \textbf{[Doc. \ref{conf:2014-cbsoft-trilha-fish}]}

\item BRASIL, Jean Louis; GARCIA, Vinicius Cardoso; ASSAD, Rodrigo Elia; CARVALHO, Mauricio. \textbf{Uma Proposta de Solução para Distribuição de Conteúdos Educacionais}. In: Trilha da Indústria do Congresso Brasileiro de Software: Teoria e prática (CBSoft), September 28 to October 03, 2014, Maceió - AL, Brazil. \textbf{[Doc. \ref{conf:2014-cbsoft-trilha-jean}]}

\item NASCIMENTO, Leandro Marques do; GARCIA, Vinicius Cardoso; ASSAD, Rodrigo Elia; MEIRA, Silvio Romero de Lemos; DAMASCENO, Julio César; NEVES, Pedro Henriques de Faria. \textbf{Uma Ferramenta para Gerenciamento de Infraestruturas de Computação em Nuvem}. In: Trilha da Indústria do Congresso Brasileiro de Software: Teoria e prática (CBSoft), September 28 to October 03, 2014, Maceió - AL, Brazil. \textbf{[Doc. \ref{conf:2014-cbsoft-trilha-kid}]}

\item LINS, Helaine Solange; SOARES, Sergio castelo Branco; GARCIA, Vinicius Cardoso. \textbf{Análise da Qualidade de Experimentos da Computação em Nuvem}. In: IV Workshop de Teses e Dissertações do CBSoft (WTDSoft 2014) - Congresso Brasileiro de Software: Teoria e Prática (CBSoft), September 28 to October 03, 2014, Maceió - AL, Brazil. \textbf{[Doc. \ref{conf:2014-cbsoft-wtdsoft}]}

\item DA SILVA, Carlo Marcelo Revoredo; DA SILVA, José Lutiano Costa; ASSAD, Rodrigo Elia; DE QUEIROZ, Ruy José Guerra Barretto; GARCIA, Vinicius Cardoso. \textbf{Aegis: A Lightweight Tool for Prevent Frauds in Web Browsers}. In: 2014 IEEE Intelligence \& Security Informatics Conference (IEEE JISIC), September 24-26, 2014, the Hague, the Netherlands. doi: 10.1109/JISIC.2014.11 \textbf{[Doc. \ref{conf:2014-JISIC}]}%\mbox{\sffamily{\bfseries{[Doc. \ref{conf:2014-JISIC}]}}

\item DA SILVA, Carlo Marcelo Revoredo; DA SILVA, Jose Lutiano Costa; MELO, Ricardo Marinho; RODRIGUES, Ricardo Batista; LUCIEN, Lucien Rocha; DE MELO, Sandro Pereira; COLARES Adolfo; GARCIA, Vinicius Cardoso. \textbf{A Privacy Maturity Model for Cloud Storage Services}. In: 2014 IEEE International Conference on Cloud Computing (IEEE CLOUD), June 27 - July 2, 2014, Alaska, USA. doi: 10.1109/CLOUD.2014.135 \mbox{\sffamily{\bfseries{[Doc. \ref{conf:2014-cloud-wip}]}}}

\end{enumerate}

%------------------------------------------------------------------------------

% \subsubsection{Autoria de artigos completos publicados em anais de congresso, em jornais e revistas de circulação internacional na sua área}
% \vspace{0.3cm}

% \begin{enumerate}
% \renewcommand{\labelenumi}{{\large\bfseries\arabic{enumi}.}}

% \item .

% \end{enumerate}

%------------------------------------------------------------------------------
% TODO
\subsubsection{Arbitragem de Artigos Técnico-Científicos Nacionais e Internacionais na sua área de atuação}
\vspace{0.3cm}

\begin{enumerate}
\renewcommand{\labelenumi}{{\large\bfseries\arabic{enumi}.}}

\item   \textbf{Peri\'{o}dico:} IEEE’s Transactions on Software Engineering \textbf{[Doc. \ref{reviewer:2016-tse}]}\\
        \textbf{Editora:} IEEE Computer Society\\
        \textbf{ISSN:} 0098-5589\\
        \textbf{URL:} \url{http://www.computer.org/web/tse}

\item   \textbf{Peri\'{o}dico:} IEEE Transactions on Services Computing \textbf{[Doc. \ref{reviewer:2015-tsc}]}\\
        \textbf{Editora:} IEEE Computer Society\\
        \textbf{ISSN:} 1939-1374\\
        \textbf{URL:} \url{http://www.computer.org/web/tsc}

\item   \textbf{Peri\'{o}dico:} Applied Computing and Informatics \textbf{[Doc. \ref{reviewer:2015-aci}]}\\
        \textbf{Editora:} Elsevier B.V.\\
        \textbf{ISSN:} 2210-8327\\
        \textbf{URL:} \url{http://www.journals.elsevier.com/applied-computing-and-informatics}

\item   \textbf{Peri\'{o}dico:} Journal of Software Engineering Research and Development Research \textbf{[Doc. \ref{reviewer:2015-jserd}]}\\
        \textbf{Editora:} Springer
        \textbf{ISSN:} 2195-1721\\
        \textbf{URL:} \url{http://www.jserd.com/}

\item \textbf{Evento:} \\
      \textbf{Ano:}  \\
      \textbf{Organiza\c{c}\~{a}o:} 

\item \textbf{Evento:} Membro do Comitê Técnico de Programa do Simpósio Brasileiro de Qualidade de Software (SBQS'2015)- Trabalhos Técnicos \textbf{[Doc. \ref{reviewer:2015-sbqs}]}\\
      \textbf{Ano:} 2015 \\
      \textbf{Organiza\c{c}\~{a}o:} Sociedade Brasileira de Computa\c{c}\~{a}o

\item \textbf{Evento:} Program Committee of The Second Latin-American School on Software Engineering (ELA-ES 2015) \textbf{[Doc. \ref{reviewer:2015-elaes}]}\\
      \textbf{Ano:} 2015 \\
      \textbf{Organiza\c{c}\~{a}o:} Sociedade Brasileira de Computação

\item \textbf{Evento:} Program Committee Member of the First International Workshop on Fully Enterprise 2.0 (FE2.0) \textbf{[Doc. \ref{reviewer:2015-fe20}]}\\
      \textbf{Ano:} 2015 \\
      \textbf{Organiza\c{c}\~{a}o:} The International Federation for Information Processing (IFIP) WG8.1

\item \textbf{Evento:} Program Committee Member of the 9th Brazilian Symposium on Software Components, Architectures and Reuse (SBCARS 2015) \textbf{[Doc. \ref{reviewer:2015-sbcars}]}\\
      \textbf{Ano:} 2015 \\
      \textbf{Organiza\c{c}\~{a}o:} Sociedade Brasileira de Computa\c{c}\~{a}o

\item \textbf{Evento:} Conference Planning Committee member of the 45th IEEE Frontiers in Education (FIE'2015) \textbf{[Doc. \ref{reviewer:2015-fie}]}\\
      \textbf{Ano:} 2015 \\
      \textbf{Organiza\c{c}\~{a}o:} IEEE Computer Society \& IEEE Education Society

\item \textbf{Evento:} Membro do Comitê de Programa Técnico do 42º Seminário Integrado de Software e Hardware (SEMISH) do Congresso da Sociedade Brasileira de Computação (CSBC) \textbf{[Doc. \ref{reviewer:2015-csbc-semish}]}\\
      \textbf{Ano:} 2015 \\
      \textbf{Organiza\c{c}\~{a}o:} Sociedade Brasileira de Computa\c{c}\~{a}o

\item \textbf{Evento:} Technical Reviewer of the Systems Science \& Engineering Special Session of the IEEE International Conference on Systems, Man, and Cybernetics (SMC'2015) \textbf{[Doc. \ref{reviewer:2015-smc}]}\\
      \textbf{Ano:} 2015 \\
      \textbf{Organiza\c{c}\~{a}o:} IEEE Systems, Man, and Cybernetics Society

\item \textbf{Evento:} Membro do Comitê Técnico de Programa do Simpósio Brasileiro de Engenharia de Software (SBES'2015), Trilha Ideias Inovadoras \textbf{[Doc. \ref{reviewer:2015-sbes-inovadoras}]}\\
      \textbf{Ano:} 2015 \\
      \textbf{Organiza\c{c}\~{a}o:} Sociedade Brasileira de Computa\c{c}\~{a}o

\item \textbf{Evento:} Technical Program Committee for Brazilian Conference on Software: Theory and Practice (CBSoft'2015) - Industry Track \textbf{[Doc. \ref{reviewer:2015-cbsoft-industry}]}\\
      \textbf{Ano:} 2015 \\
      \textbf{Organiza\c{c}\~{a}o:} Sociedade Brasileira de Computa\c{c}\~{a}o

\item \textbf{Evento:} Program Committee Member for 6th Brazilian Workshop on Agile Methods (WBMA 2015) \textbf{[Doc. \ref{reviewer:2015-wbma}]}\\
      \textbf{Ano:} 2015 \\
      \textbf{Organiza\c{c}\~{a}o:} Agile Alliance Brazil

\item \textbf{Evento:} Program Committee Member of the XXI Brazilian Symposium on Multimedia (Webmedia'2015), Full and Short Papers Tracks \textbf{[Doc. \ref{reviewer:2015-webmedia}]}\\
      \textbf{Ano:} 2015 \\
      \textbf{Organiza\c{c}\~{a}o:} Sociedade Brasileira de Computa\c{c}\~{a}o

\item \textbf{Evento:} Program Committee Member of the Tenth International Conference on Software Engineering Advances (ICSEA 2015) \textbf{[Doc. \ref{reviewer:2015-icsea}]}\\
      \textbf{Ano:} 2015 \\
      \textbf{Organiza\c{c}\~{a}o:} International Academy, Research, and Industry Association (IARIA)

\item \textbf{Evento:} Technical Programme Committee Member of The 2nd International Workshop on Smart City Clouds: Technologies, Systems and Applications (SCCTSA 2015) \textbf{[Doc. \ref{reviewer:2015-scctsa}]} \\
      \textbf{Ano:} 2015 \\
      \textbf{Organiza\c{c}\~{a}o:} IEEE/ACM

\item \textbf{Evento:} Membro do Comitê de Programa do XVII Encontro de Computação e Informática do Tocantins (ENCOINFO'2015) \textbf{[Doc. \ref{reviewer:2015-encoinfo}]}\\
      \textbf{Ano:} 2015 \\
      \textbf{Organiza\c{c}\~{a}o:} Centro Universitário Luterano de Palmas (CEULP/ULBRA)

\item \textbf{Evento:} Program Committee Member of the XVII Latin-Iberian-American Conference on Operations Research (CLAIO) \textbf{[Doc. \ref{reviewer:2014-claio}]}\\
      \textbf{Ano:} 2014 \\
      \textbf{Organiza\c{c}\~{a}o:} The Mexican Society of Operations Research (SMIO)

\item \textbf{Evento:} Membro de Comitê de Programa do XLI Seminário Integrado de Software e Hardware (SEMISH), Congresso da Sociedade Brasileira de Computação (CSBC) \textbf{[Doc. \ref{reviewer:2014-csbc-semish}]}\\
      \textbf{Ano:} 2014 \\
      \textbf{Organiza\c{c}\~{a}o:} Sociedade Brasileira de Computa\c{c}\~{a}o

\item \textbf{Evento:} Program Committee member of XX Brazilian Symposium on Multimedia and the Web (WebMedia'2014), Full and Short Papers Track \textbf{[Doc. \ref{reviewer:2014-webmedia}]}\\
      \textbf{Ano:} 2014 \\
      \textbf{Organiza\c{c}\~{a}o:} Sociedade Brasileira de Computa\c{c}\~{a}o

\item \textbf{Evento:} Technical Program Committee member of Workshop on Undergraduate Research Work (WTIC) / XX Brazilian Symposium on Multimedia and the Web (WebMedia'2014) \textbf{[Doc. \ref{reviewer:2014-webmedia-wtic}]}\\
      \textbf{Ano:} 2014 \\
      \textbf{Organiza\c{c}\~{a}o:} Sociedade Brasileira de Computa\c{c}\~{a}o

\item \textbf{Evento:} Member of The Ninth International Conference on Software Engineering Advances (ICSEA) Technical Program Committee and Reviewer \textbf{[Doc. \ref{reviewer:2014-icsea}]}\\
      \textbf{Ano:} 2014 \\
      \textbf{Organiza\c{c}\~{a}o:} International Academy, Research, and Industry Association (IARIA)

\item \textbf{Evento:} Membro do Comitê Técnico da Trilha da Indústria do Congresso Brasileiro de Software: Teoria e Prática (CBSoft) \textbf{[Doc. \ref{reviewer:2014-industria-cbsoft}]}\\
      \textbf{Ano:} 2014 \\
      \textbf{Organiza\c{c}\~{a}o:} Sociedade Brasileira de Computa\c{c}\~{a}o

\item \textbf{Evento:} Membro de Comitê de Programa do IV Workshop de Teses e Dissertações (WTDSoft'2014) do Congresso Brasileiro de Software: Teoria e Prática (CBSoft'2014) \textbf{[Doc. \ref{reviewer:2014-wtdsoft-cbsoft}]}\\
      \textbf{Ano:} 2014 \\
      \textbf{Organiza\c{c}\~{a}o:} Sociedade Brasileira de Computa\c{c}\~{a}o

\item \textbf{Evento:} Program Committee Member of the Eighth Brazilian Symposium on Software Components, Architectures and Reuse (SBCARS 2014) \textbf{[Doc. \ref{reviewer:2014-sbcars}]} \\
      \textbf{Ano:} 2014 \\
      \textbf{Organiza\c{c}\~{a}o:} Sociedade Brasileira de Computa\c{c}\~{a}o

\item \textbf{Evento:} Membro do Comitê Revisor da Trilha de Trabalhos Técnicos do Simpósio Brasileiro de Qualidade de Software (SBQS) \textbf{[Doc. \ref{reviewer:2014-sbqs}]}\\
      \textbf{Ano:} 2014 \\
      \textbf{Organiza\c{c}\~{a}o:} Sociedade Brasileira de Computa\c{c}\~{a}o

\end{enumerate}


%------------------------------------------------------------------------------
% TODO
\subsubsection{Coordena\c{c}\~{a}o e/ou Participa\c{c}\~{a}o em Projetos Aprovados por \'{O}rg\~{a}os de Fomento}
\vspace{0.3cm}

\begin{enumerate}
\renewcommand{\labelenumi}{{\large\bfseries\arabic{enumi}.}}

% \item       \begin{description}
%             \item \textbf{T\'{\i}tulo do projeto:} \\
%                 Codifica\c{c}\~{a}o Distribu\'{\i}da usando C\'{o}digos Turbo Produto e LDPC Aplicada a Redes de Sensores sem Fio \Doc
%             \item \textbf{Fun\c{c}\~{a}o no projeto:} \\
%                 Coordenador e pesquisador principal
%             \item \textbf{Pessoal envolvido:} \\
%                 1 pesquisador doutor, 3 estudantes de mestrado e 1 estudante de gradua\c{c}\~{a}o
%             \item \textbf{Financiador:} \\
%                 FACEPE em parceria com o CNPq
%             \item \textbf{Per\'{\i}odo (in\'{\i}cio-fim):} \\
%                 2010-2013
%             \end{description}

\item \textbf{T\'{\i}tulo do projeto:} BIGStore - Evolução da plataforma Ustore para Armazenamento, Manipulação e Experimentação de Grandes Volumes de Dados \textbf{[Doc. \ref{project:2015-facepe-pepe}]}\\
      \textbf{Fun\c{c}\~{a}o no projeto:} Integrante\\
      \textbf{N\'{u}mero do processo:} SIN-0199-1.03/15\\
      \textbf{Financiador/Edital:} Fundação de Amparo à Ciência e Tecnologia do Estado de Pernambuco, Edital FACEPE 23/2014 - PESQUISADOR NA EMPRESA DE PERNAMBUCO (PEPE)\\
      \textbf{Per\'{\i}odo (in\'{\i}cio-fim):} 2015 - 2018\\
      \textbf{Situação:} Em andamento.\\
      \textbf{Natureza:} Desenvolvimento tecnológico.

\item \textbf{T\'{\i}tulo do projeto:} Análise das Tendências Tecnológicas para Computação em Nuvem e Redes de Longa Distância \textbf{[Doc. \ref{project:2014-citex-ufrpe}]}\\
      \textbf{Fun\c{c}\~{a}o no projeto:} Integrante\\
      \textbf{N\'{u}mero do processo:} EME 14-119-00\\
      \textbf{Financiador/Edital:} Termo de Execução Descentralizada tendo como partícipes o Departamento de Ciência e Tecnologia – DCT, por intermédio do Centro Integrado de Telemática do Exército (Repassadora) e a Universidade Federal Rural de Pernambuco (Recebedora)\\
      \textbf{Per\'{\i}odo (in\'{\i}cio-fim):} 2014 - 2016\\
      \textbf{Situação:} Em andamento \\
      \textbf{Natureza:} Desenvolvimento tecnológico.

\item \textbf{T\'{\i}tulo do projeto:} Smart City Data Mediation \textbf{[Doc. \ref{project:2013-cnpq-485420-2013-9}]}\\
      \textbf{Fun\c{c}\~{a}o no projeto:} Integrante\\
      \textbf{N\'{u}mero do processo:} 485420/2013-9\\
      \textbf{Financiador/Edital:} Conselho Nacional de Desenvolvimento Científico e Tecnológico, Edital MCTI/CNPq Nº 14/2013\\
      \textbf{Per\'{\i}odo (in\'{\i}cio-fim):} 2013 - 2016\\
      \textbf{Situação:} Em andamento \\
      \textbf{Natureza:} Pesquisa.

% \item \textbf{T\'{\i}tulo do projeto:} BaSiC - Barramento de Serviços para Cidades Inteligentes \textbf{[Doc. \ref{project:2013-isi}]}\\
%       \textbf{Fun\c{c}\~{a}o no projeto:} Integrante\\
%       \textbf{N\'{u}mero do processo:} \\
%       \textbf{Financiador/Edital:} Instituto SENAI de Inovação para Tecnologias da Informação (ISI), Conselho Nacional de Desenvolvimento Científico e Tecnológico (CNPq), Fundação de Amparo à Ciência e Tecnologia do Estado de Pernambuco (FACEPE)\\
%       \textbf{Per\'{\i}odo (in\'{\i}cio-fim):} \\
%       \textbf{Situação:} Em andamento \\
%       \textbf{Natureza:} Pesquisa.

% TODO
% \item \textbf{T\'{\i}tulo do projeto:} Uma Abordagem para Indexação e Buscas Full-Text Baseadas em Conteúdo para Sistemas de Armazenamento em Nuvem \colorbox{yellow}{\textbf{[Doc. \ref{project:2013-ustore-ustorage}]}}\\
%       \textbf{Fun\c{c}\~{a}o no projeto:} Coordenador de pesquisa\\
%       \textbf{N\'{u}mero do processo:} \\
%       \textbf{Financiador/Edital:} \\
%       \textbf{Per\'{\i}odo (in\'{\i}cio-fim):} 2013 - \\
%       \textbf{Situação:} Em andamento \\
%       \textbf{Natureza:} Desenvolvimento tecnológico.

\item \textbf{T\'{\i}tulo do projeto:} Uma Arquitetura de Referência para Softwares Baseados em Máquinas Sociais \textbf{[Doc. \ref{project:2013-facepe-ibpg-0719-7-03-12}]}\\
      \textbf{Fun\c{c}\~{a}o no projeto:} Coordenador \\
      \textbf{N\'{u}mero do processo:} IBPG-0719-1.03/12\\
      \textbf{Financiador/Edital:} Fundação de Amparo à Ciência e Tecnologia do Estado de Pernambuco (FACEPE), Edital FACEPE 17/2012\\
      \textbf{Per\'{\i}odo (in\'{\i}cio-fim):} 2013 - 2015 \\
      \textbf{Situação:} Concluído \\
      \textbf{Natureza:} Pesquisa.

\item \textbf{T\'{\i}tulo do projeto:} Uma Plataforma de Cidades Inteligentes baseada na Internet das Coisas \textbf{[Doc. \ref{project:2012-fapesp}]}\\
      \textbf{Fun\c{c}\~{a}o no projeto:} Integrante\\
      \textbf{N\'{u}mero do processo:} Processo CNPq 2012/10157-1, Projeto Regular FAPESP\\
      \textbf{Financiador/Edital:} Fundação de Amparo à Pesquisa do Estado de São Paulo (FAPESP)\\
      \textbf{Per\'{\i}odo (in\'{\i}cio-fim):} 2012 - 2014\\
      \textbf{Situação:} Concluído \\
      \textbf{Natureza:} Pesquisa.

\item \textbf{T\'{\i}tulo do projeto:} Um Ambiente como Serviço para Gerenciamento de Implantação Ágil de Aplicações na Nuvem \textbf{[Doc. \ref{project:2012-facepe-phd}]}\\
      \textbf{Fun\c{c}\~{a}o no projeto:} Coordenador\\
      \textbf{N\'{u}mero do processo:} IBPG-0499-1.03/11\\
      \textbf{Financiador/Edital:} Fundação de Amparo à Ciência e Tecnologia do Estado de Pernambuco (FACEPE), Edital FACEPE 11/2011\\
      \textbf{Per\'{\i}odo (in\'{\i}cio-fim):} 2012 - 2016\\
      \textbf{Situação:} Em andamento \\
      \textbf{Natureza:} Pesquisa.

% TODO
\item \textbf{T\'{\i}tulo do projeto:} Nuvem Educacional Media Center: Uma Proposta de Solução para Distribuição de Conteúdos Educacionais \textbf{[Doc. \ref{project:2012-ustore-midiacenter}]}\\
      \textbf{Fun\c{c}\~{a}o no projeto:} Coordenador Adjunto\\
      \textbf{N\'{u}mero do processo:} \\
      \textbf{Financiador/Edital:} Ministério da Educação (MEC), Programa PROINFO\\
      \textbf{Per\'{\i}odo (in\'{\i}cio-fim):} 2013 - 2014\\
      \textbf{Situação:} Concluído \\
      \textbf{Natureza:} Desenvolvimento tecnológico.

\end{enumerate}

%------------------------------------------------------------------------------

\subsubsection{Consultoria \`{a}s Institui\c{c}\~{o}es de Fomento \`{a} Pesquisa, Ensino e Extens\~{a}o}
\vspace{0.3cm}

\begin{enumerate}
\renewcommand{\labelenumi}{{\large\bfseries\arabic{enumi}.}}

\item   \textbf{Fun\c{c}\~{a}o:} Avaliador de Projetos de Pesquisa (modalidade Subven\c{c}\~{a}o Econ\^{o}mica - PAPPE Integra\c{c}\~{a}o) - Edital 10.2/2012 \textbf{[Doc. \ref{consulting:2015-facepe-pepe}]}\\
        \textbf{Institui\c{c}\~{a}o:} Fundação de Amparo à Ciência e Tecnologia do Estado de Pernambuco (FACEPE).


\end{enumerate}

%------------------------------------------------------------------------------

\subsubsection{Pr\^{e}mios Recebidos pela Produ\c{c}\~{a}o Cient\'{\i}fica e T\'{e}cnica}
\vspace{0.3cm}

Nada a declarar neste subgrupo.

%\begin{enumerate}
%\renewcommand{\labelenumi}{{\large\bfseries\arabic{enumi}.}}

%\item Melhor artigo \colorbox{yellow}{\textbf{[Doc. \ref{award:2014}]}}

%\end{enumerate}

%%%%%%%%%%%%%%%%%%%%%%%%%%%%%%%%%%%%%%%%%%%%%%%%%%%%%%%%%%%%%%%%%%%%%%%%%%%%%%%
% Subgrupo 2.2 - Produção Científica
%%%%%%%%%%%%%%%%%%%%%%%%%%%%%%%%%%%%%%%%%%%%%%%%%%%%%%%%%%%%%%%%%%%%%%%%%%%%%%%

\subsection{Produção Científica}
\vspace{0.3cm}

%------------------------------------------------------------------------------

\subsubsection{Trabalhos Publicados em Peri\'{o}dicos Especializados do Pa\'{\i}s ou do Exterior}
\vspace{0.3cm}

\begin{enumerate}
\renewcommand{\labelenumi}{{\large\bfseries\arabic{enumi}.}}

% TODO
\item MORAIS, I. S. ; FERREIRA, S. E. ; ANDRADE, R. C. ; FURTADO, F. ; GARCIA, Vinicius Cardoso ; CARVALHO, F. . ``GUIDELINE PARA PRIORIZAÇÃO DE SOLICITAÇÕES CORRETIVAS URGENTES BASEADO EM METODOLOGIAS ÁGEIS PARA O CONTEXTO DE FÁBRICAS DE SOFTWARE ORIENTADAS A PRODUTO''. In: REVISTA ELETRÔNICA ENG TECH SCIENCE, v. 3, p. 22-32, 2015 \colorbox{yellow}{(Qualis XX)} \textbf{[Doc. \ref{journal:2015-revista-engtech}]}

% TODO
\item BRITO, Kellyton dos Santos; COSTA, Marcos Antônio da Silva; GARCIA, Vinicius Cardoso and MEIRA, Silvio Romero de Lemos. ``Is Brazilian Open Government Data Actually Open Data?: An Analysis of the Current Scenario''. In: International Journal of E-Planning Research (IJEPR) 4 (2015): 2, doi:10.4018/ijepr.2015040104 \colorbox{yellow}{(Qualis XX)} \textbf{[Doc. \ref{journal:2015-ijepr}]}

% TODO
\item AFONSO, Ricardo Alexandre ; BRITO, Kellyton dos Santos ; DO NASCIMENTO, CLÓVIS HOLANDA ; COSTA, L.  C. ; ÁLVARO, ALEXANDRE ; Garcia, Vinicius Cardoso . ``(Br-SCMM) Brazilian Smart City Maturity Model: A Perspective from the Health Domain''. Studies in Health Technology and Informatics, v. 216 (MEDINFO 2015: eHealth-enabled Health), p. 983, 2015, doi: 10.3233/978-1-61499-564-7-983 \colorbox{yellow}{(Qualis XX)} \textbf{[Doc. \ref{journal:2015-medinfo}]}

\end{enumerate}

%%%%%%%%%%%%%%%%%%%%%%%%%%%%%%%%%%%%%%%%%%%%%%%%%%%%%%%%%%%%%%%%%%%%%%%%%%%%%%%
% Grupo 3: Atividades de Extens\~{a}o
%%%%%%%%%%%%%%%%%%%%%%%%%%%%%%%%%%%%%%%%%%%%%%%%%%%%%%%%%%%%%%%%%%%%%%%%%%%%%%%
\newpage
\section{Atividades de Extens\~{a}o}

%%%%%%%%%%%%%%%%%%%%%%%%%%%%%%%%%%%%%%%%%%%%%%%%%%%%%%%%%%%%%%%%%%%%%%%%%%%%%%%
% Subgrupo 3.1 - Coordenação e Orientação
%%%%%%%%%%%%%%%%%%%%%%%%%%%%%%%%%%%%%%%%%%%%%%%%%%%%%%%%%%%%%%%%%%%%%%%%%%%%%%%
\subsection{Subgrupo 3.1 - Coordenação e Orientação}
\vspace{0.3cm}

Nada a declarar neste subgrupo.

%%%%%%%%%%%%%%%%%%%%%%%%%%%%%%%%%%%%%%%%%%%%%%%%%%%%%%%%%%%%%%%%%%%%%%%%%%%%%%%
% Subgrupo 3.2 - Coordenação de Eventos e Conferencista
%%%%%%%%%%%%%%%%%%%%%%%%%%%%%%%%%%%%%%%%%%%%%%%%%%%%%%%%%%%%%%%%%%%%%%%%%%%%%%%
\subsection{Coordenação de Eventos e Conferencista}
\vspace{0.3cm}

%------------------------------------------------------------------------------

\subsubsection{Comiss\~{a}o Organizadora de Eventos Internacional, Nacional, Regional ou Local}
\vspace{0.3cm}

\begin{enumerate}
\renewcommand{\labelenumi}{{\large\bfseries\arabic{enumi}.}}

    \item Membro do Comitê Técnico de Programa do Simpósio Brasileiro de Qualidade de Software (SBQS'2015)- Trabalhos Técnicos \textbf{[Doc. \ref{reviewer:2015-sbqs}]}

    \item Program Committee of The Second Latin-American School on Software Engineering (ELA-ES 2015) \textbf{[Doc. \ref{reviewer:2015-elaes}]}

    \item Program Committee Member of the First International Workshop on Fully Enterprise 2.0 (FE2.0 2015) \textbf{[Doc. \ref{reviewer:2015-fe20}]}

    \item Program Committee Member of the 9th Brazilian Symposium on Software Components, Architectures and Reuse (SBCARS 2015) \textbf{[Doc. \ref{reviewer:2015-sbcars}]}

    \item Conference Planning Committee member of the 45th IEEE Frontiers in Education (FIE'2015) \textbf{[Doc. \ref{reviewer:2015-fie}]}

    \item Membro do Comitê de Programa Técnico do 42º Seminário Integrado de Software e Hardware (SEMISH'2015) do Congresso da Sociedade Brasileira de Computação (CSBC'2015) \textbf{[Doc. \ref{reviewer:2015-csbc-semish}]}

    \item Technical Reviewer of the Systems Science \& Engineering Special Session of the IEEE International Conference on Systems, Man, and Cybernetics (SMC'2015) \textbf{[Doc. \ref{reviewer:2015-smc}]}

    \item Membro do Comitê Técnico de Programa do Simpósio Brasileiro de Engenharia de Software (SBES'2015), Trilha Ideias Inovadoras \textbf{[Doc. \ref{reviewer:2015-sbes-inovadoras}]}

    \item Technical Program Committee for Brazilian Conference on Software: Theory and Practice (CBSoft'2015) - Industry Track \textbf{[Doc. \ref{reviewer:2015-cbsoft-industry}]}

    \item Program Committee Member for 6th Brazilian Workshop on Agile Methods (WBMA 2015) \textbf{[Doc. \ref{reviewer:2015-wbma}]}

    \item Program Committee Member of the XXI Brazilian Symposium on Multimedia (Webmedia'2015), Full and Short Papers Tracks \textbf{[Doc. \ref{reviewer:2015-webmedia}]}

    \item Program Committee Member of the Tenth International Conference on Software Engineering Advances (ICSEA'2015) \textbf{[Doc. \ref{reviewer:2015-icsea}]}

    \item Technical Programme Committee Member of The 2nd International Workshop on Smart City Clouds: Technologies, Systems and Applications (SCCTSA'2015) \textbf{[Doc. \ref{reviewer:2015-scctsa}]}

    \item Membro do Comitê de Programa do XVII Encontro de Computação e Informática do Tocantins (ENCOINFO'2015) \textbf{[Doc. \ref{reviewer:2015-encoinfo}]}

    \item Program Committee Member of the XVII Latin-Iberian-American Conference on Operations Research (CLAIO'2014) \textbf{[Doc. \ref{reviewer:2014-claio}]}

    \item Membro de Comitê de Programa do XLI Seminário Integrado de Software e Hardware (SEMISH'2014), Congresso da Sociedade Brasileira de Computação (CSBC'2014) \textbf{[Doc. \ref{reviewer:2014-csbc-semish}]}

    \item Program Committee member of XX Brazilian Symposium on Multimedia and the Web (WebMedia'2014), Full and Short Papers Track \textbf{[Doc. \ref{reviewer:2014-webmedia}]}

    \item Technical Program Committee member of Workshop on Undergraduate Research Work (WTIC) / XX Brazilian Symposium on Multimedia and the Web (WebMedia'2014) \textbf{[Doc. \ref{reviewer:2014-webmedia-wtic}]}

    \item Member of The Ninth International Conference on Software Engineering Advances (ICSEA'2014) Technical Program Committee and Reviewer \textbf{[Doc. \ref{reviewer:2014-icsea}]}

    \item Membro do Comitê Técnico da Trilha da Indústria do Congresso Brasileiro de Software: Teoria e Prática (CBSoft'2014) \textbf{[Doc. \ref{reviewer:2014-industria-cbsoft}]}

    \item Membro de Comitê de Programa do IV Workshop de Teses e Dissertações (WTDSoft'2014) do Congresso Brasileiro de Software: Teoria e Prática (CBSoft'2014) \textbf{[Doc. \ref{reviewer:2014-wtdsoft-cbsoft}]}

    \item Program Committee Member of the Eighth Brazilian Symposium on Software Components, Architectures and Reuse (SBCARS'2014) \textbf{[Doc. \ref{reviewer:2014-sbcars}]}

    \item Membro do Comitê Revisor da Trilha de Trabalhos Técnicos do Simpósio Brasileiro de Qualidade de Software (SBQS'2014) \textbf{[Doc. \ref{reviewer:2014-sbqs}]}

\end{enumerate}

%%%%%%%%%%%%%%%%%%%%%%%%%%%%%%%%%%%%%%%%%%%%%%%%%%%%%%%%%%%%%%%%%%%%%%%%%%%%%%%
% Grupo 4: Atividades de Forma\c{c}\~{a}o e Capacita\c{c}\~{a}o Acad\^{e}mica
%%%%%%%%%%%%%%%%%%%%%%%%%%%%%%%%%%%%%%%%%%%%%%%%%%%%%%%%%%%%%%%%%%%%%%%%%%%%%%%
\newpage
\section{Atividades de Forma\c{c}\~{a}o e Capacita\c{c}\~{a}o Acad\^{e}mica}
\vspace{0.3cm}

%------------------------------------------------------------------------------

\subsection{Atualiza\c{c}\~{a}o e Cursos de Capacita\c{c}\~{a}o ou Extens\~{a}o na \'{A}rea de Conhecimento ou Afins com no M\'{\i}nimo 40h}
\vspace{0.3cm}

Nada a declarar neste subgrupo.

% \begin{enumerate}
% \renewcommand{\labelenumi}{{\large\bfseries\arabic{enumi}.}}

% \item   \textbf{Curso:}  Machine Learning (Coursera) \Doc \\
%         \textbf{Modalidade:} Ensino \`{a} dist\^{a}ncia (EAD) \\
%         \textbf{Carga Hor\'{a}ria:} 50 h  \\
%         \textbf{Instrutor:} Prof. Andrew Ng \\
%         \textbf{Institui\c{c}\~{a}o:} Stanford University

% \end{enumerate}


%%%%%%%%%%%%%%%%%%%%%%%%%%%%%%%%%%%%%%%%%%%%%%%%%%%%%%%%%%%%%%%%%%%%%%%%%%%%%%%
% Grupo 5: Atividades Administrativas
%%%%%%%%%%%%%%%%%%%%%%%%%%%%%%%%%%%%%%%%%%%%%%%%%%%%%%%%%%%%%%%%%%%%%%%%%%%%%%%
\newpage
\section{Atividades Administrativas}
\vspace{0.3cm}

%------------------------------------------------------------------------------

\subsection{Membro de Comiss\~{a}o Tempor\'{a}ria}
\vspace{0.3cm}

\begin{enumerate}
\renewcommand{\labelenumi}{{\large\bfseries\arabic{enumi}.}}


\item   \textbf{Fun\c{c}\~{a}o:} \\
        \textbf{Comiss\~{a}o:} \\
        \textbf{Per\'{\i}odo:} 

\end{enumerate}

%------------------------------------------------------------------------------

\subsection{Coordenador de Curso Pós-Graduação \textbf{strictu sensu} }
\vspace{0.3cm}

\begin{enumerate}
\renewcommand{\labelenumi}{{\large\bfseries\arabic{enumi}.}}

\item   \textbf{Fun\c{c}\~{a}o:} Tutor da turma do Mestrado Profissional em Ciência da Computação com ênfase em Sistemas de Informação \\
        \textbf{Per\'{\i}odo:} 01 de Dezembro de 2014 a 01 de Dezembro de 2017.

\end{enumerate}

%------------------------------------------------------------------------------

\subsection{Membro de Núcleo Docente Estruturante}
\vspace{0.3cm}

\begin{enumerate}
\renewcommand{\labelenumi}{{\large\bfseries\arabic{enumi}.}}

\item   \textbf{Fun\c{c}\~{a}o:} Membro do Núcleo Docente Estruturante do Curso de Graduação em Sistemas de Informação \\
        \textbf{Per\'{\i}odo:} 21 de Julho de 2015 a 20 de Julho de 2016.

\item   \textbf{Fun\c{c}\~{a}o:} Membro do Núcleo Docente Estruturante do Curso de Graduação em Sistemas de Informação \\
        \textbf{Per\'{\i}odo:} 21 de Julho de 2014 a 20 de Julho de 2015.

\end{enumerate}

%------------------------------------------------------------------------------

\subsection{Membro de Colegiados de Curso de Gradua\c{c}\~{a}o e P\'{o}s-Gradua\c{c}\~{a}o}
\vspace{0.3cm}

\begin{enumerate}
\renewcommand{\labelenumi}{{\large\bfseries\arabic{enumi}.}}

\item   \textbf{Fun\c{c}\~{a}o:} Membro do Colegiado da P\'{o}s-Gradua\c{c}\~{a}o \textbf{(em curso)}  \Doc \\
        \textbf{Per\'{\i}odo:} Desde Setembro de 2010.

\item   \textbf{Fun\c{c}\~{a}o:} Membro do Colegiado do Curso de Graduação em Ciência da Computação  \\
        \textbf{Per\'{\i}odo:} 15 de Outubro de 2015 a 14 de Outubro de 2016.

\item   \textbf{Fun\c{c}\~{a}o:} Membro do Colegiado do Curso de Graduação em Sistemas de Informação  \\
        \textbf{Per\'{\i}odo:} 21 de Julho de 2015 a 20 de Julho de 2016.

\item   \textbf{Fun\c{c}\~{a}o:} Membro do Colegiado do Curso de Graduação em Sistemas de Informação  \\
        \textbf{Per\'{\i}odo:} 21 de Julho de 2014 a 20 de Julho de 2015.

\end{enumerate}

%%%%%%%%%%%%%%%%%%%%%%%%%%%%%%%%%%%%%%%%%%%%%%%%%%%%%%%%%%%%%%%%%%%%%%%%%%%%%%%
% LISTA DE ANEXOS
%%%%%%%%%%%%%%%%%%%%%%%%%%%%%%%%%%%%%%%%%%%%%%%%%%%%%%%%%%%%%%%%%%%%%%%%%%%%%%%

\newpage
\section{Lista de Anexos}

A Tabela \ref{Tab:ListaAnexos} cont\'{e}m a numera\c{c}\~{a}o e uma pequena descri\c{c}\~{a}o dos documentos comprobat\'{o}rios em anexo. Quando for o caso, h\'{a} uma indica\c{c}\~{a}o entre par\^{e}nteses, ao final da descri\c{c}\~{a}o, do setor respons\'{a}vel pela emiss\~{a}o do documento.


\begin{table}[h]
\small
\caption{\texttt{Rela\c{c}\~{a}o numerada dos Anexos comprobat\'{o}rios}.}
\begin{tabular}{cl}
\toprule
\large{\textbf{\texttt{Doc}}} & \multicolumn{1}{c}{\large{\textbf{\texttt{Descri\c{c}\~{a}o}}}} \\
\otoprule
  1 & Declara\c{c}\~{a}o de orienta\c{c}\~{a}o de Mestrado de Maxwell Silva (PPGES/UPE). \\
  %\cmidrule{1-2}
  2 & Declara\c{c}\~{a}o de orienta\c{c}\~{a}o de Mestrado de Bruna Melo (PPGES/UPE). \\
  %\cmidrule{1-2}
  3 & Declara\c{c}\~{a}o de orienta\c{c}\~{a}o de TCC de Diocleciano Neto e Lucas Paes (Gradua\c{c}\~{a}o EC CIn). \\
  %\cmidrule{1-2}
  4 & Termo de orienta\c{c}\~{a}o de Monitoria de Lucas Harada - 2013.2 (PROACAD/UFPE). \\
  %\cmidrule{1-2}
  5 & Declara\c{c}\~{a}o de orienta\c{c}\~{a}o de Monitoria de Lucas Harada - 2013.1 (Sec Grad CIn). \\
  %\cmidrule{1-2}
  6 & Declara\c{c}\~{a}o de orienta\c{c}\~{a}o de IC de Djeefther Albuquerque (PROPESQ/UFPE). \\
  %\cmidrule{1-2}
  7 & Declara\c{c}\~{a}o de participa\c{c}\~{a}o em banca de Doutorado de Isaac Benchimol (PPGEE/UFPE). \\
  %\cmidrule{1-2}
  8 & Declara\c{c}\~{a}o de participa\c{c}\~{a}o em banca de Doutorado de Walter Guimar\~{a}es (PPGEE/UFPE). \\
  \cmidrule{1-2}
  9 & Declara\c{c}\~{a}o de participa\c{c}\~{a}o em banca de Mestrado de Joyce Teixeira (PGCC/UFPE). \\
    & C\'{o}pia da Portaria PGCC 166/2012 - Composi\c{c}\~{a}o de banca de Mestrado (PGCC/UFPE). \\
  \cmidrule{1-2}
  10 & Declara\c{c}\~{a}o de participa\c{c}\~{a}o em banca de Mestrado de Antonio Assun\c{c}\~{a}o (PPGES/UPE). \\
     & C\'{o}pia da Portaria PPGES 11/2013 - Composi\c{c}\~{a}o de banca de Mestrado (PPGES/UPE). \\
  \cmidrule{1-2}
  11 & Declara\c{c}\~{a}o de participa\c{c}\~{a}o em banca de Mestrado de Maxwell Silva (PPGES/UPE). \\
     & C\'{o}pia da Portaria PPGES 15/2013 - Composi\c{c}\~{a}o de banca de Mestrado (PPGES/UPE). \\
  \cmidrule{1-2}
  12 & Declara\c{c}\~{a}o de participa\c{c}\~{a}o em banca de Mestrado de Bruna Melo (PPGES/UPE). \\
     & C\'{o}pia da Portaria PPGES 16/2013 - Composi\c{c}\~{a}o de banca de Mestrado (PPGES/UPE). \\
  \cmidrule{1-2}
  13 & Declara\c{c}\~{a}o de participa\c{c}\~{a}o em banca de TCC (Gradua\c{c}\~{a}o-EC CIn). \\
  %\cmidrule{1-2}
  14 & Declara\c{c}\~{a}o de particip. em banca de Sele\c{c}\~{a}o de Prof. Tempor\'{a}rio do CIn/UFPE (Dire\c{c}\~{a}o CIn). \\
  %\cmidrule{1-2}
  15 & Relat\'{o}rio de disciplinas lecionadas na gradua\c{c}\~{a}o no per\'{\i}odo de avalia\c{c}\~{a}o (SIGA-UFPE).\\
  %\cmidrule{1-2}
  16 & C\'{o}pia do certificado de participa\c{c}\~{a}o no XXX SBrT. \\
  %\cmidrule{1-2}
  17 & C\'{o}pia do certificado de participa\c{c}\~{a}o no XXXI SBrT. \\
  %\cmidrule{1-2}
  18 & C\'{o}pias do resumo publicado e do e-mail de aceita\c{c}\~{a}o de trabalho no XXXI SBrT. \\
  %\cmidrule{1-2}
  19 & C\'{o}pias do artigo publicado e do e-mail de aceita\c{c}\~{a}o de trabalho no XXX SBrT. \\
  %\cmidrule{1-2}
  20 a 23 & C\'{o}pias dos artigos publicados e dos e-mails de aceita\c{c}\~{a}o de trabalho no CBA 2012 e XXXI SBrT. \\
  %\cmidrule{1-2}
  24 a 31 & C\'{o}pias dos e-mails de convite para revis\~{a}o e finaliza\c{c}\~{a}o do processo (peri\'{o}dicos e eventos). \\
  \cmidrule{1-2}
  32 & C\'{o}pia do Resultado do Edital 10/2010 (pags 1,2 e 6) e Termo de Outorga do projeto (FACEPE). \\
     & C\'{o}pia de correspond\^{e}ncia comprobat\'{o}ria de presta\c{c}\~{a}o de contas do projeto (FACEPE). \\
  \cmidrule{1-2}
  33 & C\'{o}pias de e-mail de convite para avaliador e de cabe\c{c}alho do parecer emitido (FACEPE). \\
  %\cmidrule{1-2}
  34 & C\'{o}pia do certificado de premia\c{c}\~{a}o de melhor artigo de IC do XXXI SBrT. \\
  %\cmidrule{1-2}
  35 & C\'{o}pias do artigo publicado no peri\'{o}dico TEMA e do e-mail de aceita\c{c}\~{a}o. \\
  %\cmidrule{1-2}
  36 & C\'{o}pias do artigo publicado no peri\'{o}dico EURASIP J. Adv. Sig. Proc. e do e-mail de aceita\c{c}\~{a}o. \\
  %\cmidrule{1-2}
  37 & C\'{o}pias dos e-mails comprobat\'{o}rios de convite e resposta ao convite (MIC-CSC2012). \\
  %\cmidrule{1-2}
  38 & C\'{o}pias dos e-mails comprobat\'{o}rios de convite e resposta ao convite (XXXI SBrT). \\
  %\cmidrule{1-2}
  39 & C\'{o}pia da declara\c{c}\~{a}o enviada \`{a} Comiss\~{a}o Organizadora da SBPC 2013 (Dire\c{c}\~{a}o CIn). \\
  %\cmidrule{1-2}
  40 & Online Course Statement of Accomplishment: Machine Learning - Stanford University - Coursera. \\
  %\cmidrule{1-2}
  41 - 42 & Declara\c{c}\~{o}es de participa\c{c}\~{a}o na Comiss\~{a}o de Sele\c{c}\~{a}o do Mestrado CIn - 2012 e 2013. \\
  %\cmidrule{1-2}
  43 & Declara\c{c}\~{a}o comprobat\'{o}ria de cargo ocupado - Subcoordena\c{c}\~{a}o de Editais. \\
  %\cmidrule{1-2}
  44 & Declara\c{c}\~{a}o de participa\c{c}\~{a}o no Colegiado da P\'{o}s-Gradua\c{c}\~{a}o do CIn. \\
  %\cmidrule{1-2}
  45 & C\'{o}pia da Portaria de Designa\c{c}\~{a}o 005/2013 - Colegiado Gradua\c{c}\~{a}o EC CIn (Dire\c{c}\~{a}o CIn). \\
\bottomrule
\end{tabular}
\label{Tab:ListaAnexos}
\end{table}

% Appendix
\clearpage
%\addappheadtotoc
\appendix
%\appendixpage
\newpage
\section{Documentos comprobatórios}
Esta seção contém os documentos comprobatórios referentes às atividades listadas neste memorial.
\addcontentsline{toc}{section}{Documentos comprobatórios}
\renewcommand{\thesubsection}{\arabic{subsection}}
% \renewcommand{\subsection}{
% \titleformat{\subsection}
%   {\Huge\bfseries\center\vspace{.4\textwidth}\thispagestyle{fancy}} % format
%   {}                % label
%   {0pt}             % sep
%   {\huge}           % before-code
% }

%%%%%%%%%%%%%%%%%%%%%%%%%%%%%%%%%%%%%%%%%%%%%%%%%%%%%%%%%%%%%%%%%%%%%%%%%%%%%%%
% Grupo 1 - Atividades de Ensino
%%%%%%%%%%%%%%%%%%%%%%%%%%%%%%%%%%%%%%%%%%%%%%%%%%%%%%%%%%%%%%%%%%%%%%%%%%%%%%%

%%%%%%%%%%%%%%%%%%%%%%%%%%%%%%%%%%%%%%%%%%%%%%%%%%%%%%%%%%%%%%%%%%%%%%%%%%%%%%%
% Subgrupo 1.1 - Orienta\c{c}\~{o}es e Co-Orienta\c{c}\~{o}es
%%%%%%%%%%%%%%%%%%%%%%%%%%%%%%%%%%%%%%%%%%%%%%%%%%%%%%%%%%%%%%%%%%%%%%%%%%%%%%%

\newpage
\subsection{Orientações Conluídas}
\label{app:orientacoes-concluidas}
Esta subseção apresenta o comprovante de orientações de Dissertação de Mestrado e Tese de Doutorado concluídas.
\includepdf[pages=-, scale=1,pagecommand=\thispagestyle{empty}]{\detokenize{GRUPO 1 – Atividades - Ensino/Sub-Grupo 1.1 - Orientações e Co-Orientações/2015_Orientadoes_MSc_PhD_Concluidas}}

\newpage
\subsection{Co-Orientações Conluídas}
\label{app:co-orientacoes-concluidas}
Esta subseção apresenta o comprovante de co-orientações de Dissertação de Mestrado e Tese de Doutorado concluídas.
\includepdf[pages=-, scale=1,pagecommand=\thispagestyle{empty}]{\detokenize{GRUPO 1 – Atividades - Ensino/Sub-Grupo 1.1 - Orientações e Co-Orientações/2015_Co-Orientadoes_MSc_PhD_Concluidas}}

\newpage
\subsection{Orientações em Andamento}
\label{app:orientacoes-andamento}
Esta subseção apresenta o comprovante de orientações de Dissertação de Mestrado e Tese de Doutorado em andamento.
\includepdf[pages=-, scale=1, pagecommand=\thispagestyle{empty}]{\detokenize{GRUPO 1 – Atividades - Ensino/Sub-Grupo 1.1 - Orientações e Co-Orientações/2015_UFPE_MSc_PhD_Orientacao_Andamento}}

\newpage
\subsection{Co-Orientações em Andamento}
\label{app:co-orientacoes-andamento}
Esta subseção apresenta o comprovante de co-orientações de Dissertação de Mestrado e Tese de Doutorado em andamento.
\includepdf[pages=-, scale=1,pagecommand=\thispagestyle{empty}]{\detokenize{GRUPO 1 – Atividades - Ensino/Sub-Grupo 1.1 - Orientações e Co-Orientações/2015_UFPE_MSc_PhD_Co-Orientacao_Andamento}}

\newpage
\subsection{Orientações de Mestrado Profissional Conluídas}
\label{app:orientacoes-mprof-concluidas}
Esta subseção apresenta o comprovante de orientações de Dissertação de Mestrado Profissional em Ciência da Computação em concluidas.
\includepdf[pages=-, scale=1,pagecommand=\thispagestyle{empty}]{\detokenize{GRUPO 1 – Atividades - Ensino/Sub-Grupo 1.1 - Orientações e Co-Orientações/2015_UFPE_MPROF_Orientacao_Andamento}}

\newpage
\subsection{Orientações de Mestrado Profissional em Andamento}
\label{app:orientacoes-mprof-andamento}
Esta subseção apresenta o comprovante de orientações de Dissertação de Mestrado Profissional em Ciência da Computação em andamento.
\includepdf[pages=-, scale=1,pagecommand=\thispagestyle{empty}]{\detokenize{GRUPO 1 – Atividades - Ensino/Sub-Grupo 1.1 - Orientações e Co-Orientações/2015_UFPE_MPROF_Orientacao_Andamento}}

\newpage
\subsection{Orientações de Trabalhos de Conclusão de Curso Concluídas}
\label{app:orientacoes-tcc-concluidas}
Esta subseção apresenta o comprovante de orientações de Trabalhos de Conclusão de Curso concluídas.
\includepdf[pages=-, scale=1,pagecommand=\thispagestyle{empty}]{\detokenize{GRUPO 1 – Atividades - Ensino/Sub-Grupo 1.1 - Orientações e Co-Orientações/2015_Orientadoes_TCC_Concluidas}}

\newpage
\subsection{Orientações de Monitorias}
\label{app:orientacoes-monitoria}
Esta subseção apresenta o comprovante de orientações de Monitoria no período de 2014-2 a 2016-1 impressa diretamente da página do Sistema de Gerenciamento de Monitores Online (GMon)\footnote{URL: \url{https://www.cin.ufpe.br/~gmon}, Último acesso em XX/XX/XXXX}.
\includepdf[pages=-, scale=1,pagecommand=\thispagestyle{empty}]{\detokenize{GRUPO 1 – Atividades - Ensino/Sub-Grupo 1.1 - Orientações e Co-Orientações/2014.2_2016.1_UFPE_Orientacao_Monitorias}}

\newpage
\subsection{Orientações de Trabalhos de Apoio Acad\^{e}mico}
\label{app:orientacoes-bia}
Esta subseção apresenta o comprovante de orientações de Trabalhos de Apoio Acad\^{e}mico.
\includepdf[pages=-, scale=1,pagecommand=\thispagestyle{empty}]{\detokenize{GRUPO 1 – Atividades - Ensino/Sub-Grupo 1.1 - Orientações e Co-Orientações/20150710_Declaracao-Orientador-BIA_Victor-Barros-de-Miranda-Neves}}



%%%%%%%%%%%%%%%%%%%%%%%%%%%%%%%%%%%%%%%%%%%%%%%%%%%%%%%%%%%%%%%%%%%%%%%%%%%%%%%
% Subgrupo 1.2 - Participa\c{c}\~{a}o em Comiss\~{o}es Examinadoras
%%%%%%%%%%%%%%%%%%%%%%%%%%%%%%%%%%%%%%%%%%%%%%%%%%%%%%%%%%%%%%%%%%%%%%%%%%%%%%%

\newpage
\subsection{Parecerista dos cursos de Sistemas de Informação da Avaliação de Cursos Superiores do Guia do Estudante (GE) 2015}
\label{app:2015-guia-estudante}
Esta subseção apresenta o comprovante da atuação como parecerista dos cursos de Sistemas de Informação da Avaliação de Cursos Superiores do Guia do Estudante (GE) 2015, promovido pela Editora Abril.
\includepdf[pages=-, scale=1,pagecommand=\thispagestyle{empty}]{\detokenize{GRUPO 1 – Atividades - Ensino/Sub-Grupo 1.2 - Participação em Comissões Examinadoras/1 - Bancas Examinadoras de Concurso/20150916_Parecerista do Guia do Estudante}}

\newpage
\subsection{Membro Titular Externo da Comissão Examinadora do Concurso Público de Provas e Títulos para Professor da Carreira de Magistério Superior}
\label{app:2014-ufrpe-edital-082014}
Esta subseção apresenta o comprovante da atuação como Membro Titular Externo da Comissão Examinadora do Concurso Público de Provas e Títulos para Professor da Carreira de Magistério Superior, da Classe A, da área ENSISO (Engenharia e Sistemas de Software), referente ao Edital 08/2014.
\includepdf[pages=-, scale=1,pagecommand=\thispagestyle{empty}]{\detokenize{GRUPO 1 – Atividades - Ensino/Sub-Grupo 1.2 - Participação em Comissões Examinadoras/1 - Bancas Examinadoras de Concurso/20150312_UFRPE_Edital_08-2014_Declaracao}}

\newpage
\subsection{Parecerista dos cursos de Sistemas de Informação da Avaliação de Cursos Superiores do Guia do Estudante (GE) 2014}
\label{app:2014-guia-estudante}
Esta subseção apresenta o comprovante da atuação como parecerista dos cursos de Sistemas de Informação da Avaliação de Cursos Superiores do Guia do Estudante (GE) 2014, promovido pela Editora Abril.
\includepdf[pages=-, scale=1,pagecommand=\thispagestyle{empty}]{\detokenize{GRUPO 1 – Atividades - Ensino/Sub-Grupo 1.2 - Participação em Comissões Examinadoras/1 - Bancas Examinadoras de Concurso/20141022_Parecerista do Guia do Estudante}}

\newpage
\subsection{Comissão Avaliadora dos trabalhos na área de Ciências Exatas apresentados na 19ª Jornada de Iniciação Científica da FACEPE}
\label{app:2015-jic-facepe}
Esta subseção apresenta o comprovante da atuação como Membro da Comissão Avaliadora dos trabalhos na área de Ciências Exatas apresentados na 19ª Jornada de Iniciação Científica da Fundação de Amparo a Ciência e Tecnologia do Estado de Pernambuco (FACEPE).
\includepdf[pages=-, scale=1,pagecommand=\thispagestyle{empty}]{\detokenize{GRUPO 1 – Atividades - Ensino/Sub-Grupo 1.2 - Participação em Comissões Examinadoras/2 - Bancas de Congressos de Iniciação Científica ou de Extensão/20150612_19a-JIC-FACEPE_Comissao_Avaliadora}}

\newpage
\subsection{Avaliação do Relatório Final do Programa Institucional de Bolsas de Iniciação Científica – PIBIC/CNPq/UFPE (2013/2014) e XXII Congresso de Iniciação Científica da UFPE (2014)}
\label{app:2014-pibic-ufpe}
Esta subseção apresenta o comprovante da atuação no Processo de avaliação do Relatório Final do Programa Institucional de Bolsas de Iniciação Científica – PIBIC/CNPq/UFPE (2013/2014), bem como do Resumo do XXII Congresso de Iniciação Científica da UFPE (2014) do(s) seguinte(s) projeto(s): 14014723PO
\includepdf[pages=-, scale=1,pagecommand=\thispagestyle{empty}]{\detokenize{GRUPO 1 – Atividades - Ensino/Sub-Grupo 1.2 - Participação em Comissões Examinadoras/2 - Bancas de Congressos de Iniciação Científica ou de Extensão/20140929 - Certificado avaliacao CONIC}}

%------------------------------------------------------------------------------

\newpage
\subsection{Participação em Bancas de Trabalho de Conclus\~{a}o de Curso}
\label{app:bancas-tcc}
Esta subseção apresenta o comprovante da atuação membro nas Bancas Examinadoras de Trabalho de Conclusão de Curso no Centro de Informática/UFPE.
%\includepdf[pages=-, scale=1,pagecommand=\thispagestyle{empty}]{\detokenize{GRUPO 1 – Atividades - Ensino/Sub-Grupo 1.2 - Participação em Comissões Examinadoras/5 - Bancas de Mestrado/}}

%------------------------------------------------------------------------------

\newpage
\subsection{Participação em Banca de Disserta\c{c}\~{a}o de Mestrado de Vinicius Nordi Esperança}
\label{app:2016-msc-vne}
Esta subseção apresenta o comprovante da atuação membro na Banca Examinadora de Defesa de Disserta\c{c}\~{a}o de Mestrado de Vinicius Nordi Esperança no Departamento de Computação/UFSCar.
%\includepdf[pages=-, scale=1,pagecommand=\thispagestyle{empty}]{\detokenize{GRUPO 1 – Atividades - Ensino/Sub-Grupo 1.2 - Participação em Comissões Examinadoras/5 - Bancas de Mestrado/20160307_UFSCar_BANCA_MSc_VINICIUS-NORDI-ESPERANCA_PORTARIA}}
%\includepdf[pages=-, scale=1,pagecommand=\thispagestyle{empty}]{\detokenize{GRUPO 1 – Atividades - Ensino/Sub-Grupo 1.2 - Participação em Comissões Examinadoras/5 - Bancas de Mestrado/20160307_UFSCar_BANCA_MSc_VINICIUS-NORDI-ESPERANCA_DECLARACAO}}

\newpage
\subsection{Participação em Banca de Disserta\c{c}\~{a}o de Mestrado de Herbertt Barros Mangueira Diniz}
\label{app:2016-msc-hbmd}
Esta subseção apresenta o comprovante da atuação membro na Banca Examinadora de Defesa de Disserta\c{c}\~{a}o de Mestrado de Herbertt Barros Mangueira Diniz no Centro de Informática/UFPE.
\includepdf[pages=-, scale=1,pagecommand=\thispagestyle{empty}]{\detokenize{GRUPO 1 – Atividades - Ensino/Sub-Grupo 1.2 - Participação em Comissões Examinadoras/5 - Bancas de Mestrado/20160304_UFPE_BANCA_MSc_HERBERTT-BARROS-MANGUEIRA-DINIZ_PORTARIA}}
\includepdf[pages=-, scale=1,pagecommand=\thispagestyle{empty}]{\detokenize{GRUPO 1 – Atividades - Ensino/Sub-Grupo 1.2 - Participação em Comissões Examinadoras/5 - Bancas de Mestrado/20160304_UFPE_BANCA_MSc_HERBERTT-BARROS-MANGUEIRA-DINIZ_DECLARACAO}}

\newpage
\subsection{Participação em Banca de Disserta\c{c}\~{a}o de Mestrado de Ricardo Marinho de Melo}
\label{app:2016-msc-rmm}
Esta subseção apresenta o comprovante da atuação membro na Banca Examinadora de Defesa de Disserta\c{c}\~{a}o de Mestrado de Ricardo Marinho de Melo no Centro de Informática/UFPE.
\includepdf[pages=-, scale=1,pagecommand=\thispagestyle{empty}]{\detokenize{GRUPO 1 – Atividades - Ensino/Sub-Grupo 1.2 - Participação em Comissões Examinadoras/5 - Bancas de Mestrado/20160303_UFPE_BANCA_MSc_RICARDO-MARINHO-DE-MELO_PORTARIA}}
\includepdf[pages=-, scale=1,pagecommand=\thispagestyle{empty}]{\detokenize{GRUPO 1 – Atividades - Ensino/Sub-Grupo 1.2 - Participação em Comissões Examinadoras/5 - Bancas de Mestrado/20160303_UFPE_BANCA_MSc_RICARDO-MARINHO-DE-MELO_DECLARACAO}}

\newpage
\subsection{Participação em Banca de Disserta\c{c}\~{a}o de Mestrado Profissional de Hilson Barbosa da Silva}
\label{app:2016-mprof-hbs}
Esta subseção apresenta o comprovante da atuação como membro na Banca Examinadora de Defesa de Disserta\c{c}\~{a}o de Mestrado Profissional de Hilson Barbosa da Silva no Centro de Informática/UFPE.
\includepdf[pages=-, scale=1,pagecommand=\thispagestyle{empty}]{\detokenize{GRUPO 1 – Atividades - Ensino/Sub-Grupo 1.2 - Participação em Comissões Examinadoras/5 - Bancas de Mestrado/20160122_UFPE_BANCA_MPROF_HILSON-BARBOSA-DA-SILVA_PORTARIA}}
\includepdf[pages=-, scale=1,pagecommand=\thispagestyle{empty}]{\detokenize{GRUPO 1 – Atividades - Ensino/Sub-Grupo 1.2 - Participação em Comissões Examinadoras/5 - Bancas de Mestrado/20160122_UFPE_BANCA_MPROF_HILSON-BARBOSA-DA-SILVA_DECLARACAO}}

\newpage
\subsection{Participação em Banca de Disserta\c{c}\~{a}o de Mestrado Profissional de José Rafael Moraes Garcia da Rocha}
\label{app:2016-mprof-jrmgr}
Esta subseção apresenta o comprovante da atuação como membro na Banca Examinadora de Defesa de Disserta\c{c}\~{a}o de Mestrado Profissional de José Rafael Moraes Garcia da Rocha no Centro de Informática/UFPE.
\includepdf[pages=-, scale=1,pagecommand=\thispagestyle{empty}]{\detokenize{GRUPO 1 – Atividades - Ensino/Sub-Grupo 1.2 - Participação em Comissões Examinadoras/5 - Bancas de Mestrado/20160115_UFPE_BANCA_MPROF_JOSE-RAFAEL-MORAES-GARCIA-DA-ROCHA_PORTARIA}}
\includepdf[pages=-, scale=1,pagecommand=\thispagestyle{empty}]{\detokenize{GRUPO 1 – Atividades - Ensino/Sub-Grupo 1.2 - Participação em Comissões Examinadoras/5 - Bancas de Mestrado/20160115_UFPE_BANCA_MPROF_JOSE-RAFAEL-MORAES-GARCIA-DA-ROCHA_DECLARACAO}}

\newpage
\subsection{Participação em Banca de Disserta\c{c}\~{a}o de Mestrado de Helaine Solange Lins Barreiros}
\label{app:2015-msc-hslb}
Esta subseção apresenta o comprovante da atuação membro na Banca Examinadora de Defesa de Disserta\c{c}\~{a}o de Mestrado de Helaine Solange Lins Barreiros no Centro de Informática/UFPE.
\includepdf[pages=-, scale=1,pagecommand=\thispagestyle{empty}]{\detokenize{GRUPO 1 – Atividades - Ensino/Sub-Grupo 1.2 - Participação em Comissões Examinadoras/5 - Bancas de Mestrado/20150831_UFPE_BANCA_MSc_HELAINE-SOLANGE-LINS-BARREIROS_PORTARIA}}
\includepdf[pages=-, scale=1,pagecommand=\thispagestyle{empty}]{\detokenize{GRUPO 1 – Atividades - Ensino/Sub-Grupo 1.2 - Participação em Comissões Examinadoras/5 - Bancas de Mestrado/20150831_UFPE_BANCA_MSc_HELAINE-SOLANGE-LINS-BARREIROS_DECLARACAO}}

\newpage
\subsection{Participação em Banca de Disserta\c{c}\~{a}o de Mestrado de José Paulo da Silva Lima}
\label{app:2015-msc-jpsl}
Esta subseção apresenta o comprovante da atuação como membro na Banca Examinadora de Defesa de Disserta\c{c}\~{a}o de Mestrado de José Paulo da Silva Lima no Centro de Informática/UFPE.
\includepdf[pages=-, scale=1,pagecommand=\thispagestyle{empty}]{\detokenize{GRUPO 1 – Atividades - Ensino/Sub-Grupo 1.2 - Participação em Comissões Examinadoras/5 - Bancas de Mestrado/20150831_UFPE_BANCA_MSc_JOSE-PAULO-DA-SILVA-LIMA_PORTARIA}}
\includepdf[pages=-, scale=1,pagecommand=\thispagestyle{empty}]{\detokenize{GRUPO 1 – Atividades - Ensino/Sub-Grupo 1.2 - Participação em Comissões Examinadoras/5 - Bancas de Mestrado/20150831_UFPE_BANCA_MSc_JOSE-PAULO-DA-SILVA-LIMA_DECLARACAO}}

\newpage
\subsection{Participação em Banca de Disserta\c{c}\~{a}o de Mestrado Profissional de Giselle Cirstina Moreno Tavares}
\label{app:2015-mprof-gcmt}
Esta subseção apresenta o comprovante da atuação como membro na Banca Examinadora de Defesa de Disserta\c{c}\~{a}o de Mestrado Profissional de Giselle Cirstina Moreno Tavares.
\includepdf[pages=-, scale=1,pagecommand=\thispagestyle{empty}]{\detokenize{GRUPO 1 – Atividades - Ensino/Sub-Grupo 1.2 - Participação em Comissões Examinadoras/5 - Bancas de Mestrado/20150618_UFPE_BANCA_MPROF_GISELLE-CRISTINA-MORENO-TAVARES_PORTARIA}}
\includepdf[pages=-, scale=1,pagecommand=\thispagestyle{empty}]{\detokenize{GRUPO 1 – Atividades - Ensino/Sub-Grupo 1.2 - Participação em Comissões Examinadoras/5 - Bancas de Mestrado/20150618_UFPE_BANCA_MPROF_GISELLE-CRISTINA-MORENO-TAVARES_DECLARACAO}}

\newpage
\subsection{Participação em Banca de Disserta\c{c}\~{a}o de Mestrado Profissional de Álvaro Magnum Barbosa Neto}
\label{app:2015-mprof-ambn}
Esta subseção apresenta o comprovante da atuação como membro na Banca Examinadora de Defesa de Disserta\c{c}\~{a}o de Mestrado Profissional de Álvaro Magnum Barbosa Neto no Centro de Informática/UFPE.
\includepdf[pages=-, scale=1,pagecommand=\thispagestyle{empty}]{\detokenize{GRUPO 1 – Atividades - Ensino/Sub-Grupo 1.2 - Participação em Comissões Examinadoras/5 - Bancas de Mestrado/20150522_UFPE_BANCA_MPROF_ALVARO-MAGNUM-BARBOSA-NETO_PORTARIA}}
\includepdf[pages=-, scale=1,pagecommand=\thispagestyle{empty}]{\detokenize{GRUPO 1 – Atividades - Ensino/Sub-Grupo 1.2 - Participação em Comissões Examinadoras/5 - Bancas de Mestrado/20150522_UFPE_BANCA_MPROF_ALVARO-MAGNUM-BARBOSA-NETO_DECLARACAO}}

\newpage
\subsection{Participação em Banca de Disserta\c{c}\~{a}o de Mestrado de Paulo Artur de Sousa Duarte}
\label{app:2014-msc-ufc-pasd}
Esta subseção apresenta o comprovante da atuação como membro na Banca Examinadora de Defesa de Disserta\c{c}\~{a}o de Mestrado de Paulo Artur de Sousa Duarte no Programa de Pós-Graduação em Ciências da Computação/UFC.
\includepdf[pages=-, scale=1,pagecommand=\thispagestyle{empty}]{\detokenize{GRUPO 1 – Atividades - Ensino/Sub-Grupo 1.2 - Participação em Comissões Examinadoras/5 - Bancas de Mestrado/20141114_UFC_MSC_PAULO-ARTUR-DE-SOUSA-DUARTE_DECLARACAO}}

\newpage
\subsection{Participação em Banca de Disserta\c{c}\~{a}o de Mestrado de João Carlos Sedraz Silva}
\label{app:2014-msc-jcss}
Esta subseção apresenta o comprovante da atuação como membro na Banca Examinadora de Defesa de Disserta\c{c}\~{a}o de Mestrado de João Carlos Sedraz Silva no Centro de Informática/UFPE.
\includepdf[pages=-, scale=1,pagecommand=\thispagestyle{empty}]{\detokenize{GRUPO 1 – Atividades - Ensino/Sub-Grupo 1.2 - Participação em Comissões Examinadoras/5 - Bancas de Mestrado/20140926_UFPE_BANCA_MSc_JOAO-CARLOS-SEDRAZ-SILVA_PORTARIA}}
\includepdf[pages=-, scale=1,pagecommand=\thispagestyle{empty}]{\detokenize{GRUPO 1 – Atividades - Ensino/Sub-Grupo 1.2 - Participação em Comissões Examinadoras/5 - Bancas de Mestrado/20140926_UFPE_BANCA_MSc_JOAO-CARLOS-SEDRAZ-SILVA_DECLARACAO}}

\newpage
\subsection{Participação em Banca de Disserta\c{c}\~{a}o de Mestrado de Thiago Jamir e Silva}
\label{app:2014-msc-tjs}
Esta subseção apresenta o comprovante da atuação como membro na Banca Examinadora de Defesa de Disserta\c{c}\~{a}o de Mestrado de Thiago Jamir e Silva no Centro de Informática/UFPE.
\includepdf[pages=-, scale=1,pagecommand=\thispagestyle{empty}]{\detokenize{GRUPO 1 – Atividades - Ensino/Sub-Grupo 1.2 - Participação em Comissões Examinadoras/5 - Bancas de Mestrado/20140905_UFPE_BANCA_MSc_THIAGO-JAMIR-E-SILVA_PORTARIA}}
\includepdf[pages=-, scale=1,pagecommand=\thispagestyle{empty}]{\detokenize{GRUPO 1 – Atividades - Ensino/Sub-Grupo 1.2 - Participação em Comissões Examinadoras/5 - Bancas de Mestrado/20140905_UFPE_BANCA_MSc_THIAGO-JAMIR-E-SILVA_DECLARACAO}}

\newpage
\subsection{Participação em Banca de Disserta\c{c}\~{a}o de Mestrado de Jean Louis Brasil Fernandes Costa}
\label{app:2014-msc-jlbfc}
Esta subseção apresenta o comprovante da atuação como membro na Banca Examinadora de Defesa de Disserta\c{c}\~{a}o de Mestrado de Jean Louis Brasil Fernandes Costa no Centro de Informática/UFPE.
\includepdf[pages=-, scale=1,pagecommand=\thispagestyle{empty}]{\detokenize{GRUPO 1 – Atividades - Ensino/Sub-Grupo 1.2 - Participação em Comissões Examinadoras/5 - Bancas de Mestrado/20140904_UFPE_BANCA_MSc_JEAN-LOUIS-BRASIL-FERNANDES_PORTARIA}}
\includepdf[pages=-, scale=1,pagecommand=\thispagestyle{empty}]{\detokenize{GRUPO 1 – Atividades - Ensino/Sub-Grupo 1.2 - Participação em Comissões Examinadoras/5 - Bancas de Mestrado/20140904_UFPE_BANCA_MSc_JEAN-LOUIS-BRASIL-FERNANDES_DECLARACAO}}

\newpage
\subsection{Participação em Banca de Disserta\c{c}\~{a}o de Mestrado de Héldon José Oliveira Albuquerque}
\label{app:2014-msc-hjoa}
Esta subseção apresenta o comprovante da atuação como membro na Banca Examinadora de Defesa de Disserta\c{c}\~{a}o de Mestrado de Héldon José Oliveira Albuquerque no Programa de Pós-Graduação em Sistemas e Computação do Centro de Ciências Exatas e da Terra/UFRN.
\includepdf[pages=-, scale=1,pagecommand=\thispagestyle{empty}]{\detokenize{GRUPO 1 – Atividades - Ensino/Sub-Grupo 1.2 - Participação em Comissões Examinadoras/5 - Bancas de Mestrado/20140829_UFRN_BANCA_MSc_HELDON-JOSE-OLIVEIRA-ALBUQUERQUE_DECLARACAO}}

\newpage
\subsection{Participação em Banca de Disserta\c{c}\~{a}o de Mestrado de Welington Manoel da Silva.}
\label{app:2014-msc-wms}
Esta subseção apresenta o comprovante da atuação como membro na Banca Examinadora de Defesa de Disserta\c{c}\~{a}o de Mestrado de Welington Manoel da Silva  no Centro de Informática/UFPE.
\includepdf[pages=-, scale=1,pagecommand=\thispagestyle{empty}]{\detokenize{GRUPO 1 – Atividades - Ensino/Sub-Grupo 1.2 - Participação em Comissões Examinadoras/5 - Bancas de Mestrado/20140825_UFPE_BANCA_MSc_WELINGTON-MANOEL-DA-SILVA_PORTARIA}}
\includepdf[pages=-, scale=1,pagecommand=\thispagestyle{empty}]{\detokenize{GRUPO 1 – Atividades - Ensino/Sub-Grupo 1.2 - Participação em Comissões Examinadoras/5 - Bancas de Mestrado/20140825_UFPE_BANCA_MSc_WELINGTON-MANOEL-DA-SILVA_DECLARACAO}}

%------------------------------------------------------------------------------

\newpage
\subsection{Participação em Banca de Qualifica\c{c}\~ao / Proposta de Tese de Doutorado de Ioram Schechtman Sette.}
\label{app:2016-quali-phd-prga}
Esta subseção apresenta o comprovante da atuação como membro na Banca Examinadora de Defesa de Proposta de Tese de Doutorado de Paola Rodrigues Godoy Accioly no Centro de Informática/UFPE.
\includepdf[pages=-, scale=1,pagecommand=\thispagestyle{empty}]{\detokenize{GRUPO 1 – Atividades - Ensino/Sub-Grupo 1.2 - Participação em Comissões Examinadoras/6 - Bancas de Qualificação de Doutorado/20160301_UFPE_QUALI-PHD_PAOLA-RODRIGUES-GODOY-ACCIOLY_PORTARIA}}
\includepdf[pages=-, scale=1,pagecommand=\thispagestyle{empty}]{\detokenize{GRUPO 1 – Atividades - Ensino/Sub-Grupo 1.2 - Participação em Comissões Examinadoras/6 - Bancas de Qualificação de Doutorado/20160301_UFPE_QUALI-PHD_PAOLA-RODRIGUES-GODOY-ACCIOLY_DECLARACAO}}

\newpage
\subsection{Participação em Banca de Qualifica\c{c}\~ao / Proposta de Tese de Doutorado de Ioram Schechtman Sette.}
\label{app:2015-quali-phd-iss}
Esta subseção apresenta o comprovante da atuação como membro na Banca Examinadora de Defesa de Proposta de Tese de Doutorado de Ioram Schechtman Sette no Centro de Informática/UFPE.
\includepdf[pages=-, scale=1,pagecommand=\thispagestyle{empty}]{\detokenize{GRUPO 1 – Atividades - Ensino/Sub-Grupo 1.2 - Participação em Comissões Examinadoras/6 - Bancas de Qualificação de Doutorado/20151218_UFPE_QUALI-PHD_IORAM-SCHECHTMAN-SETTE_PORTARIA}}
\includepdf[pages=-, scale=1,pagecommand=\thispagestyle{empty}]{\detokenize{GRUPO 1 – Atividades - Ensino/Sub-Grupo 1.2 - Participação em Comissões Examinadoras/6 - Bancas de Qualificação de Doutorado/20151218_UFPE_QUALI-PHD_IORAM-SCHECHTMAN-SETTE_DECLARACAO}}

\newpage
\subsection{Participação em Banca de Qualifica\c{c}\~ao / Proposta de Tese de Doutorado de Thiago Monteiro Prota.}
\label{app:2015-quali-phd-tmp}
Esta subseção apresenta o comprovante da atuação como membro na Banca Examinadora de Defesa de Proposta de Tese de Doutorado de Thiago Monteiro Prota no Centro de Informática/UFPE.
\includepdf[pages=-, scale=1,pagecommand=\thispagestyle{empty}]{\detokenize{GRUPO 1 – Atividades - Ensino/Sub-Grupo 1.2 - Participação em Comissões Examinadoras/6 - Bancas de Qualificação de Doutorado/20150918_UFPE_QUALI-PHD_THIAGO-MONTEIRO-PROTA_PORTARIA}}
\includepdf[pages=-, scale=1,pagecommand=\thispagestyle{empty}]{\detokenize{GRUPO 1 – Atividades - Ensino/Sub-Grupo 1.2 - Participação em Comissões Examinadoras/6 - Bancas de Qualificação de Doutorado/20150918_UFPE_QUALI-PHD_THIAGO-MONTEIRO-PROTA_DECLARACAO}}

\newpage
\subsection{Participação em Banca de Qualifica\c{c}\~ao / Proposta de Tese de Doutorado de Bruno Neiva Moreno.}
\label{app:2015-quali-phd-bnm}
Esta subseção apresenta o comprovante da atuação como membro na Banca Examinadora de Defesa de Proposta de Tese de Doutorado de Bruno Neiva Moreno no Centro de Informática/UFPE.
\includepdf[pages=-, scale=1,pagecommand=\thispagestyle{empty}]{\detokenize{GRUPO 1 – Atividades - Ensino/Sub-Grupo 1.2 - Participação em Comissões Examinadoras/6 - Bancas de Qualificação de Doutorado/20150406_UFPE_QUALI-PHD_BRUNO-NEIVA-MORENO_PORTARIA}}
\includepdf[pages=-, scale=1,pagecommand=\thispagestyle{empty}]{\detokenize{GRUPO 1 – Atividades - Ensino/Sub-Grupo 1.2 - Participação em Comissões Examinadoras/6 - Bancas de Qualificação de Doutorado/20150406_UFPE_QUALI-PHD_BRUNO-NEIVA-MORENO_DECLARACAO}}

\newpage
\subsection{Participação em Banca de Qualifica\c{c}\~ao / Proposta de Tese de Doutorado de Julio Cesar Damasceno.}
\label{app:2015-quali-phd-jcd}
Esta subseção apresenta o comprovante da atuação como membro na Banca Examinadora de Defesa de Proposta de Tese de Doutorado de Julio Cesar Damasceno no Centro de Informática/UFPE.
\includepdf[pages=-, scale=1,pagecommand=\thispagestyle{empty}]{\detokenize{GRUPO 1 – Atividades - Ensino/Sub-Grupo 1.2 - Participação em Comissões Examinadoras/6 - Bancas de Qualificação de Doutorado/20150320_UFPE_QUALI-PHD_JULIO-CESAR-DAMASCENO_PORTARIA}}
\includepdf[pages=-, scale=1,pagecommand=\thispagestyle{empty}]{\detokenize{GRUPO 1 – Atividades - Ensino/Sub-Grupo 1.2 - Participação em Comissões Examinadoras/6 - Bancas de Qualificação de Doutorado/20150320_UFPE_QUALI-PHD_JULIO-CESAR-DAMASCENO_DECLARACAO}}

\newpage
\subsection{Participação em Banca de Qualifica\c{c}\~ao / Proposta de Tese de Doutorado de Liliane Sheyla da Silva Fonseca.}
\label{app:2015-quali-phd-lssf}
Esta subseção apresenta o comprovante da atuação como membro na Banca Examinadora de Defesa de Proposta de Tese de Doutorado de Liliane Sheyla da Silva Fonseca no Centro de Informática/UFPE.
\includepdf[pages=-, scale=1,pagecommand=\thispagestyle{empty}]{\detokenize{GRUPO 1 – Atividades - Ensino/Sub-Grupo 1.2 - Participação em Comissões Examinadoras/6 - Bancas de Qualificação de Doutorado/20150211_UFPE_QUALI-PHD_LILIANE_SHEYLA_DA_SILVA_FONSECA_PORTARIA}}
\includepdf[pages=-, scale=1,pagecommand=\thispagestyle{empty}]{\detokenize{GRUPO 1 – Atividades - Ensino/Sub-Grupo 1.2 - Participação em Comissões Examinadoras/6 - Bancas de Qualificação de Doutorado/20150211_UFPE_QUALI-PHD_LILIANE_SHEYLA_DA_SILVA_FONSECA_DECLARACAO}}

\newpage
\subsection{Participação em Banca de Qualifica\c{c}\~ao / Proposta de Tese de Doutorado de Rodrigo da Cruz Fujioka.}
\label{app:2015-quali-phd-rcf}
Esta subseção apresenta o comprovante da atuação como membro na Banca Examinadora de Defesa de Proposta de Tese de Doutorado de Rodrigo da Cruz Fujioka no Centro de Informática/UFPE.
\includepdf[pages=-, scale=1,pagecommand=\thispagestyle{empty}]{\detokenize{GRUPO 1 – Atividades - Ensino/Sub-Grupo 1.2 - Participação em Comissões Examinadoras/6 - Bancas de Qualificação de Doutorado/20150123_UFPE_QUALI-PHD_RODRIGO-DA-CRUZ-FUJIOKA_PORTARIA}}
\includepdf[pages=-, scale=1,pagecommand=\thispagestyle{empty}]{\detokenize{GRUPO 1 – Atividades - Ensino/Sub-Grupo 1.2 - Participação em Comissões Examinadoras/6 - Bancas de Qualificação de Doutorado/20150123_UFPE_QUALI-PHD_RODRIGO-DA-CRUZ-FUJIOKA_DECLARACAO}}

\newpage
\subsection{Participação em Banca de Qualifica\c{c}\~ao / Proposta de Tese de Doutorado de Elias Adriano Nogueira da Silva.}
\label{app:2014-quali-phd-eans}
Esta subseção apresenta o comprovante da atuação como membro na Banca Examinadora de Defesa de Proposta de Tese de Doutorado de Elias Adriano Nogueira da Silva no Instituto de Ciências Matemáticas e de Computação (ICMC)/USP, Campus São Carlos/SP.
\includepdf[pages=-, scale=1,pagecommand=\thispagestyle{empty}]{\detokenize{GRUPO 1 – Atividades - Ensino/Sub-Grupo 1.2 - Participação em Comissões Examinadoras/6 - Bancas de Qualificação de Doutorado/20140821_USP_QUALI-PHD_ELIAS-ADRIANO-NOGUEIRA-DA-SILVA_DECLARACAO}}

%------------------------------------------------------------------------------

\newpage
\subsection{Participação em Banca de Tese de Doutorado de Rodrigo Alves Costa.}
\label{app:2016-phd-rac2}
Esta subseção apresenta o comprovante da atuação como membro na Banca Examinadora de Defesa de Tese de Doutorado de Rodrigo Alves Costa no Centro de Informática/UFPE.
\includepdf[pages=-, scale=1,pagecommand=\thispagestyle{empty}]{\detokenize{GRUPO 1 – Atividades - Ensino/Sub-Grupo 1.2 - Participação em Comissões Examinadoras/7 - Bancas de Doutorado/20160304_UFPE_PHD_RODRIGO-ALVES-COSTA_PORTARIA}}
\includepdf[pages=-, scale=1,pagecommand=\thispagestyle{empty}]{\detokenize{GRUPO 1 – Atividades - Ensino/Sub-Grupo 1.2 - Participação em Comissões Examinadoras/7 - Bancas de Doutorado/20160304_UFPE_PHD_RODRIGO-ALVES-COSTA_DECLARACAO}}

\newpage
\subsection{Participação em Banca de Tese de Doutorado de Julio Cesar Damasceno.}
\label{app:2015-phd-jcd}
Esta subseção apresenta o comprovante da atuação como membro na Banca Examinadora de Defesa de Tese de Doutorado de Julio Cesar Damasceno no Centro de Informática/UFPE.
\includepdf[pages=-, scale=1,pagecommand=\thispagestyle{empty}]{\detokenize{GRUPO 1 – Atividades - Ensino/Sub-Grupo 1.2 - Participação em Comissões Examinadoras/7 - Bancas de Doutorado/20150903_UFPE_PHD_JULIO-CESAR-DAMASCENO_PORTARIA}}
\includepdf[pages=-, scale=1,pagecommand=\thispagestyle{empty}]{\detokenize{GRUPO 1 – Atividades - Ensino/Sub-Grupo 1.2 - Participação em Comissões Examinadoras/7 - Bancas de Doutorado/20150903_UFPE_PHD_JULIO-CESAR-DAMASCENO_DECLARACAO}}

\newpage
\subsection{Participação em Banca de Tese de Doutorado de Rodrigo Gusmão de Carvalho Rocha.}
\label{app:2015-phd-rgcr}
Esta subseção apresenta o comprovante da atuação como membro na Banca Examinadora de Defesa de Tese de Doutorado de Rodrigo Gusmão de Carvalho Rocha no Centro de Informática/UFPE.
\includepdf[pages=-, scale=1,pagecommand=\thispagestyle{empty}]{\detokenize{GRUPO 1 – Atividades - Ensino/Sub-Grupo 1.2 - Participação em Comissões Examinadoras/7 - Bancas de Doutorado/20150220_UFPE_PHD_RODRIGO_GUSMAO_DE_CARVALHO_ROCHA_PORTARIA}}
\includepdf[pages=-, scale=1,pagecommand=\thispagestyle{empty}]{\detokenize{GRUPO 1 – Atividades - Ensino/Sub-Grupo 1.2 - Participação em Comissões Examinadoras/7 - Bancas de Doutorado/20150220_UFPE_PHD_RODRIGO_GUSMAO_DE_CARVALHO_ROCHA_DECLARACAO}}

\newpage
\subsection{Participação em Banca de Tese de Doutorado de Cleviton Vinicius Fonseca Monteiro.}
\label{app:2015-phd-cvfm}
Esta subseção apresenta o comprovante da atuação como membro na Banca Examinadora de Defesa de Tese de Doutorado de Cleviton Vinicius Fonseca Monteiro no Centro de Informática/UFPE.
\includepdf[pages=-, scale=1,pagecommand=\thispagestyle{empty}]{\detokenize{GRUPO 1 – Atividades - Ensino/Sub-Grupo 1.2 - Participação em Comissões Examinadoras/7 - Bancas de Doutorado/20141121_UFPE_PHD_CLEVITON-VINICIUS-FONSECA-MONTEIRO_PORTARIA}}
\includepdf[pages=-, scale=1,pagecommand=\thispagestyle{empty}]{\detokenize{GRUPO 1 – Atividades - Ensino/Sub-Grupo 1.2 - Participação em Comissões Examinadoras/7 - Bancas de Doutorado/20141121_UFPE_PHD_CLEVITON-VINICIUS-FONSECA-MONTEIRO_DECLARACAO}}

\newpage
\subsection{Participa\c{c}\~{a}o como Membro Externo da Comissão de Seleção para o Doutorado Sanduíche - PDSE da CAPES}
\label{app:2015-pgcomp-pdse}
Esta subseção apresenta o comprovante da atuação como membro Externo da Comissão de Seleção para o Doutorado Sanduíche - PDSE da CAPES.
\includepdf[pages=-, scale=1,pagecommand=\thispagestyle{empty}]{\detokenize{GRUPO 1 – Atividades - Ensino/Sub-Grupo 1.2 - Participação em Comissões Examinadoras/8 - Participação em Banca de Seleção para ingresso e exames de qualificação em programa de pós-graduação stricto sensu/20150720_Membro-Comissao-PGCOMP-PDSE}}

%%%%%%%%%%%%%%%%%%%%%%%%%%%%%%%%%%%%%%%%%%%%%%%%%%%%%%%%%%%%%%%%%%%%%%%%%%%%%%%
% Subgrupo 1.3 - Atividades de Ensino na Graduação e na Pós-Graduação
%%%%%%%%%%%%%%%%%%%%%%%%%%%%%%%%%%%%%%%%%%%%%%%%%%%%%%%%%%%%%%%%%%%%%%%%%%%%%%%



%%%%%%%%%%%%%%%%%%%%%%%%%%%%%%%%%%%%%%%%%%%%%%%%%%%%%%%%%%%%%%%%%%%%%%%%%%%%%%%
% Subgrupo 1.4 - Avaliação Didática Docente pelo Discente
%%%%%%%%%%%%%%%%%%%%%%%%%%%%%%%%%%%%%%%%%%%%%%%%%%%%%%%%%%%%%%%%%%%%%%%%%%%%%%%

\newpage
\subsection{Avaliação Didática Docente pelo Discente}
\label{evaluation:2016-1}
Esta subseção apresenta o comprovante da Avaliação Didática Docente pelo Discente para o período do primeiro semestre de 2016.
%\includepdf[pages=-, scale=1,pagecommand=\thispagestyle{empty}]{\detokenize{GRUPO 1 – Atividades - Ensino/Sub-Grupo 1.4 - Avaliação Didática Docente pelo Discente/2016.1 Relatório de avaliação do docente}}

\newpage
\subsection{Avaliação Didática Docente pelo Discente}
\label{evaluation:2015-2}
Esta subseção apresenta o comprovante da Avaliação Didática Docente pelo Discente para o período do segundo semestre de 2015.
\includepdf[pages=-, scale=1,pagecommand=\thispagestyle{empty}]{\detokenize{GRUPO 1 – Atividades - Ensino/Sub-Grupo 1.4 - Avaliação Didática Docente pelo Discente/2015.2 Relatório de avaliação do docente}}

\newpage
\subsection{Avaliação Didática Docente pelo Discente}
\label{evaluation:2015-1}
Esta subseção apresenta o comprovante da Avaliação Didática Docente pelo Discente para o período do primeiro semestre de 2015.
\includepdf[pages=-, scale=1,pagecommand=\thispagestyle{empty}]{\detokenize{GRUPO 1 – Atividades - Ensino/Sub-Grupo 1.4 - Avaliação Didática Docente pelo Discente/2015.1 Relatório de avaliação do docente}}

%%%%%%%%%%%%%%%%%%%%%%%%%%%%%%%%%%%%%%%%%%%%%%%%%%%%%%%%%%%%%%%%%%%%%%%%%%%%%%%
% Subgrupo 2.1 - Produtividade de Pesquisa
%%%%%%%%%%%%%%%%%%%%%%%%%%%%%%%%%%%%%%%%%%%%%%%%%%%%%%%%%%%%%%%%%%%%%%%%%%%%%%%


\newpage
\subsection{Participa\c{c}\~{a}o em Eventos Cient\'{\i}ficos (com apresenta\c{c}\~{a}o de trabalho ou oferecimento de cursos, palestras ou debates}
\label{app:2015-cbsoft}
Esta subseção apresenta o comprovante da participação no Congresso Brasileiro de Software: Teoria e Prática (CBSoft'2015) com seus respectivos propósitos.
\includepdf[pages=-, scale=1,pagecommand=\thispagestyle{empty}]{\detokenize{GRUPO 2 – Atividades - Produção Científica Técnica Artística e Cultural/Sub-Grupo 2.1 - Produtividade de Pesquisa/B - Participação com apresentação de trabalho ou oferecimento de cursos palestras ou debates em congressos seminários e simpósios/20150925_CBSoft2015_Participante}}

\newpage
\subsection{Participa\c{c}\~{a}o em Eventos Cient\'{\i}ficos (com apresenta\c{c}\~{a}o de trabalho ou oferecimento de cursos, palestras ou debates}
\label{app:2015-enacomp}
Esta subseção apresenta o comprovante da participação no XII Encontro Anual de Computação (EnAComp) com seus respectivos propósitos.
\includepdf[pages=-, scale=1,pagecommand=\thispagestyle{empty}]{\detokenize{GRUPO 2 – Atividades - Produção Científica Técnica Artística e Cultural/Sub-Grupo 2.1 - Produtividade de Pesquisa/B - Participação com apresentação de trabalho ou oferecimento de cursos palestras ou debates em congressos seminários e simpósios/20150918_ENACOMP-Participante}}
\includepdf[pages=-, scale=1,pagecommand=\thispagestyle{empty}]{\detokenize{GRUPO 2 – Atividades - Produção Científica Técnica Artística e Cultural/Sub-Grupo 2.1 - Produtividade de Pesquisa/B - Participação com apresentação de trabalho ou oferecimento de cursos palestras ou debates em congressos seminários e simpósios/20150918_ENACOMP-Keynote-NIS}}
\includepdf[pages=-, scale=1,pagecommand=\thispagestyle{empty}]{\detokenize{GRUPO 2 – Atividades - Produção Científica Técnica Artística e Cultural/Sub-Grupo 2.1 - Produtividade de Pesquisa/B - Participação com apresentação de trabalho ou oferecimento de cursos palestras ou debates em congressos seminários e simpósios/20150918_ENACOMP-Minicurso-ESS}}

\newpage
\subsection{Participa\c{c}\~{a}o em Eventos Cient\'{\i}ficos (com apresenta\c{c}\~{a}o de trabalho ou oferecimento de cursos, palestras ou debates}
\label{app:2015-19jic-facepe}
Esta subseção apresenta o comprovante da participação na 19ª Jornada de Iniciação Científica da FACEPE com seus respectivos propósitos.
\includepdf[pages=-, scale=1,pagecommand=\thispagestyle{empty}]{\detokenize{GRUPO 2 – Atividades - Produção Científica Técnica Artística e Cultural/Sub-Grupo 2.1 - Produtividade de Pesquisa/B - Participação com apresentação de trabalho ou oferecimento de cursos palestras ou debates em congressos seminários e simpósios/20150612_19a JIC-FACEPE - Ofício}}
\includepdf[pages=-, scale=1,pagecommand=\thispagestyle{empty}]{\detokenize{GRUPO 2 – Atividades - Produção Científica Técnica Artística e Cultural/Sub-Grupo 2.1 - Produtividade de Pesquisa/B - Participação com apresentação de trabalho ou oferecimento de cursos palestras ou debates em congressos seminários e simpósios/20150612_19a-JIC-FACEPE_Palestrante}}

\newpage
\subsection{Participa\c{c}\~{a}o em Eventos Cient\'{\i}ficos (com apresenta\c{c}\~{a}o de trabalho ou oferecimento de cursos, palestras ou debates}
\label{app:2015-xi-sbsi}
Esta subseção apresenta o comprovante da participação no XI Simpósio Brasileiro de Sistemas de Informação (SBSI) com seus respectivos propósitos.
\includepdf[pages=-, scale=1,pagecommand=\thispagestyle{empty}]{\detokenize{GRUPO 2 – Atividades - Produção Científica Técnica Artística e Cultural/Sub-Grupo 2.1 - Produtividade de Pesquisa/B - Participação com apresentação de trabalho ou oferecimento de cursos palestras ou debates em congressos seminários e simpósios/20150529_SBSI - Certificado minicurso}}
\includepdf[pages=-, scale=1,pagecommand=\thispagestyle{empty}]{\detokenize{GRUPO 2 – Atividades - Produção Científica Técnica Artística e Cultural/Sub-Grupo 2.1 - Produtividade de Pesquisa/B - Participação com apresentação de trabalho ou oferecimento de cursos palestras ou debates em congressos seminários e simpósios/20150529_SBSI - SmartCluster}}
\includepdf[pages=-, scale=1,pagecommand=\thispagestyle{empty}]{\detokenize{GRUPO 2 – Atividades - Produção Científica Técnica Artística e Cultural/Sub-Grupo 2.1 - Produtividade de Pesquisa/B - Participação com apresentação de trabalho ou oferecimento de cursos palestras ou debates em congressos seminários e simpósios/20150529_SBSI - Um Modelo de Recomendação}}

\newpage
\subsection{Participa\c{c}\~{a}o em Eventos Cient\'{\i}ficos (com apresenta\c{c}\~{a}o de trabalho ou oferecimento de cursos, palestras ou debates}
\label{app:2015-ep-senac}
Esta subseção apresenta o comprovante da participação no Encontro Pedagógico 2015, promovido pelo Serviço Nacional de Aprendizagem Comercial (SENAC) com seus respectivos propósitos.
\includepdf[pages=-, scale=1,pagecommand=\thispagestyle{empty}]{\detokenize{GRUPO 2 – Atividades - Produção Científica Técnica Artística e Cultural/Sub-Grupo 2.1 - Produtividade de Pesquisa/B - Participação com apresentação de trabalho ou oferecimento de cursos palestras ou debates em congressos seminários e simpósios/20150206_SENAC_doc20150515105941}}

\newpage
\subsection{Participa\c{c}\~{a}o em Eventos Cient\'{\i}ficos (com apresenta\c{c}\~{a}o de trabalho ou oferecimento de cursos, palestras ou debates}
\label{app:2014-cbsoft}
Esta subseção apresenta o comprovante da participação no V Congresso Brasileiro de Software: Teoria e Prática (CBSoft) com seus respectivos propósitos.
\includepdf[pages=-, scale=1,pagecommand=\thispagestyle{empty}]{\detokenize{GRUPO 2 – Atividades - Produção Científica Técnica Artística e Cultural/Sub-Grupo 2.1 - Produtividade de Pesquisa/B - Participação com apresentação de trabalho ou oferecimento de cursos palestras ou debates em congressos seminários e simpósios/20141003_CBSoft_Participacao}}


%---

\newpage
\subsection{Autoria de artigos completos publicados em anais de congresso, em jornais e revistas de circulação nacional e internacional na sua área}
\label{conf:2016-noms-b}
Esta subseção apresenta o comprovante da autoria de artigo completo publicado em proceedings do IEEE/IFIP Network Operations and Management Symposium (NOMS 2016)\footnote{\url{http://noms2016.ieee-noms.org}}.
\includepdf[pages=-, scale=1,pagecommand=\thispagestyle{empty}]{\detokenize{GRUPO 2 – Atividades - Produção Científica Técnica Artística e Cultural/Sub-Grupo 2.1 - Produtividade de Pesquisa/C - Autoria de artigos completos publicados em anais de congresso em jornais e revistas de circulação nacional e internacional na sua área/20161120 NOMS 2016 paper 148608}}
%\includepdf[pages=-, scale=1,pagecommand=\thispagestyle{empty}]{\detokenize{GRUPO 2 – Atividades - Produção Científica Técnica Artística e Cultural/Sub-Grupo 2.1 - Produtividade de Pesquisa/C - Autoria de artigos completos publicados em anais de congresso em jornais e revistas de circulação nacional e internacional na sua área/20151021_WTD - WebMedia 2015 certificado for paper 103}}

\newpage
\subsection{Autoria de artigos completos publicados em anais de congresso, em jornais e revistas de circulação nacional e internacional na sua área}
\label{conf:2016-noms-a}
Esta subseção apresenta o comprovante da autoria de artigo completo publicado em proceedings do IEEE/IFIP Network Operations and Management Symposium (NOMS 2016)\footnote{\url{http://noms2016.ieee-noms.org}}.
\includepdf[pages=-, scale=1,pagecommand=\thispagestyle{empty}]{\detokenize{GRUPO 2 – Atividades - Produção Científica Técnica Artística e Cultural/Sub-Grupo 2.1 - Produtividade de Pesquisa/C - Autoria de artigos completos publicados em anais de congresso em jornais e revistas de circulação nacional e internacional na sua área/20161120 NOMS 2016 paper 148605}}
%\includepdf[pages=-, scale=1,pagecommand=\thispagestyle{empty}]{\detokenize{GRUPO 2 – Atividades - Produção Científica Técnica Artística e Cultural/Sub-Grupo 2.1 - Produtividade de Pesquisa/C - Autoria de artigos completos publicados em anais de congresso em jornais e revistas de circulação nacional e internacional na sua área/20151021_WTD - WebMedia 2015 certificado for paper 103}}

\newpage
\subsection{Autoria de artigos completos publicados em anais de congresso, em jornais e revistas de circulação nacional e internacional na sua área}
\label{conf:2015-wtd-webmedia}
Esta subseção apresenta o comprovante da autoria de artigo completo publicado em anais do IV Simpósio Brasileiro de Tecnologia da Informação (SBTI)\footnote{\url{http://www.sbti2015.com.br}}, o certificado de apresentação conferido ao meu aluno e co-autor do trabalho pela apresentação do mesmo bem como a versãp ``camera ready'' do artigo, submetida ao comitê de programa do workshop, uma vez que não existe publicações de anais para este evento.
\includepdf[pages=-, scale=1,pagecommand=\thispagestyle{empty}]{\detokenize{GRUPO 2 – Atividades - Produção Científica Técnica Artística e Cultural/Sub-Grupo 2.1 - Produtividade de Pesquisa/C - Autoria de artigos completos publicados em anais de congresso em jornais e revistas de circulação nacional e internacional na sua área/20151021_WTD - WebMedia 2015 notification for paper 103}}
\includepdf[pages=-, scale=1,pagecommand=\thispagestyle{empty}]{\detokenize{GRUPO 2 – Atividades - Produção Científica Técnica Artística e Cultural/Sub-Grupo 2.1 - Produtividade de Pesquisa/C - Autoria de artigos completos publicados em anais de congresso em jornais e revistas de circulação nacional e internacional na sua área/20151021_WTD - WebMedia 2015 certificado for paper 103}}
\includepdf[pages=-, scale=1,pagecommand=\thispagestyle{empty}]{\detokenize{GRUPO 2 – Atividades - Produção Científica Técnica Artística e Cultural/Sub-Grupo 2.1 - Produtividade de Pesquisa/C - Autoria de artigos completos publicados em anais de congresso em jornais e revistas de circulação nacional e internacional na sua área/20151021_WTD - WebMedia 2015 camera ready paper 103}}

\newpage
\subsection{Autoria de artigos completos publicados em anais de congresso, em jornais e revistas de circulação nacional e internacional na sua área}
\label{conf:2015-sbti}
Esta subseção apresenta o comprovante da autoria de artigo completo publicado em anais do IV Simpósio Brasileiro de Tecnologia da Informação (SBTI)\footnote{\url{IV Simpósio Brasileiro de Tecnologia da Informação (SBTI)}}.
\includepdf[pages=-, scale=1,pagecommand=\thispagestyle{empty}]{\detokenize{GRUPO 2 – Atividades - Produção Científica Técnica Artística e Cultural/Sub-Grupo 2.1 - Produtividade de Pesquisa/C - Autoria de artigos completos publicados em anais de congresso em jornais e revistas de circulação nacional e internacional na sua área/20151010_SBTI 2015 paper 148625 notification}}
\includepdf[pages=-, scale=1,pagecommand=\thispagestyle{empty}]{\detokenize{GRUPO 2 – Atividades - Produção Científica Técnica Artística e Cultural/Sub-Grupo 2.1 - Produtividade de Pesquisa/C - Autoria de artigos completos publicados em anais de congresso em jornais e revistas de circulação nacional e internacional na sua área/20151010_SBTI 2015 paper 148625 certificado}}

\newpage
\subsection{Autoria de artigos completos publicados em anais de congresso, em jornais e revistas de circulação nacional e internacional na sua área}
\label{conf:2015-sbseg}
Esta subseção apresenta o comprovante da autoria de artigo completo publicado em anais do XV Simpósio Brasileiro em Segurança da Informação e de Sistemas Computacionais (SBSeg'15) - Trilha de Trabalhos Completos\footnote{\url{http://sbseg2015.univali.br}}.
\includepdf[pages=-, scale=1,pagecommand=\thispagestyle{empty}]{\detokenize{GRUPO 2 – Atividades - Produção Científica Técnica Artística e Cultural/Sub-Grupo 2.1 - Produtividade de Pesquisa/C - Autoria de artigos completos publicados em anais de congresso em jornais e revistas de circulação nacional e internacional na sua área/20150831_SBSeg 2015 (full papers) paper 145725 notification}}
\includepdf[pages=-, scale=1,pagecommand=\thispagestyle{empty}]{\detokenize{GRUPO 2 – Atividades - Produção Científica Técnica Artística e Cultural/Sub-Grupo 2.1 - Produtividade de Pesquisa/C - Autoria de artigos completos publicados em anais de congresso em jornais e revistas de circulação nacional e internacional na sua área/20150831_SBSeg 2015 (full papers) paper 145725}}

\newpage
\subsection{Autoria de artigos completos publicados em anais de congresso, em jornais e revistas de circulação nacional e internacional na sua área}
\label{conf:2015-sbcup}
Esta subseção apresenta o comprovante da autoria de artigo completo publicado em anais do 7º Simpósio Brasileiro de Computação Ubíqua e Pervasiva (SBCUP)\footnote{\url{http://csbc2015.cin.ufpe.br/eventos_descricao/13}}, Evento do XXXV Congresso da Sociedade Brasileira de Computação.
\includepdf[pages=-, scale=1,pagecommand=\thispagestyle{empty}]{\detokenize{GRUPO 2 – Atividades - Produção Científica Técnica Artística e Cultural/Sub-Grupo 2.1 - Produtividade de Pesquisa/C - Autoria de artigos completos publicados em anais de congresso em jornais e revistas de circulação nacional e internacional na sua área/20150722_CSBC 2015 - SBCUP paper 144023 notification}}
\includepdf[pages=-, scale=1,pagecommand=\thispagestyle{empty}]{\detokenize{GRUPO 2 – Atividades - Produção Científica Técnica Artística e Cultural/Sub-Grupo 2.1 - Produtividade de Pesquisa/C - Autoria de artigos completos publicados em anais de congresso em jornais e revistas de circulação nacional e internacional na sua área/20150722_SBCUP_p15-wanderley}}

\newpage
\subsection{Autoria de artigos completos publicados em anais de congresso, em jornais e revistas de circulação nacional e internacional na sua área}
\label{conf:2015-dgo-41}
Esta subseção apresenta o comprovante da autoria de artigo completo publicado em proceedings do the 16th Annual International Conference on Digital Government Research (dg.o'15)\footnote{\url{http://dgo2015.dgsociety.org/about-conference}}.
\includepdf[pages=-, scale=1,pagecommand=\thispagestyle{empty}]{\detokenize{GRUPO 2 – Atividades - Produção Científica Técnica Artística e Cultural/Sub-Grupo 2.1 - Produtividade de Pesquisa/C - Autoria de artigos completos publicados em anais de congresso em jornais e revistas de circulação nacional e internacional na sua área/20150530_dg.o 2015 notification for paper 41}}
\includepdf[pages=-, scale=1,pagecommand=\thispagestyle{empty}]{\detokenize{GRUPO 2 – Atividades - Produção Científica Técnica Artística e Cultural/Sub-Grupo 2.1 - Produtividade de Pesquisa/C - Autoria de artigos completos publicados em anais de congresso em jornais e revistas de circulação nacional e internacional na sua área/20150530_dg.o 2015 paper 41}}

\newpage
\subsection{Autoria de artigos completos publicados em anais de congresso, em jornais e revistas de circulação nacional e internacional na sua área}
\label{conf:2015-dgo-37}
Esta subseção apresenta o comprovante da autoria de artigo completo publicado em proceedings do the 16th Annual International Conference on Digital Government Research (dg.o'15)\footnote{\url{http://dgo2015.dgsociety.org/about-conference}}.
\includepdf[pages=-, scale=1,pagecommand=\thispagestyle{empty}]{\detokenize{GRUPO 2 – Atividades - Produção Científica Técnica Artística e Cultural/Sub-Grupo 2.1 - Produtividade de Pesquisa/C - Autoria de artigos completos publicados em anais de congresso em jornais e revistas de circulação nacional e internacional na sua área/20150530_dg.o 2015 notification for paper 37}}
\includepdf[pages=-, scale=1,pagecommand=\thispagestyle{empty}]{\detokenize{GRUPO 2 – Atividades - Produção Científica Técnica Artística e Cultural/Sub-Grupo 2.1 - Produtividade de Pesquisa/C - Autoria de artigos completos publicados em anais de congresso em jornais e revistas de circulação nacional e internacional na sua área/20150530_dg.o 2015 paper 37}}

\newpage
\subsection{Autoria de artigos completos publicados em anais de congresso, em jornais e revistas de circulação nacional e internacional na sua área}
\label{conf:2015-sbsi-minicurso}
Esta subseção apresenta o comprovante da autoria de artigo completo publicado em anais do XI Simpósio Brasileiro de Sistemas de Informação (SBSI), , Sistemas de Informação: A Visão Sociotécnica da Computação\footnote{\url{http://www.portal.inf.ufg.br/sbsi2015/en/node/30}}.
\includepdf[pages=-, scale=1,pagecommand=\thispagestyle{empty}]{\detokenize{GRUPO 2 – Atividades - Produção Científica Técnica Artística e Cultural/Sub-Grupo 2.1 - Produtividade de Pesquisa/C - Autoria de artigos completos publicados em anais de congresso em jornais e revistas de circulação nacional e internacional na sua área/20150529_SBSI 2015 - Minicursos paper 140036 notification}}
\includepdf[pages=-, scale=1,pagecommand=\thispagestyle{empty}]{\detokenize{GRUPO 2 – Atividades - Produção Científica Técnica Artística e Cultural/Sub-Grupo 2.1 - Produtividade de Pesquisa/C - Autoria de artigos completos publicados em anais de congresso em jornais e revistas de circulação nacional e internacional na sua área/20150529_SBSI 2015 - Minicursos paper 140036}}

\newpage
\subsection{Autoria de artigos completos publicados em anais de congresso, em jornais e revistas de circulação nacional e internacional na sua área}
\label{conf:2015-sbsi-special-track}
Esta subseção apresenta o comprovante da autoria de artigo completo publicado em anais do XI Simpósio Brasileiro de Sistemas de Informação (SBSI), Government Information Systems (SIG) Special Track\footnote{\url{http://www.portal.inf.ufg.br/sbsi2015/index}}.
\includepdf[pages=-, scale=1,pagecommand=\thispagestyle{empty}]{\detokenize{GRUPO 2 – Atividades - Produção Científica Técnica Artística e Cultural/Sub-Grupo 2.1 - Produtividade de Pesquisa/C - Autoria de artigos completos publicados em anais de congresso em jornais e revistas de circulação nacional e internacional na sua área/20150529_SBSI 2015 - Special Track 4 - paper 139906 notification}}
\includepdf[pages=-, scale=1,pagecommand=\thispagestyle{empty}]{\detokenize{GRUPO 2 – Atividades - Produção Científica Técnica Artística e Cultural/Sub-Grupo 2.1 - Produtividade de Pesquisa/C - Autoria de artigos completos publicados em anais de congresso em jornais e revistas de circulação nacional e internacional na sua área/20150529_SBSI 2015 - Special Track 4 - paper 139906}}

\newpage
\subsection{Autoria de artigos completos publicados em anais de congresso, em jornais e revistas de circulação nacional e internacional na sua área}
\label{conf:2015-sbsi-main-track}
Esta subseção apresenta o comprovante da autoria de artigo completo publicado em anais do XI Simpósio Brasileiro de Sistemas de Informação (SBSI), Prediction Methods and Recommender Systems Track\footnote{\url{http://www.portal.inf.ufg.br/sbsi2015/index}}.
\includepdf[pages=-, scale=1,pagecommand=\thispagestyle{empty}]{\detokenize{GRUPO 2 – Atividades - Produção Científica Técnica Artística e Cultural/Sub-Grupo 2.1 - Produtividade de Pesquisa/C - Autoria de artigos completos publicados em anais de congresso em jornais e revistas de circulação nacional e internacional na sua área/20150529_SBSI 2015 - Main Track paper 139758 notification}}
\includepdf[pages=-, scale=1,pagecommand=\thispagestyle{empty}]{\detokenize{GRUPO 2 – Atividades - Produção Científica Técnica Artística e Cultural/Sub-Grupo 2.1 - Produtividade de Pesquisa/C - Autoria de artigos completos publicados em anais de congresso em jornais e revistas de circulação nacional e internacional na sua área/20150529_SBSI 2015 - Main Track paper 139758}}

\newpage
\subsection{Autoria de artigos completos publicados em anais de congresso, em jornais e revistas de circulação nacional e internacional na sua área}
\label{conf:2015-webist}
Esta subseção apresenta o comprovante da autoria de artigo completo publicado em proceedings do The 11th International Conference on Web Information Systems and Technologies (WEBIST)\footnote{\url{http://www.webist.org/?y=2015}}.
\includepdf[pages=-, scale=1,pagecommand=\thispagestyle{empty}]{\detokenize{GRUPO 2 – Atividades - Produção Científica Técnica Artística e Cultural/Sub-Grupo 2.1 - Produtividade de Pesquisa/C - Autoria de artigos completos publicados em anais de congresso em jornais e revistas de circulação nacional e internacional na sua área/20150522_WEBIST 2015 - Authors Notification}}
\includepdf[pages=-, scale=1,pagecommand=\thispagestyle{empty}]{\detokenize{GRUPO 2 – Atividades - Produção Científica Técnica Artística e Cultural/Sub-Grupo 2.1 - Produtividade de Pesquisa/C - Autoria de artigos completos publicados em anais de congresso em jornais e revistas de circulação nacional e internacional na sua área/20150522_WEBIST 2015 - Paper}}

\newpage
\subsection{Autoria de artigos completos publicados em anais de congresso, em jornais e revistas de circulação nacional e internacional na sua área}
\label{conf:2015-socm-www}
Esta subseção apresenta o comprovante da autoria de artigo completo publicado em proceedings do 24th International Conference on World Wide Web Companion (WWW'15 Companion)\footnote{\url{http://www.www2015.it/}}.
\includepdf[pages=-, scale=1,pagecommand=\thispagestyle{empty}]{\detokenize{GRUPO 2 – Atividades - Produção Científica Técnica Artística e Cultural/Sub-Grupo 2.1 - Produtividade de Pesquisa/C - Autoria de artigos completos publicados em anais de congresso em jornais e revistas de circulação nacional e internacional na sua área/20150522_SOCM2015 notification for paper 7}}
\includepdf[pages=-, scale=1,pagecommand=\thispagestyle{empty}]{\detokenize{GRUPO 2 – Atividades - Produção Científica Técnica Artística e Cultural/Sub-Grupo 2.1 - Produtividade de Pesquisa/C - Autoria de artigos completos publicados em anais de congresso em jornais e revistas de circulação nacional e internacional na sua área/20150522_SOCM2015 paper 7}}

\newpage
\subsection{Autoria de artigos completos publicados em anais de congresso, em jornais e revistas de circulação nacional e internacional na sua área}
\label{conf:2015-ccnc}
Esta subseção apresenta o comprovante da autoria de artigo completo publicado em proceedings do IEEE Consumer Communications \& Networking Conference (CCNC)\footnote{\url{http://ccnc2015.ieee-ccnc.org}}.
\includepdf[pages=-, scale=1,pagecommand=\thispagestyle{empty}]{\detokenize{GRUPO 2 – Atividades - Produção Científica Técnica Artística e Cultural/Sub-Grupo 2.1 - Produtividade de Pesquisa/C - Autoria de artigos completos publicados em anais de congresso em jornais e revistas de circulação nacional e internacional na sua área/20150112_CCNC2015 Your paper 1570043391 notification}}
\includepdf[pages=-, scale=1,pagecommand=\thispagestyle{empty}]{\detokenize{GRUPO 2 – Atividades - Produção Científica Técnica Artística e Cultural/Sub-Grupo 2.1 - Produtividade de Pesquisa/C - Autoria de artigos completos publicados em anais de congresso em jornais e revistas de circulação nacional e internacional na sua área/20150112_ CCNC2015_short-paper}}

\newpage
\subsection{Autoria de artigos completos publicados em anais de congresso, em jornais e revistas de circulação nacional e internacional na sua área}
\label{conf:2014-cbis}
Esta subseção apresenta o comprovante da autoria de artigo completo publicado em anais do XIV Congresso Brasileiro de Informática em Saúde (CBIS)\footnote{\url{http://www.sbis.org.br/eventos/scripts/programacao2.dll/grade4}}.
\includepdf[pages=-, scale=1,pagecommand=\thispagestyle{empty}]{\detokenize{GRUPO 2 – Atividades - Produção Científica Técnica Artística e Cultural/Sub-Grupo 2.1 - Produtividade de Pesquisa/C - Autoria de artigos completos publicados em anais de congresso em jornais e revistas de circulação nacional e internacional na sua área/20141210_CBIS - Carta de Aprovação Artigo 798}}
\includepdf[pages=-, scale=1,pagecommand=\thispagestyle{empty}]{\detokenize{GRUPO 2 – Atividades - Produção Científica Técnica Artística e Cultural/Sub-Grupo 2.1 - Produtividade de Pesquisa/C - Autoria de artigos completos publicados em anais de congresso em jornais e revistas de circulação nacional e internacional na sua área/20141210_CBIS - Artigo 798}}

\newpage
\subsection{Autoria de artigos completos publicados em anais de congresso, em jornais e revistas de circulação nacional e internacional na sua área}
\label{conf:2014-cbsoft-trilha-marco}
Esta subseção apresenta o comprovante da autoria de artigo completo publicado em anais da Trilha da Indústria do Congresso Brasileiro de Software: Teoria e Prática (CBSoft)\footnote{\url{http://www.ic.ufal.br/evento/cbsoft2014/menu/trilha-industria.html}}.
\includepdf[pages=-, scale=1,pagecommand=\thispagestyle{empty}]{\detokenize{GRUPO 2 – Atividades - Produção Científica Técnica Artística e Cultural/Sub-Grupo 2.1 - Produtividade de Pesquisa/C - Autoria de artigos completos publicados em anais de congresso em jornais e revistas de circulação nacional e internacional na sua área/20141003_CBSoft 2014 - Trilha da Indústria paper 131378 - notificacao}}
\includepdf[pages=-, scale=1,pagecommand=\thispagestyle{empty}]{\detokenize{GRUPO 2 – Atividades - Produção Científica Técnica Artística e Cultural/Sub-Grupo 2.1 - Produtividade de Pesquisa/C - Autoria de artigos completos publicados em anais de congresso em jornais e revistas de circulação nacional e internacional na sua área/20141003_CBSoft 2014 - Trilha da Indústria paper 131378}}

\newpage
\subsection{Autoria de artigos completos publicados em anais de congresso, em jornais e revistas de circulação nacional e internacional na sua área}
\label{conf:2014-cbsoft-trilha-fish}
Esta subseção apresenta o comprovante da autoria de artigo completo publicado em anais da Trilha da Indústria do Congresso Brasileiro de Software: Teoria e Prática (CBSoft)\footnote{\url{http://www.ic.ufal.br/evento/cbsoft2014/menu/trilha-industria.html}}.
\includepdf[pages=-, scale=1,pagecommand=\thispagestyle{empty}]{\detokenize{GRUPO 2 – Atividades - Produção Científica Técnica Artística e Cultural/Sub-Grupo 2.1 - Produtividade de Pesquisa/C - Autoria de artigos completos publicados em anais de congresso em jornais e revistas de circulação nacional e internacional na sua área/20141003_CBSoft 2014 - Trilha da Indústria paper 131369 - notificacao}}
\includepdf[pages=-, scale=1,pagecommand=\thispagestyle{empty}]{\detokenize{GRUPO 2 – Atividades - Produção Científica Técnica Artística e Cultural/Sub-Grupo 2.1 - Produtividade de Pesquisa/C - Autoria de artigos completos publicados em anais de congresso em jornais e revistas de circulação nacional e internacional na sua área/20141003_CBSoft 2014 - Trilha da Indústria paper 131369}}

\newpage
\subsection{Autoria de artigos completos publicados em anais de congresso, em jornais e revistas de circulação nacional e internacional na sua área}
\label{conf:2014-cbsoft-trilha-jean}
Esta subseção apresenta o comprovante da autoria de artigo completo publicado em anais da Trilha da Indústria do Congresso Brasileiro de Software: Teoria e Prática (CBSoft)\footnote{\url{http://www.ic.ufal.br/evento/cbsoft2014/menu/trilha-industria.html}}.
\includepdf[pages=-, scale=1,pagecommand=\thispagestyle{empty}]{\detokenize{GRUPO 2 – Atividades - Produção Científica Técnica Artística e Cultural/Sub-Grupo 2.1 - Produtividade de Pesquisa/C - Autoria de artigos completos publicados em anais de congresso em jornais e revistas de circulação nacional e internacional na sua área/20141003_CBSoft 2014 - Trilha da Indústria paper 131334 - notificacao}}
\includepdf[pages=-, scale=1,pagecommand=\thispagestyle{empty}]{\detokenize{GRUPO 2 – Atividades - Produção Científica Técnica Artística e Cultural/Sub-Grupo 2.1 - Produtividade de Pesquisa/C - Autoria de artigos completos publicados em anais de congresso em jornais e revistas de circulação nacional e internacional na sua área/20141003_CBSoft 2014 - Trilha da Indústria paper 131334}}

\newpage
\subsection{Autoria de artigos completos publicados em anais de congresso, em jornais e revistas de circulação nacional e internacional na sua área}
\label{conf:2014-cbsoft-trilha-kid}
Esta subseção apresenta o comprovante da autoria de artigo completo publicado em anais da Trilha da Indústria do Congresso Brasileiro de Software: Teoria e Prática (CBSoft)\footnote{\url{http://www.ic.ufal.br/evento/cbsoft2014/menu/trilha-industria.html}}.
\includepdf[pages=-, scale=1,pagecommand=\thispagestyle{empty}]{\detokenize{GRUPO 2 – Atividades - Produção Científica Técnica Artística e Cultural/Sub-Grupo 2.1 - Produtividade de Pesquisa/C - Autoria de artigos completos publicados em anais de congresso em jornais e revistas de circulação nacional e internacional na sua área/20141003_CBSoft 2014 - Trilha da Indústria paper 131194 - notificacao}}
\includepdf[pages=-, scale=1,pagecommand=\thispagestyle{empty}]{\detokenize{GRUPO 2 – Atividades - Produção Científica Técnica Artística e Cultural/Sub-Grupo 2.1 - Produtividade de Pesquisa/C - Autoria de artigos completos publicados em anais de congresso em jornais e revistas de circulação nacional e internacional na sua área/20141003_CBSoft 2014 - Trilha da Indústria paper 131194}}

\newpage
\subsection{Autoria de artigos completos publicados em anais de congresso, em jornais e revistas de circulação nacional e internacional na sua área}
\label{conf:2014-cbsoft-wtdsoft}
Esta subseção apresenta o comprovante da autoria de artigo completo publicado em anais do IV Workshop de Teses e Dissertações do CBSoft (WTDSoft 2014)\footnote{\url{http://www.ic.ufal.br/evento/cbsoft2014/menu/wtdsoft-br.html}} - Congresso Brasileiro de Software: Teoria e Prática (CBSoft)\footnote{\url{http://www.ic.ufal.br/evento/cbsoft2014/}}.
\includepdf[pages=-, scale=1,pagecommand=\thispagestyle{empty}]{\detokenize{GRUPO 2 – Atividades - Produção Científica Técnica Artística e Cultural/Sub-Grupo 2.1 - Produtividade de Pesquisa/C - Autoria de artigos completos publicados em anais de congresso em jornais e revistas de circulação nacional e internacional na sua área/20141003_CBSoft 2014 - WTDSoft 2014 paper 129447 - notification}}
\includepdf[pages=-, scale=1,pagecommand=\thispagestyle{empty}]{\detokenize{GRUPO 2 – Atividades - Produção Científica Técnica Artística e Cultural/Sub-Grupo 2.1 - Produtividade de Pesquisa/C - Autoria de artigos completos publicados em anais de congresso em jornais e revistas de circulação nacional e internacional na sua área/20141003_CBSoft 2014 - WTDSoft 2014 paper 129447}}

\newpage
\subsection{Autoria de artigos completos publicados em anais de congresso, em jornais e revistas de circulação nacional e internacional na sua área}
\label{conf:2014-JISIC}
Esta subseção apresenta o comprovante da autoria de artigo completo publicado em proceedings do 2014 IEEE Intelligence \& Security Informatics Conference (IEEE JISIC)\footnote{\url{http://www.eisic.org/eisic2014/}}.
\includepdf[pages=-, scale=1,pagecommand=\thispagestyle{empty}]{\detokenize{GRUPO 2 – Atividades - Produção Científica Técnica Artística e Cultural/Sub-Grupo 2.1 - Produtividade de Pesquisa/C - Autoria de artigos completos publicados em anais de congresso em jornais e revistas de circulação nacional e internacional na sua área/20140909_ISI-EISIC-2014 notification for paper 10}}
\includepdf[pages=-, scale=1,pagecommand=\thispagestyle{empty}]{\detokenize{GRUPO 2 – Atividades - Produção Científica Técnica Artística e Cultural/Sub-Grupo 2.1 - Produtividade de Pesquisa/C - Autoria de artigos completos publicados em anais de congresso em jornais e revistas de circulação nacional e internacional na sua área/20140909_ISI-EISIC-2014 paper 10}}

\newpage
\subsection{Autoria de artigos completos publicados em anais de congresso, em jornais e revistas de circulação nacional e internacional na sua área}
\label{conf:2014-cloud-wip}
Esta subseção apresenta o comprovante da autoria de artigo completo publicado em proceedings do 2014 IEEE International Conference on Cloud Computing (IEEE CLOUD)\footnote{\url{http://www.thecloudcomputing.org/2014/}}.
\includepdf[pages=-, scale=1,pagecommand=\thispagestyle{empty}]{\detokenize{GRUPO 2 – Atividades - Produção Científica Técnica Artística e Cultural/Sub-Grupo 2.1 - Produtividade de Pesquisa/C - Autoria de artigos completos publicados em anais de congresso em jornais e revistas de circulação nacional e internacional na sua área/20140702_CLOUD WIP Status of paper 6384}}
\includepdf[pages=-, scale=1,pagecommand=\thispagestyle{empty}]{\detokenize{GRUPO 2 – Atividades - Produção Científica Técnica Artística e Cultural/Sub-Grupo 2.1 - Produtividade de Pesquisa/C - Autoria de artigos completos publicados em anais de congresso em jornais e revistas de circulação nacional e internacional na sua área/20140702_CLOUD WIP paper 6384}}

%---
\newpage
\subsection{Arbitragem de Artigos Técnico-Científicos Nacionais e Internacionais na sua área de atuação}
\label{reviewer:2016-tse}
Esta subseção apresenta o comprovante de arbitragem de artigos do periódico IEEE’s Transactions on Software Engineering, editora IEEE Computer Society, ISSN: 0098-5589.
\includepdf[pages=-, scale=1,pagecommand=\thispagestyle{empty}]{\detokenize{GRUPO 2 – Atividades - Produção Científica Técnica Artística e Cultural/Sub-Grupo 2.1 - Produtividade de Pesquisa/D - Arbitragem de artigos técnico-científicos nacionais e internacionais na sua área de atuação/20160115_TSE-2015-12-0405 now in your Reviewer Center}}
\includepdf[pages=-, scale=1,pagecommand=\thispagestyle{empty}]{\detokenize{GRUPO 2 – Atividades - Produção Científica Técnica Artística e Cultural/Sub-Grupo 2.1 - Produtividade de Pesquisa/D - Arbitragem de artigos técnico-científicos nacionais e internacionais na sua área de atuação/20160307_Decision - TSE-2015-12-0405}}

\newpage
\subsection{Arbitragem de Artigos Técnico-Científicos Nacionais e Internacionais na sua área de atuação}
\label{reviewer:2015-tsc}
Esta subseção apresenta o comprovante de arbitragem de artigos do periódico IEEE Transactions on Services Computing, editora IEEE Computer Society, ISSN: 1939-1374.
\includepdf[pages=-, scale=1,pagecommand=\thispagestyle{empty}]{\detokenize{GRUPO 2 – Atividades - Produção Científica Técnica Artística e Cultural/Sub-Grupo 2.1 - Produtividade de Pesquisa/D - Arbitragem de artigos técnico-científicos nacionais e internacionais na sua área de atuação/2015_TSCSI-2015-06-0260 Now in Your Reviewer Center}}
\includepdf[pages=-, scale=1,pagecommand=\thispagestyle{empty}]{\detokenize{GRUPO 2 – Atividades - Produção Científica Técnica Artística e Cultural/Sub-Grupo 2.1 - Produtividade de Pesquisa/D - Arbitragem de artigos técnico-científicos nacionais e internacionais na sua área de atuação/2015_TSCSI-2015-06-0260 - Decision and Thank You}}

\newpage
\subsection{Arbitragem de Artigos Técnico-Científicos Nacionais e Internacionais na sua área de atuação}
\label{reviewer:2015-aci}
Esta subseção apresenta o comprovante de arbitragem de artigos do periódico Applied Computing and Informatics, editora Elsevier B.V., ISSN: 2210-8327.
\includepdf[pages=-, scale=1,pagecommand=\thispagestyle{empty}]{\detokenize{GRUPO 2 – Atividades - Produção Científica Técnica Artística e Cultural/Sub-Grupo 2.1 - Produtividade de Pesquisa/D - Arbitragem de artigos técnico-científicos nacionais e internacionais na sua área de atuação/2015_Applied Computing and Informatics}}
\includepdf[pages=-, scale=1,pagecommand=\thispagestyle{empty}]{\detokenize{GRUPO 2 – Atividades - Produção Científica Técnica Artística e Cultural/Sub-Grupo 2.1 - Produtividade de Pesquisa/D - Arbitragem de artigos técnico-científicos nacionais e internacionais na sua área de atuação/2015_Reviewer Invitation for ACI-D-15-00139}}
\includepdf[pages=-, scale=1,pagecommand=\thispagestyle{empty}]{\detokenize{GRUPO 2 – Atividades - Produção Científica Técnica Artística e Cultural/Sub-Grupo 2.1 - Produtividade de Pesquisa/D - Arbitragem de artigos técnico-científicos nacionais e internacionais na sua área de atuação/2015_Thank you for the review of ACI-D-15-00139}}

\newpage
\subsection{Arbitragem de Artigos Técnico-Científicos Nacionais e Internacionais na sua área de atuação}
\label{reviewer:2015-jserd}
Esta subseção apresenta o comprovante de arbitragem de artigos do periódico Journal of Software Engineering Research and Development Research, editora Springer, ISSN: 2195-1721.
\includepdf[pages=-, scale=1,pagecommand=\thispagestyle{empty}]{\detokenize{GRUPO 2 – Atividades - Produção Científica Técnica Artística e Cultural/Sub-Grupo 2.1 - Produtividade de Pesquisa/D - Arbitragem de artigos técnico-científicos nacionais e internacionais na sua área de atuação/2015_Review of Journal of Software Engineering Research and Development manuscript}}
\includepdf[pages=-, scale=1,pagecommand=\thispagestyle{empty}]{\detokenize{GRUPO 2 – Atividades - Produção Científica Técnica Artística e Cultural/Sub-Grupo 2.1 - Produtividade de Pesquisa/D - Arbitragem de artigos técnico-científicos nacionais e internacionais na sua área de atuação/2015_Your review for Journal of Software Engineering Research and Development}}

\newpage
\subsection{Arbitragem de Artigos Técnico-Científicos Nacionais e Internacionais na sua área de atuação}
\label{reviewer:2015-sbqs}
Esta subseção apresenta o comprovante de arbitragem de Trabalhos Técnicos do Simpósio Brasileiro de Qualidade de Software (SBQS'2015).
\includepdf[pages=-, scale=1,pagecommand=\thispagestyle{empty}]{\detokenize{GRUPO 2 – Atividades - Produção Científica Técnica Artística e Cultural/Sub-Grupo 2.1 - Produtividade de Pesquisa/D - Arbitragem de artigos técnico-científicos nacionais e internacionais na sua área de atuação/20150817_SBQS 2015 - Trabalhos Técnicos (Technical Papers)_ TPC invitation}}
\includepdf[pages=-, scale=1,pagecommand=\thispagestyle{empty}]{\detokenize{GRUPO 2 – Atividades - Produção Científica Técnica Artística e Cultural/Sub-Grupo 2.1 - Produtividade de Pesquisa/D - Arbitragem de artigos técnico-científicos nacionais e internacionais na sua área de atuação/20150817_SBQS 2015 - Trabalhos Técnicos (Technical Papers)_ Home (Vinicius Garcia)_JEMS}}
\includepdf[pages=-, scale=1,pagecommand=\thispagestyle{empty}]{\detokenize{GRUPO 2 – Atividades - Produção Científica Técnica Artística e Cultural/Sub-Grupo 2.1 - Produtividade de Pesquisa/D - Arbitragem de artigos técnico-científicos nacionais e internacionais na sua área de atuação/20150817_SBQS 2015 Page Committee}}

\newpage
\subsection{Arbitragem de Artigos Técnico-Científicos Nacionais e Internacionais na sua área de atuação}
\label{reviewer:2015-elaes}
Esta subseção apresenta o comprovante de arbitragem de Trabalhos Técnicos do The Second Latin-American School on Software Engineering (ELA-ES 2015).
\includepdf[pages=-, scale=1,pagecommand=\thispagestyle{empty}]{\detokenize{GRUPO 2 – Atividades - Produção Científica Técnica Artística e Cultural/Sub-Grupo 2.1 - Produtividade de Pesquisa/D - Arbitragem de artigos técnico-científicos nacionais e internacionais na sua área de atuação/20150703_ELA-ES 2015 program committee}}
\includepdf[pages=-, scale=1,pagecommand=\thispagestyle{empty}]{\detokenize{GRUPO 2 – Atividades - Produção Científica Técnica Artística e Cultural/Sub-Grupo 2.1 - Produtividade de Pesquisa/D - Arbitragem de artigos técnico-científicos nacionais e internacionais na sua área de atuação/20150703_ELA-ES 2015 (PC member)}}
\includepdf[pages=-, scale=1,pagecommand=\thispagestyle{empty}]{\detokenize{GRUPO 2 – Atividades - Produção Científica Técnica Artística e Cultural/Sub-Grupo 2.1 - Produtividade de Pesquisa/D - Arbitragem de artigos técnico-científicos nacionais e internacionais na sua área de atuação/20150703_ELA-ES 2015 Organisation Page}}

\newpage
\subsection{Arbitragem de Artigos Técnico-Científicos Nacionais e Internacionais na sua área de atuação}
\label{reviewer:2015-fe20}
Esta subseção apresenta o comprovante de arbitragem de Trabalhos Técnicos do The 1st International Workshop on Towards Fully Enterprise 2.0 (FE 2.0).
\includepdf[pages=-, scale=1,pagecommand=\thispagestyle{empty}]{\detokenize{GRUPO 2 – Atividades - Produção Científica Técnica Artística e Cultural/Sub-Grupo 2.1 - Produtividade de Pesquisa/D - Arbitragem de artigos técnico-científicos nacionais e internacionais na sua área de atuação/20151110_FE2.0_invitation to PC}}
\includepdf[pages=-, scale=1,pagecommand=\thispagestyle{empty}]{\detokenize{GRUPO 2 – Atividades - Produção Científica Técnica Artística e Cultural/Sub-Grupo 2.1 - Produtividade de Pesquisa/D - Arbitragem de artigos técnico-científicos nacionais e internacionais na sua área de atuação/20151110_FE2.0 (PC member)}}
\includepdf[pages=-, scale=1,pagecommand=\thispagestyle{empty}]{\detokenize{GRUPO 2 – Atividades - Produção Científica Técnica Artística e Cultural/Sub-Grupo 2.1 - Produtividade de Pesquisa/D - Arbitragem de artigos técnico-científicos nacionais e internacionais na sua área de atuação/20151110_FE 2.0 site}}

\newpage
\subsection{Arbitragem de Artigos Técnico-Científicos Nacionais e Internacionais na sua área de atuação}
\label{reviewer:2015-sbcars}
Esta subseção apresenta o comprovante de arbitragem de Trabalhos Técnicos do 9th Brazilian Symposium on Software Components, Architectures and Reuse (SBCARS 2015).
\includepdf[pages=-, scale=1,pagecommand=\thispagestyle{empty}]{\detokenize{GRUPO 2 – Atividades - Produção Científica Técnica Artística e Cultural/Sub-Grupo 2.1 - Produtividade de Pesquisa/D - Arbitragem de artigos técnico-científicos nacionais e internacionais na sua área de atuação/20150921_SBCARS 2015 invitation program committee}}
\includepdf[pages=-, scale=1,pagecommand=\thispagestyle{empty}]{\detokenize{GRUPO 2 – Atividades - Produção Científica Técnica Artística e Cultural/Sub-Grupo 2.1 - Produtividade de Pesquisa/D - Arbitragem de artigos técnico-científicos nacionais e internacionais na sua área de atuação/20150921_SBCARS 2015 (PC member)}}
\includepdf[pages=-, scale=1,pagecommand=\thispagestyle{empty}]{\detokenize{GRUPO 2 – Atividades - Produção Científica Técnica Artística e Cultural/Sub-Grupo 2.1 - Produtividade de Pesquisa/D - Arbitragem de artigos técnico-científicos nacionais e internacionais na sua área de atuação/20150921_SBCARS 2015 comite}}

\newpage
\subsection{Arbitragem de Artigos Técnico-Científicos Nacionais e Internacionais na sua área de atuação}
\label{reviewer:2015-fie}
Esta subseção apresenta o comprovante de arbitragem de Trabalhos Técnicos do 45th IEEE Frontiers in Education (FIE'2015).
\includepdf[pages=-, scale=1,pagecommand=\thispagestyle{empty}]{\detokenize{GRUPO 2 – Atividades - Produção Científica Técnica Artística e Cultural/Sub-Grupo 2.1 - Produtividade de Pesquisa/D - Arbitragem de artigos técnico-científicos nacionais e internacionais na sua área de atuação/20151021_FIE 2015 Invitation to serve as a reviewer}}
\includepdf[pages=-, scale=1,pagecommand=\thispagestyle{empty}]{\detokenize{GRUPO 2 – Atividades - Produção Científica Técnica Artística e Cultural/Sub-Grupo 2.1 - Produtividade de Pesquisa/D - Arbitragem de artigos técnico-científicos nacionais e internacionais na sua área de atuação/20151021_FIE 2015 My Reviews}}

\newpage
\subsection{Arbitragem de Artigos Técnico-Científicos Nacionais e Internacionais na sua área de atuação}
\label{reviewer:2015-csbc-semish}
Esta subseção apresenta o comprovante de arbitragem de Trabalhos Técnicos do 42º Seminário Integrado de Software e Hardware (SEMISH) do Congresso da Sociedade Brasileira de Computação (CSBC).
\includepdf[pages=-, scale=1,pagecommand=\thispagestyle{empty}]{\detokenize{GRUPO 2 – Atividades - Produção Científica Técnica Artística e Cultural/Sub-Grupo 2.1 - Produtividade de Pesquisa/D - Arbitragem de artigos técnico-científicos nacionais e internacionais na sua área de atuação/20150720_CSBC 2015 - SEMISH_ TPC invitation}}
\includepdf[pages=-, scale=1,pagecommand=\thispagestyle{empty}]{\detokenize{GRUPO 2 – Atividades - Produção Científica Técnica Artística e Cultural/Sub-Grupo 2.1 - Produtividade de Pesquisa/D - Arbitragem de artigos técnico-científicos nacionais e internacionais na sua área de atuação/20150720_JEMS_CSBC 2015 - SEMISH_ Home (Vinicius Garcia)}}

\newpage
\subsection{Arbitragem de Artigos Técnico-Científicos Nacionais e Internacionais na sua área de atuação}
\label{reviewer:2015-smc}
Esta subseção apresenta o comprovante de arbitragem de Trabalhos Técnicos do Systems Science \& Engineering Special Session of the IEEE International Conference on Systems, Man, and Cybernetics (SMC'2015).
\includepdf[pages=-, scale=1,pagecommand=\thispagestyle{empty}]{\detokenize{GRUPO 2 – Atividades - Produção Científica Técnica Artística e Cultural/Sub-Grupo 2.1 - Produtividade de Pesquisa/D - Arbitragem de artigos técnico-científicos nacionais e internacionais na sua área de atuação/20151009_SMC 2015 invitation for Technical Reviewer for Special Session on _Cyber-Physical Clouds_ at IEEE SMC2015}}
\includepdf[pages=-, scale=1,pagecommand=\thispagestyle{empty}]{\detokenize{GRUPO 2 – Atividades - Produção Científica Técnica Artística e Cultural/Sub-Grupo 2.1 - Produtividade de Pesquisa/D - Arbitragem de artigos técnico-científicos nacionais e internacionais na sua área de atuação/20151009_SMC 2015 reviewer options - ConfDriver}}

\newpage
\subsection{Arbitragem de Artigos Técnico-Científicos Nacionais e Internacionais na sua área de atuação}
\label{reviewer:2015-sbes-inovadoras}
Esta subseção apresenta o comprovante de arbitragem de Trabalhos Técnicos do Simpósio Brasileiro de Engenharia de Software (SBES'2015), Trilha Ideias Inovadoras.
\includepdf[pages=-, scale=1,pagecommand=\thispagestyle{empty}]{\detokenize{GRUPO 2 – Atividades - Produção Científica Técnica Artística e Cultural/Sub-Grupo 2.1 - Produtividade de Pesquisa/D - Arbitragem de artigos técnico-científicos nacionais e internacionais na sua área de atuação/20150921_CBSoft 2015 - SBES 2015 - Insightful Ideas_ Review invitation}}
\includepdf[pages=-, scale=1,pagecommand=\thispagestyle{empty}]{\detokenize{GRUPO 2 – Atividades - Produção Científica Técnica Artística e Cultural/Sub-Grupo 2.1 - Produtividade de Pesquisa/D - Arbitragem de artigos técnico-científicos nacionais e internacionais na sua área de atuação/20150921_JEMS_CBSoft 2015 - SBES 2015 - Insightful Ideas_ Home (Vinicius Garcia)}}

\newpage
\subsection{Arbitragem de Artigos Técnico-Científicos Nacionais e Internacionais na sua área de atuação}
\label{reviewer:2015-cbsoft-industry}
Esta subseção apresenta o comprovante de arbitragem de Trabalhos Técnicos do Brazilian Conference on Software: Theory and Practice (CBSoft'2015) - Industry Track.
\includepdf[pages=-, scale=1,pagecommand=\thispagestyle{empty}]{\detokenize{GRUPO 2 – Atividades - Produção Científica Técnica Artística e Cultural/Sub-Grupo 2.1 - Produtividade de Pesquisa/D - Arbitragem de artigos técnico-científicos nacionais e internacionais na sua área de atuação/20150921_CBSoft 2015 - Industry Track_ TPC invitation}}
\includepdf[pages=-, scale=1,pagecommand=\thispagestyle{empty}]{\detokenize{GRUPO 2 – Atividades - Produção Científica Técnica Artística e Cultural/Sub-Grupo 2.1 - Produtividade de Pesquisa/D - Arbitragem de artigos técnico-científicos nacionais e internacionais na sua área de atuação/20150921_JEMS_CBSoft 2015 - Industry Track_ Home (Vinicius Garcia)}}

\newpage
\subsection{Arbitragem de Artigos Técnico-Científicos Nacionais e Internacionais na sua área de atuação}
\label{reviewer:2015-wbma}
Esta subseção apresenta o comprovante de arbitragem de Trabalhos Técnicos do 6th Brazilian Workshop on Agile Methods (WBMA 2015).
\includepdf[pages=-, scale=1,pagecommand=\thispagestyle{empty}]{\detokenize{GRUPO 2 – Atividades - Produção Científica Técnica Artística e Cultural/Sub-Grupo 2.1 - Produtividade de Pesquisa/D - Arbitragem de artigos técnico-científicos nacionais e internacionais na sua área de atuação/20151021_WBMA 2015 program committee}}
\includepdf[pages=-, scale=1,pagecommand=\thispagestyle{empty}]{\detokenize{GRUPO 2 – Atividades - Produção Científica Técnica Artística e Cultural/Sub-Grupo 2.1 - Produtividade de Pesquisa/D - Arbitragem de artigos técnico-científicos nacionais e internacionais na sua área de atuação/20151021_WBMA 2015 (PC member of Short Paper)}}
\includepdf[pages=-, scale=1,pagecommand=\thispagestyle{empty}]{\detokenize{GRUPO 2 – Atividades - Produção Científica Técnica Artística e Cultural/Sub-Grupo 2.1 - Produtividade de Pesquisa/D - Arbitragem de artigos técnico-científicos nacionais e internacionais na sua área de atuação/20151021_WBMA 2015 (PC member of Full Paper)}}
\includepdf[pages=-, scale=1,pagecommand=\thispagestyle{empty}]{\detokenize{GRUPO 2 – Atividades - Produção Científica Técnica Artística e Cultural/Sub-Grupo 2.1 - Produtividade de Pesquisa/D - Arbitragem de artigos técnico-científicos nacionais e internacionais na sua área de atuação/20151021_WBMA - Agile Brazil Site}}

\newpage
\subsection{Arbitragem de Artigos Técnico-Científicos Nacionais e Internacionais na sua área de atuação}
\label{reviewer:2015-webmedia}
Esta subseção apresenta o comprovante de arbitragem de Trabalhos Técnicos do XXI Brazilian Symposium on Multimedia (Webmedia'2015), Full and Short Papers Tracks.
\includepdf[pages=-, scale=1,pagecommand=\thispagestyle{empty}]{\detokenize{GRUPO 2 – Atividades - Produção Científica Técnica Artística e Cultural/Sub-Grupo 2.1 - Produtividade de Pesquisa/D - Arbitragem de artigos técnico-científicos nacionais e internacionais na sua área de atuação/20151027_Invitation to WebMedia 2015 Program Committee (PC)}}
\includepdf[pages=-, scale=1,pagecommand=\thispagestyle{empty}]{\detokenize{GRUPO 2 – Atividades - Produção Científica Técnica Artística e Cultural/Sub-Grupo 2.1 - Produtividade de Pesquisa/D - Arbitragem de artigos técnico-científicos nacionais e internacionais na sua área de atuação/20151027_WebMedia 2015 (PC member of Full Papers - WebMedia 2015)}}
\includepdf[pages=-, scale=1,pagecommand=\thispagestyle{empty}]{\detokenize{GRUPO 2 – Atividades - Produção Científica Técnica Artística e Cultural/Sub-Grupo 2.1 - Produtividade de Pesquisa/D - Arbitragem de artigos técnico-científicos nacionais e internacionais na sua área de atuação/20151027_WebMedia 2015 (PC member of Short Papers - WebMedia 2015)}}
\includepdf[pages=-, scale=1,pagecommand=\thispagestyle{empty}]{\detokenize{GRUPO 2 – Atividades - Produção Científica Técnica Artística e Cultural/Sub-Grupo 2.1 - Produtividade de Pesquisa/D - Arbitragem de artigos técnico-científicos nacionais e internacionais na sua área de atuação/20151027_WebMedia 2015 Committees}}

\newpage
\subsection{Arbitragem de Artigos Técnico-Científicos Nacionais e Internacionais na sua área de atuação}
\label{reviewer:2015-icsea}
Esta subseção apresenta o comprovante de arbitragem de Trabalhos Técnicos do Tenth International Conference on Software Engineering Advances (ICSEA 2015).
\includepdf[pages=-, scale=1,pagecommand=\thispagestyle{empty}]{\detokenize{GRUPO 2 – Atividades - Produção Científica Técnica Artística e Cultural/Sub-Grupo 2.1 - Produtividade de Pesquisa/D - Arbitragem de artigos técnico-científicos nacionais e internacionais na sua área de atuação/20151115_ICSEA 2015_ Reminder to review contribution 10186}}
\includepdf[pages=-, scale=1,pagecommand=\thispagestyle{empty}]{\detokenize{GRUPO 2 – Atividades - Produção Científica Técnica Artística e Cultural/Sub-Grupo 2.1 - Produtividade de Pesquisa/D - Arbitragem de artigos técnico-científicos nacionais e internacionais na sua área de atuação/20151115_ICSEA 2015_ Request to review contribution 10269}}
\includepdf[pages=-, scale=1,pagecommand=\thispagestyle{empty}]{\detokenize{GRUPO 2 – Atividades - Produção Científica Técnica Artística e Cultural/Sub-Grupo 2.1 - Produtividade de Pesquisa/D - Arbitragem de artigos técnico-científicos nacionais e internacionais na sua área de atuação/20151115_ICSEA 2015 Committees}}

\newpage
\subsection{Arbitragem de Artigos Técnico-Científicos Nacionais e Internacionais na sua área de atuação}
\label{reviewer:2015-scctsa}
Esta subseção apresenta o comprovante de arbitragem de artigos técnico-científicos no 2nd International Workshop on Smart City Clouds: Technologies, Systems and Applications (SCCTSA 2015).
\includepdf[pages=-, scale=1,pagecommand=\thispagestyle{empty}]{\detokenize{GRUPO 2 – Atividades - Produção Científica Técnica Artística e Cultural/Sub-Grupo 2.1 - Produtividade de Pesquisa/D - Arbitragem de artigos técnico-científicos nacionais e internacionais na sua área de atuação/20150818_SCCTSA 2015 program committee}}

\newpage
\subsection{Arbitragem de Artigos Técnico-Científicos Nacionais e Internacionais na sua área de atuação}
\label{reviewer:2015-encoinfo}
Esta subseção apresenta o comprovante de arbitragem de artigos técnico-científicos no XVII Encontro de Computação e Informática do Tocantins (ENCOINFO'2015).
\includepdf[pages=-, scale=1,pagecommand=\thispagestyle{empty}]{\detokenize{GRUPO 2 – Atividades - Produção Científica Técnica Artística e Cultural/Sub-Grupo 2.1 - Produtividade de Pesquisa/D - Arbitragem de artigos técnico-científicos nacionais e internacionais na sua área de atuação/20151019 ENCOINFO comite}}

\newpage
\subsection{Arbitragem de Artigos Técnico-Científicos Nacionais e Internacionais na sua área de atuação}
\label{reviewer:2014-claio}
Esta subseção apresenta o comprovante de arbitragem de artigos técnico-científicos no XVII Latin-Iberian-American Conference on Operations Research (CLAIO'2014).
\includepdf[pages=-, scale=1,pagecommand=\thispagestyle{empty}]{\detokenize{GRUPO 2 – Atividades - Produção Científica Técnica Artística e Cultural/Sub-Grupo 2.1 - Produtividade de Pesquisa/D - Arbitragem de artigos técnico-científicos nacionais e internacionais na sua área de atuação/20141010_Claio (Mexico)}}

\newpage
\subsection{Arbitragem de Artigos Técnico-Científicos Nacionais e Internacionais na sua área de atuação}
\label{reviewer:2014-csbc-semish}
Esta subseção apresenta o comprovante de arbitragem de artigos técnico-científicos no XLI Seminário Integrado de Software e Hardware (SEMISH), Congresso da Sociedade Brasileira de Computação (CSBC).
\includepdf[pages=-, scale=1,pagecommand=\thispagestyle{empty}]{\detokenize{GRUPO 2 – Atividades - Produção Científica Técnica Artística e Cultural/Sub-Grupo 2.1 - Produtividade de Pesquisa/D - Arbitragem de artigos técnico-científicos nacionais e internacionais na sua área de atuação/20140721_CSBC 2014 - SEMISH_ TPC invitation}}
\includepdf[pages=-, scale=1,pagecommand=\thispagestyle{empty}]{\detokenize{GRUPO 2 – Atividades - Produção Científica Técnica Artística e Cultural/Sub-Grupo 2.1 - Produtividade de Pesquisa/D - Arbitragem de artigos técnico-científicos nacionais e internacionais na sua área de atuação/20140721_JEMS_CSBC 2014 - SEMISH_ Home (Vinicius Garcia)}}
\includepdf[pages=-, scale=1,pagecommand=\thispagestyle{empty}]{\detokenize{GRUPO 2 – Atividades - Produção Científica Técnica Artística e Cultural/Sub-Grupo 2.1 - Produtividade de Pesquisa/D - Arbitragem de artigos técnico-científicos nacionais e internacionais na sua área de atuação/20140721_SEMISH Site}}

\newpage
\subsection{Arbitragem de Artigos Técnico-Científicos Nacionais e Internacionais na sua área de atuação}
\label{reviewer:2014-webmedia}
Esta subseção apresenta o comprovante de arbitragem de artigos técnico-científicos no XX Brazilian Symposium on Multimedia and the Web (WebMedia'2014), Full and Short Papers Track.
\includepdf[pages=-, scale=1,pagecommand=\thispagestyle{empty}]{\detokenize{GRUPO 2 – Atividades - Produção Científica Técnica Artística e Cultural/Sub-Grupo 2.1 - Produtividade de Pesquisa/D - Arbitragem de artigos técnico-científicos nacionais e internacionais na sua área de atuação/20141118_WebMedia2014 - Notification for PC Members}}
\includepdf[pages=-, scale=1,pagecommand=\thispagestyle{empty}]{\detokenize{GRUPO 2 – Atividades - Produção Científica Técnica Artística e Cultural/Sub-Grupo 2.1 - Produtividade de Pesquisa/D - Arbitragem de artigos técnico-científicos nacionais e internacionais na sua área de atuação/20141118_Webmedia 2014 - Short Papers Track}}
\includepdf[pages=-, scale=1,pagecommand=\thispagestyle{empty}]{\detokenize{GRUPO 2 – Atividades - Produção Científica Técnica Artística e Cultural/Sub-Grupo 2.1 - Produtividade de Pesquisa/D - Arbitragem de artigos técnico-científicos nacionais e internacionais na sua área de atuação/20141118_WebMedia2014 (PC member of WebMedia2014)}}
\includepdf[pages=-, scale=1,pagecommand=\thispagestyle{empty}]{\detokenize{GRUPO 2 – Atividades - Produção Científica Técnica Artística e Cultural/Sub-Grupo 2.1 - Produtividade de Pesquisa/D - Arbitragem de artigos técnico-científicos nacionais e internacionais na sua área de atuação/20141118_WebMedia2014 (PC member of WebMedia2014 Short Papers (Artigos Curtos))}}

\newpage
\subsection{Arbitragem de Artigos Técnico-Científicos Nacionais e Internacionais na sua área de atuação}
\label{reviewer:2014-webmedia-wtic}
Esta subseção apresenta o comprovante de arbitragem de artigos técnico-científicos no Workshop on Undergraduate Research Work (WTIC) / XX Brazilian Symposium on Multimedia and the Web (WebMedia'2014).
\includepdf[pages=-, scale=1,pagecommand=\thispagestyle{empty}]{\detokenize{GRUPO 2 – Atividades - Produção Científica Técnica Artística e Cultural/Sub-Grupo 2.1 - Produtividade de Pesquisa/D - Arbitragem de artigos técnico-científicos nacionais e internacionais na sua área de atuação/20141118_invitation to WTIC_WebMedia 2014 program committee}}
\includepdf[pages=-, scale=1,pagecommand=\thispagestyle{empty}]{\detokenize{GRUPO 2 – Atividades - Produção Científica Técnica Artística e Cultural/Sub-Grupo 2.1 - Produtividade de Pesquisa/D - Arbitragem de artigos técnico-científicos nacionais e internacionais na sua área de atuação/20141118_WebMedia2014 (PC member of Workshop on Undergraduate Research Work (WTIC))}}

\newpage
\subsection{Arbitragem de Artigos Técnico-Científicos Nacionais e Internacionais na sua área de atuação}
\label{reviewer:2014-icsea}
Esta subseção apresenta o comprovante de arbitragem de artigos técnico-científicos no The Ninth International Conference on Software Engineering Advances (ICSEA'2014).
\includepdf[pages=-, scale=1,pagecommand=\thispagestyle{empty}]{\detokenize{GRUPO 2 – Atividades - Produção Científica Técnica Artística e Cultural/Sub-Grupo 2.1 - Produtividade de Pesquisa/D - Arbitragem de artigos técnico-científicos nacionais e internacionais na sua área de atuação/20141012_ICSEA 2014_ Request to review contribution 10286}}
\includepdf[pages=-, scale=1,pagecommand=\thispagestyle{empty}]{\detokenize{GRUPO 2 – Atividades - Produção Científica Técnica Artística e Cultural/Sub-Grupo 2.1 - Produtividade de Pesquisa/D - Arbitragem de artigos técnico-científicos nacionais e internacionais na sua área de atuação/20141012_ICSEA 2014 Committees}}

\newpage
\subsection{Arbitragem de Artigos Técnico-Científicos Nacionais e Internacionais na sua área de atuação}
\label{reviewer:2014-industria-cbsoft}
Esta subseção apresenta o comprovante de arbitragem de artigos técnico-científicos na Trilha da Indústria do Congresso Brasileiro de Software: Teoria e Prática (CBSoft'2014).
\includepdf[pages=-, scale=1,pagecommand=\thispagestyle{empty}]{\detokenize{GRUPO 2 – Atividades - Produção Científica Técnica Artística e Cultural/Sub-Grupo 2.1 - Produtividade de Pesquisa/D - Arbitragem de artigos técnico-científicos nacionais e internacionais na sua área de atuação/20140928_CBSoft 2014 - Trilha da Indústria_ TPC invitation}}
\includepdf[pages=-, scale=1,pagecommand=\thispagestyle{empty}]{\detokenize{GRUPO 2 – Atividades - Produção Científica Técnica Artística e Cultural/Sub-Grupo 2.1 - Produtividade de Pesquisa/D - Arbitragem de artigos técnico-científicos nacionais e internacionais na sua área de atuação/20140928_JEMS_CBSoft 2014 - Trilha da Indústria_ Home (Vinicius Garcia)}}
\includepdf[pages=-, scale=1,pagecommand=\thispagestyle{empty}]{\detokenize{GRUPO 2 – Atividades - Produção Científica Técnica Artística e Cultural/Sub-Grupo 2.1 - Produtividade de Pesquisa/D - Arbitragem de artigos técnico-científicos nacionais e internacionais na sua área de atuação/20140928_CBSoft 2014 - Trilha CFP Site}}

\newpage
\subsection{Arbitragem de Artigos Técnico-Científicos Nacionais e Internacionais na sua área de atuação}
\label{reviewer:2014-wtdsoft-cbsoft}
Esta subseção apresenta o comprovante de arbitragem de artigos técnico-científicos no IV Workshop de Teses e Dissertações (WTDSoft'2014) do Congresso Brasileiro de Software: Teoria e Prática (CBSoft'2014).
\includepdf[pages=-, scale=1,pagecommand=\thispagestyle{empty}]{\detokenize{GRUPO 2 – Atividades - Produção Científica Técnica Artística e Cultural/Sub-Grupo 2.1 - Produtividade de Pesquisa/D - Arbitragem de artigos técnico-científicos nacionais e internacionais na sua área de atuação/20140928_CBSoft 2014 - WTDSoft 2014_ Review invitation}}
\includepdf[pages=-, scale=1,pagecommand=\thispagestyle{empty}]{\detokenize{GRUPO 2 – Atividades - Produção Científica Técnica Artística e Cultural/Sub-Grupo 2.1 - Produtividade de Pesquisa/D - Arbitragem de artigos técnico-científicos nacionais e internacionais na sua área de atuação/20140928_JEMS_CBSoft 2014 - WTDSoft 2014_ Home (Vinicius Garcia)}}
\includepdf[pages=-, scale=1,pagecommand=\thispagestyle{empty}]{\detokenize{GRUPO 2 – Atividades - Produção Científica Técnica Artística e Cultural/Sub-Grupo 2.1 - Produtividade de Pesquisa/D - Arbitragem de artigos técnico-científicos nacionais e internacionais na sua área de atuação/20140928_CBSoft 2014 WTDSoft CFP Site}}

\newpage
\subsection{Arbitragem de Artigos Técnico-Científicos Nacionais e Internacionais na sua área de atuação}
\label{reviewer:2014-sbcars}
Esta subseção apresenta o comprovante de arbitragem de artigos técnico-científicos no Eighth Brazilian Symposium on Software Components, Architectures and Reuse (SBCARS 2014).
\includepdf[pages=-, scale=1,pagecommand=\thispagestyle{empty}]{\detokenize{GRUPO 2 – Atividades - Produção Científica Técnica Artística e Cultural/Sub-Grupo 2.1 - Produtividade de Pesquisa/D - Arbitragem de artigos técnico-científicos nacionais e internacionais na sua área de atuação/20140928_SBCARS 2014 invitation to program committee}}
\includepdf[pages=-, scale=1,pagecommand=\thispagestyle{empty}]{\detokenize{GRUPO 2 – Atividades - Produção Científica Técnica Artística e Cultural/Sub-Grupo 2.1 - Produtividade de Pesquisa/D - Arbitragem de artigos técnico-científicos nacionais e internacionais na sua área de atuação/20140928_SBCARS 2014 (PC member)}}
\includepdf[pages=-, scale=1,pagecommand=\thispagestyle{empty}]{\detokenize{GRUPO 2 – Atividades - Produção Científica Técnica Artística e Cultural/Sub-Grupo 2.1 - Produtividade de Pesquisa/D - Arbitragem de artigos técnico-científicos nacionais e internacionais na sua área de atuação/20140930_SBCARS_Proceedings}}

\newpage
\subsection{Arbitragem de Artigos Técnico-Científicos Nacionais e Internacionais na sua área de atuação}
\label{reviewer:2014-sbqs}
Esta subseção apresenta o comprovante de arbitragem de artigos técnico-científicos no Simpósio Brasileiro de Qualidade de Software (SBQS'2014).
\includepdf[pages=-, scale=1,pagecommand=\thispagestyle{empty}]{\detokenize{GRUPO 2 – Atividades - Produção Científica Técnica Artística e Cultural/Sub-Grupo 2.1 - Produtividade de Pesquisa/D - Arbitragem de artigos técnico-científicos nacionais e internacionais na sua área de atuação/20140808_SBQS 2014 - Trabalhos Técnicos_ TPC invitation}}
\includepdf[pages=-, scale=1,pagecommand=\thispagestyle{empty}]{\detokenize{GRUPO 2 – Atividades - Produção Científica Técnica Artística e Cultural/Sub-Grupo 2.1 - Produtividade de Pesquisa/D - Arbitragem de artigos técnico-científicos nacionais e internacionais na sua área de atuação/20140808_SBQS 2014 - Trabalhos Tecnicos_ Home (Vinicius Garcia)_JEMS}}


%---
\newpage
\subsection{Coordena\c{c}\~{a}o e/ou Participa\c{c}\~{a}o em Projetos Aprovados por \'{O}rg\~{a}os de Fomento}
\label{project:2015-facepe-pepe}
Esta subseção apresenta o comprovante de coordena\c{c}\~{a}o e/ou participa\c{c}\~{a}o no Projeto \textit{``BIGStore - Evolução da plataforma Ustore para Armazenamento, Manipulação e Experimentação de Grandes Volumes de Dados''} aprovado pela Fundação de Amparo à Ciência e Tecnologia do Estado de Pernambuco (FACEPE), Edital FACEPE 23/2014 - PESQUISADOR NA EMPRESA DE PERNAMBUCO (PEPE).
\includepdf[pages=-, scale=1,pagecommand=\thispagestyle{empty}]{\detokenize{GRUPO 2 – Atividades - Produção Científica Técnica Artística e Cultural/Sub-Grupo 2.1 - Produtividade de Pesquisa/E - Coordenação e ou participação em projetos aprovados por órgãos de fomento/2015-2018_Formulário SIN - BIGSTORE}}
\includepdf[pages=-, scale=1,pagecommand=\thispagestyle{empty}]{\detokenize{GRUPO 2 – Atividades - Produção Científica Técnica Artística e Cultural/Sub-Grupo 2.1 - Produtividade de Pesquisa/E - Coordenação e ou participação em projetos aprovados por órgãos de fomento/2015-2018_Edital_FACEPE_23-2014-PEPE_RESULTADO_Final}}
\includepdf[pages=-, scale=1,pagecommand=\thispagestyle{empty}]{\detokenize{GRUPO 2 – Atividades - Produção Científica Técnica Artística e Cultural/Sub-Grupo 2.1 - Produtividade de Pesquisa/E - Coordenação e ou participação em projetos aprovados por órgãos de fomento/2015-2018_BIGStore - Carta}}

\newpage
\subsection{Coordena\c{c}\~{a}o e/ou Participa\c{c}\~{a}o em Projetos Aprovados por \'{O}rg\~{a}os de Fomento}
\label{project:2014-citex-ufrpe}
Esta subseção apresenta o comprovante de coordena\c{c}\~{a}o e/ou participa\c{c}\~{a}o no Projeto \textit{``Análise das Tendências Tecnológicas para Computação em Nuvem e Redes de Longa Distância''} objeto do Termo de Execução Descentralizada número EME 14-119-00, tendo como partícipes o Departamento de Ciência e Tecnologia – DCT, por intermédio do Centro Integrado de Telemática do Exército (Repassadora) e a Universidade Federal Rural de Pernambuco (Recebedora).
\includepdf[pages=-, scale=1,pagecommand=\thispagestyle{empty}]{\detokenize{GRUPO 2 – Atividades - Produção Científica Técnica Artística e Cultural/Sub-Grupo 2.1 - Produtividade de Pesquisa/E - Coordenação e ou participação em projetos aprovados por órgãos de fomento/2014-2016_Projeto UFRPE e CITEx - Convenio}}
\includepdf[pages=-, scale=1,pagecommand=\thispagestyle{empty}]{\detokenize{GRUPO 2 – Atividades - Produção Científica Técnica Artística e Cultural/Sub-Grupo 2.1 - Produtividade de Pesquisa/E - Coordenação e ou participação em projetos aprovados por órgãos de fomento/2014-2016_Projeto UFRPE e CITEx - Plano de Trabalho assinado}}
\includepdf[pages=-, scale=1,pagecommand=\thispagestyle{empty}]{\detokenize{GRUPO 2 – Atividades - Produção Científica Técnica Artística e Cultural/Sub-Grupo 2.1 - Produtividade de Pesquisa/E - Coordenação e ou participação em projetos aprovados por órgãos de fomento/2014-2016_Projeto UFRPE e CITEx - Declaracao - Ceca}}

\newpage
\subsection{Coordena\c{c}\~{a}o e/ou Participa\c{c}\~{a}o em Projetos Aprovados por \'{O}rg\~{a}os de Fomento}
\label{project:2013-cnpq-485420-2013-9}
Esta subseção apresenta o comprovante de coordena\c{c}\~{a}o e/ou participa\c{c}\~{a}o no Projeto \textit{``Smart City Data Mediation''} aprovado pelo Conselho Nacional de Desenvolvimento Científico e Tecnológico (CNPq), Edital MCTI/CNPq Nº 14/2013.
\includepdf[pages=-, scale=1,pagecommand=\thispagestyle{empty}]{\detokenize{GRUPO 2 – Atividades - Produção Científica Técnica Artística e Cultural/Sub-Grupo 2.1 - Produtividade de Pesquisa/E - Coordenação e ou participação em projetos aprovados por órgãos de fomento/2013-2016_Smart City Data Mediation - CNPq}}
\includepdf[pages=-, scale=1,pagecommand=\thispagestyle{empty}]{\detokenize{GRUPO 2 – Atividades - Produção Científica Técnica Artística e Cultural/Sub-Grupo 2.1 - Produtividade de Pesquisa/E - Coordenação e ou participação em projetos aprovados por órgãos de fomento/2013-2016_Smart City Data Mediation - CNPq Termo}}

\newpage
\subsection{Coordena\c{c}\~{a}o e/ou Participa\c{c}\~{a}o em Projetos Aprovados por \'{O}rg\~{a}os de Fomento}
\label{project:2013-ustore-ustorage}
Esta subseção apresenta o comprovante de coordena\c{c}\~{a}o e/ou participa\c{c}\~{a}o no Projeto \textit{``Uma Abordagem para Indexação e Buscas Full-Text Baseadas em Conteúdo para Sistemas de Armazenamento em Nuvem''} apoiado pela Ustore.
%\includepdf[pages=-, scale=1,pagecommand=\thispagestyle{empty}]{\detokenize{GRUPO 2 – Atividades - Produção Científica Técnica Artística e Cultural/Sub-Grupo 2.1 - Produtividade de Pesquisa/E - Coordenação e ou participação em projetos aprovados por órgãos de fomento/2013-2016_Smart City Data Mediation - CNPq}}

\newpage
\subsection{Coordena\c{c}\~{a}o e/ou Participa\c{c}\~{a}o em Projetos Aprovados por \'{O}rg\~{a}os de Fomento}
\label{project:2013-facepe-ibpg-0719-7-03-12}
Esta subseção apresenta o comprovante de coordena\c{c}\~{a}o e/ou participa\c{c}\~{a}o no Projeto \textit{``Uma Arquitetura de Referência para Softwares Baseados em Máquinas Sociais''} aprovado pela Fundação de Amparo à Ciência e Tecnologia do Estado de Pernambuco (FACEPE), Edital FACEPE 17/2012.
\includepdf[pages=-, scale=1,pagecommand=\thispagestyle{empty}]{\detokenize{GRUPO 2 – Atividades - Produção Científica Técnica Artística e Cultural/Sub-Grupo 2.1 - Produtividade de Pesquisa/E - Coordenação e ou participação em projetos aprovados por órgãos de fomento/2013_2015_Facepe_MSc_IBPG-0719-1.03_12}}

\newpage
\subsection{Coordena\c{c}\~{a}o e/ou Participa\c{c}\~{a}o em Projetos Aprovados por \'{O}rg\~{a}os de Fomento}
\label{project:2012-fapesp}
Esta subseção apresenta o comprovante de coordena\c{c}\~{a}o e/ou participa\c{c}\~{a}o no Projeto \textit{``Uma Plataforma de Cidades Inteligentes baseada na Internet das Coisas''} aprovado pela Fundação de Amparo à Pesquisa do Estado de São Paulo (FAPESP).
\includepdf[pages=-, scale=1,pagecommand=\thispagestyle{empty}]{\detokenize{GRUPO 2 – Atividades - Produção Científica Técnica Artística e Cultural/Sub-Grupo 2.1 - Produtividade de Pesquisa/E - Coordenação e ou participação em projetos aprovados por órgãos de fomento/2012_2014_FAPESP_termo_de_outorga_-_processo-2012_10157-1}}
\includepdf[pages=-, scale=1,pagecommand=\thispagestyle{empty}]{\detokenize{GRUPO 2 – Atividades - Produção Científica Técnica Artística e Cultural/Sub-Grupo 2.1 - Produtividade de Pesquisa/E - Coordenação e ou participação em projetos aprovados por órgãos de fomento/2012_2014_FAPESP_Carta de participação - Vinicius Cardoso Garcia}}

\newpage
\subsection{Coordena\c{c}\~{a}o e/ou Participa\c{c}\~{a}o em Projetos Aprovados por \'{O}rg\~{a}os de Fomento}
\label{project:2012-facepe-phd}
Esta subseção apresenta o comprovante de coordena\c{c}\~{a}o e/ou participa\c{c}\~{a}o no Projeto \textit{``Um Ambiente como Serviço para Gerenciamento de Implantação Ágil de Aplicações na Nuvem''} aprovado pela Fundação de Amparo à Ciência e Tecnologia do Estado de Pernambuco (FACEPE), Edital FACEPE 11/2011.
\includepdf[pages=-, scale=1,pagecommand=\thispagestyle{empty}]{\detokenize{GRUPO 2 – Atividades - Produção Científica Técnica Artística e Cultural/Sub-Grupo 2.1 - Produtividade de Pesquisa/E - Coordenação e ou participação em projetos aprovados por órgãos de fomento/2012-2016_TermoOutorga_FACEPE_IBPG-0499-1.03_11_BolsaPhD}}

\newpage
\subsection{Coordena\c{c}\~{a}o e/ou Participa\c{c}\~{a}o em Projetos Aprovados por \'{O}rg\~{a}os de Fomento}
\label{project:2012-ustore-midiacenter}
Esta subseção apresenta o comprovante de coordena\c{c}\~{a}o e/ou participa\c{c}\~{a}o no Projeto \textit{``Nuvem Educacional Media Center: Uma Proposta de Solução para Distribuição de Conteúdos Educacionais''} apoiado pelo Ministério da Educação (MEC), Programa PROINFO.
\includepdf[pages=-, scale=1,pagecommand=\thispagestyle{empty}]{\detokenize{GRUPO 2 – Atividades - Produção Científica Técnica Artística e Cultural/Sub-Grupo 2.1 - Produtividade de Pesquisa/E - Coordenação e ou participação em projetos aprovados por órgãos de fomento/2013-2014_PROGRAMA PROINFO}}
\includepdf[pages=-, scale=1,pagecommand=\thispagestyle{empty}]{\detokenize{GRUPO 2 – Atividades - Produção Científica Técnica Artística e Cultural/Sub-Grupo 2.1 - Produtividade de Pesquisa/E - Coordenação e ou participação em projetos aprovados por órgãos de fomento/2013-2014_PROINFO - Comprovante Rendimento FNDE 2014}}

%---
\newpage
\subsection{Consultoria \`{a}s Institui\c{c}\~{o}es de Fomento \`{a} Pesquisa, Ensino e Extens\~{a}o}
\label{consulting:2015-facepe-pepe}
Esta subseção apresenta o comprovante de consultoria \`{a} Fundação de Amparo à Ciência e Tecnologia do Estado de Pernambuco (FACEPE) no Seminário de Avaliação Final dos Projetos Aprovados no Edital 10.2/2012 - PAPPE INTEGRAÇÃO 04a RODADA.
\includepdf[pages=-, scale=1,pagecommand=\thispagestyle{empty}]{\detokenize{GRUPO 2 – Atividades - Produção Científica Técnica Artística e Cultural/Sub-Grupo 2.1 - Produtividade de Pesquisa/F - Consultoria às instituições de fomento à pesquisa ensino e extensão/20151009_Declaração Avaliador PAPPE FACEPE - Vinicius Cardoso}}


%%%%%%%%%%%%%%%%%%%%%%%%%%%%%%%%%%%%%%%%%%%%%%%%%%%%%%%%%%%%%%%%%%%%%%%%%%%%%%%
% Subgrupo 2.2 - Produção Científica
%%%%%%%%%%%%%%%%%%%%%%%%%%%%%%%%%%%%%%%%%%%%%%%%%%%%%%%%%%%%%%%%%%%%%%%%%%%%%%%

% \newpage
% \subsection{Trabalhos Publicados em Peri\'{o}dicos Especializados do Pa\'{\i}s ou do Exterior}
% \label{journal:2014-ijca}
% Esta subseção apresenta o comprovante de publicação do artigo ``GUIDELINE PARA PRIORIZAÇÃO DE SOLICITAÇÕES CORRETIVAS URGENTES BASEADO EM METODOLOGIAS ÁGEIS PARA O CONTEXTO DE FÁBRICAS DE SOFTWARE ORIENTADAS A PRODUTO'' na REVISTA ELETRÔNICA ENG TECH SCIENCE, v. 3, p. 22-32, 2015.
% \includepdf[pages=-, scale=1,pagecommand=\thispagestyle{empty}]{\detokenize{GRUPO 2 – Atividades - Produção Científica Técnica Artística e Cultural/Sub-Grupo 2.2 - Produção Científica/}}

\newpage
\subsection{Trabalhos Publicados em Peri\'{o}dicos Especializados do Pa\'{\i}s ou do Exterior}
\label{journal:2015-revista-engtech}
Esta subseção apresenta o comprovante de publicação do artigo ``GUIDELINE PARA PRIORIZAÇÃO DE SOLICITAÇÕES CORRETIVAS URGENTES BASEADO EM METODOLOGIAS ÁGEIS PARA O CONTEXTO DE FÁBRICAS DE SOFTWARE ORIENTADAS A PRODUTO'' na REVISTA ELETRÔNICA ENG TECH SCIENCE, v. 3, p. 22-32, 2015.
\includepdf[pages=-, scale=1,pagecommand=\thispagestyle{empty}]{\detokenize{GRUPO 2 – Atividades - Produção Científica Técnica Artística e Cultural/Sub-Grupo 2.2 - Produção Científica/2015_engtech_76-219-1-PB}}

\newpage
\subsection{Trabalhos Publicados em Peri\'{o}dicos Especializados do Pa\'{\i}s ou do Exterior}
\label{journal:2015-ijepr}
Esta subseção apresenta o comprovante de publicação do artigo ``Is Brazilian Open Government Data Actually Open Data?: An Analysis of the Current Scenario'' no In: International Journal of E-Planning Research (IJEPR) 4 (2015): 2, doi:10.4018/ijepr.2015040104.
\includepdf[pages=-, scale=1,pagecommand=\thispagestyle{empty}]{\detokenize{GRUPO 2 – Atividades - Produção Científica Técnica Artística e Cultural/Sub-Grupo 2.2 - Produção Científica/2015_IJEPR - final version}}

\newpage
\subsection{Trabalhos Publicados em Peri\'{o}dicos Especializados do Pa\'{\i}s ou do Exterior}
\label{journal:2015-medinfo}
Esta subseção apresenta o comprovante de publicação do artigo ```(Br-SCMM) Brazilian Smart City Maturity Model: A Perspective from the Health Domain'' no Studies in Health Technology and Informatics, v. 216 (MEDINFO 2015: eHealth-enabled Health), p. 983, 2015, doi: 10.3233/978-1-61499-564-7-983.
\includepdf[pages=-, scale=1,pagecommand=\thispagestyle{empty}]{\detokenize{GRUPO 2 – Atividades - Produção Científica Técnica Artística e Cultural/Sub-Grupo 2.2 - Produção Científica/2015_SHTI216-0983}}

%%%%%%%%%%%%%%%%%%%%%%%%%%%%%%%%%%%%%%%%%%%%%%%%%%%%%%%%%%%%%%%%%%%%%%%%%%%%%%%
% Grupo 3: Atividades de Extens\~{a}o
%%%%%%%%%%%%%%%%%%%%%%%%%%%%%%%%%%%%%%%%%%%%%%%%%%%%%%%%%%%%%%%%%%%%%%%%%%%%%%%

%%%%%%%%%%%%%%%%%%%%%%%%%%%%%%%%%%%%%%%%%%%%%%%%%%%%%%%%%%%%%%%%%%%%%%%%%%%%%%%
% Subgrupo 3.1 - Coordenação e Orientação
%%%%%%%%%%%%%%%%%%%%%%%%%%%%%%%%%%%%%%%%%%%%%%%%%%%%%%%%%%%%%%%%%%%%%%%%%%%%%%%

%%%%%%%%%%%%%%%%%%%%%%%%%%%%%%%%%%%%%%%%%%%%%%%%%%%%%%%%%%%%%%%%%%%%%%%%%%%%%%%
% Subgrupo 3.2 - Coordenação de Eventos e Conferencista
%%%%%%%%%%%%%%%%%%%%%%%%%%%%%%%%%%%%%%%%%%%%%%%%%%%%%%%%%%%%%%%%%%%%%%%%%%%%%%%

%%%%%%%%%%%%%%%%%%%%%%%%%%%%%%%%%%%%%%%%%%%%%%%%%%%%%%%%%%%%%%%%%%%%%%%%%%%%%%%
% Grupo 4: Atividades de Forma\c{c}\~{a}o e Capacita\c{c}\~{a}o Acad\^{e}mica
%%%%%%%%%%%%%%%%%%%%%%%%%%%%%%%%%%%%%%%%%%%%%%%%%%%%%%%%%%%%%%%%%%%%%%%%%%%%%%%

%%%%%%%%%%%%%%%%%%%%%%%%%%%%%%%%%%%%%%%%%%%%%%%%%%%%%%%%%%%%%%%%%%%%%%%%%%%%%%%
% Grupo 5: Atividades Administrativas
%%%%%%%%%%%%%%%%%%%%%%%%%%%%%%%%%%%%%%%%%%%%%%%%%%%%%%%%%%%%%%%%%%%%%%%%%%%%%%%

%%%%%%%%%%%%%%%%%%%%%%%%%%%%%%%%%%%%%%%%%%%%%%%%%%%%%%%%%%%%%%%%%%%%%%%%%%%%%%%
% \newpage
% \subsection{Portaria de Progressão}
% \label{app:2014-portaria-progressao}
% \includepdf[pages=-, scale=1,pagecommand=\thispagestyle{empty}]{\detokenize{GRUPO 1/20141205_Portaria-de-Progressao-Funcional_5929-2014.pdf}}


\end{document}

%%% EOF
